% !TEX encoding = UTF-8
% !TEX TS-program = pdflatex
% !TEX root = apprendimento_automatico.tex
% !TEX spellcheck = it-IT

\section{Lezione 16 - Naive Bayes}

Questa tipologia di classificatore risulta essere una delle tecniche più semplici e popolari, che in alcuni casi fornisce delle prestazioni simili a quello offerte dalle reti neurali e SVM.

Conviene utilizzare questo classificatore quando:

\begin{itemize}
	\item Si ha a disposizione un elevato numero di dati;
	\item Gli attributi che descrivono le istanze sono condizionalmente indipendenti data la classificazione.
\end{itemize}

Il caso d'uso più famoso per questo classificatore è la categorizzazione di documenti testuali basata sulla frequenza delle parole.

\subsection{Funzionamento del classificatore}

Si ha a disposizione un training set composto da $(x_i, y_i)$, dove $x_i$ sono le istanze sulle quali fare apprendimento e $y_i$ è l'etichetta dell'istanza.

Ogni istanza $x_i$ è descritta da una tupla di attributi $<a_1, \ldots, a_n$ e le possibili etichette $y_i$ sono dei valori $v_j \in V$.

La funzione che si vuole apprendere è quindi $f :  X \rightarrow V $ e questo viene fatto ritornando l'etichetta più probabile:

$$
v_{MAP} = argmax_{v_j in V} P(v_j | a_1 , \ldots, a_n)
$$ 

ovvero l'etichetta che ha probabilità a posteriori maggiore data la descrizione dell'esempio.

Utilizzando la regola di Bayes si ottiene:

\begin{align*}
v_{MAP}  &= argmax_{v_j in V}\frac{ P( a_1 , \ldots, a_n | v_j)P(v_j)}{P( a_1 , \ldots, a_n)} \\
				&= argmax_{v_j in V} P( a_1 , \ldots, a_n | v_j)P(v_j)
\end{align*}

Il classificatore Naive Bayes assume che i valori degli attributi di una determinata istanza sono tra loro condizionalmente indipendenti data la classificazione, ovvero

$$
P(a_1,\ldots, a_n | v_j) = \prod_i P(a_i | v_j)
$$

con questa ipotesi il classificatore naive di Bayes diventa:

$$ 
v_{NB} = argmax_{v_j \in V} P(v_j)\prod_i P(a_i | v_j)
$$

\subsection{Apprendimento del classificatore}

L'apprendimento del classificatore consiste nello stimare i valori per $P(v_j)$ e $P(a_i | v_j)$ ed è possibile farlo andando a contare le occorrenze all'interno del training set.

\begin{enumerate}
	\item Per ogni valore target $v_j$:
		\begin{enumerate}
			\item $\hat{P}(v_j) \leftarrow$ stima di $P(v_j)$
			\item Per ogni valore $c_k$ di ogni attributo $a_i$:
				\begin{enumerate}
					\item $\hat{P}(c_k | v_j) \leftarrow$ stima di $P(c_k | v_j)$
				\end{enumerate}
		\end{enumerate}
\end{enumerate}

La classificazione viene poi effettuata usando le probabilità stimate.

Rispetto agli altri algoritmi di apprendimento finora visti, non viene effettuata una ricerca nello spazio delle ipotesi, ma viene costruita stimando le probabilità delle varie etichette.


\subsection{Considerazioni aggiuntive}

L'assunzione dell'indipendenza condizionale utilizzata dal Naive Bayes non sempre viene rispettata, tuttavia questo non influisce sulle prestazioni del classificatore.

Oltretutto, le stime effettuate utilizzando il training set non è necessario che siano precise, basta che valga la relazione:

$$ 
argmax_{v_j \in V} \hat{P}(v_j)\prod_i \hat{P}(a_i | v_j) = argmax_{v_j \in V} P(v_j)\prod_i P(a_i | v_j)
$$

Durante la stima delle probabilità può succedere che nessun esempio del training set sia classificato con l'etichetta $v_j$ ed abbia l'attributo $a_i = c$.
Se questo succede, la probabilità a posteriori stimata risulta essere 0 e la classificazione $v_j$ risulterà essere sempre improbabile.

$$ \hat{P}(c | v_j) = 0 \text{ e } \hat{P}(v_j)\prod_i \hat{P}(a_i | v_j) = 0$$

Una soluzione tipica è quella di andare a prendere in considerazioni degli \textbf{esempi virtuali} che non compaiono nel training set, ovvero:

$$
\hat{P}(c | v_j) = \frac{n_k + mp}{n +m}
$$

dove:

\begin{itemize}
	\item $n$ è il numero di esempi di apprendimento con $v = v_j$;
	\item $n_k$ è il numero di esempi di apprendimento con $v = v_j \text{ e } a_i = c$;
	\item $m$ è il numero di esempi virtuali, detto anche \textbf{equivalent sample size};
	\item $p$ è una stima per la probabilità $\hat{P}(c | v_j)$ nota a priori, che può essere influenza da delle conoscenze sul dominio oppure può essere uniforme per tutti i possibili valori dell'attributo.
\end{itemize}









