% !TEX encoding = UTF-8
% !TEX TS-program = pdflatex
% !TEX root = ../apprendimento_automatico.tex
% !TEX spellcheck = it-IT
\section{Lezione 2 - Ripasso di probabilità}\label{lezione-2---ripasso-di-probabilituxe0}

\subsection{Problemi tipici in modo matematico}\label{problemi-tipici-in-modo-matematico}

Viene utilizzata la seguente notazione:

\begin{itemize}
\item $\Theta$: insieme di parametri sul quale effettuare l'apprendimento.
\item $X$: insieme di dati sui quali applicare l'algoritmo.
\item $Y$: etichette da assegnare ai dati.
\end{itemize}

\begin{itemize}
\item \textbf{Classificazione binaria}: $h(\Theta): X \rightarrow \{-1,+1\}$ funzione
  che mappa un dato valore in -1 o +1 (oppure 0 o 1) la funzione
  \emph{h} è sempre parametrica, in quanto i parametri rappresentano
  l'apprendimento;
\item \textbf{Classificazione multiclasse}: $h(\Theta): X \rightarrow Y$ funzione che associa ad un determinato valore in $X$ un'etichetta in $Y$;
\item \textbf{Regressione}: $h(\Theta): X \rightarrow {\rm I\!R}$
\item \textbf{Ranking di istanze e classi}: $h(\Theta): X \times Y \rightarrow {\rm I\!R}$ dati elementi del
  prodotto cartesiano tra \textit{X} (esempio) e \textit{Y} (etichetta) associa un
  punteggio espresso da un numero reale. Una funzione che valuta la
  coppia $(x,y)$ con \emph{x} valore e \emph{y} classificazione.
\item \textbf{Novelty detection}: $h(\Theta): X \rightarrow [0,1]$  funzione che dato un esempio calcola il fattore di rischio come numero reale da 0 a 1.
\item \textbf{Clustering}: $h(\Theta): X \rightarrow \{1,\ldots1,k\}$ funzione che ad un esempio associa una valutazione.
\item \textbf{Associazioni (Basket Analysis)}: $P(Y|X)$
\end{itemize}

\subsection{Ripasso di probabilità e statistica}\label{ripasso-di-probabilituxe0-e-statistica}
\todo{sistemare}
\emph{Evento}: qualcosa che può essere o vero o falso.

La probabilità che si verifichi un'evento è un numero compreso tra 0 e
1, \texttt{0\ \textless{}=\ P(E)\ \textless{}=\ 1}. Questo numero può
essere calcolato usando la frequenza con la quale si verifica l'evento.

Dato un insieme di eventi E\_i mutuamente esclusivi tra loro. La
probabilità dell'unione di tutti gli eventi è la somma delle probabilità
dei singoli eventi.

La probabilità che si verifichi un evento o il suo complementare è 1.
(sempre se gli eventi sono mutuamente esclusivi).

La probabilità dell'unione di due eventi non esclusivi è data dalla
probabilità che si verifichi uno o l'altro, meno la probabilità che si
verifichino entrambi contemporaneamente.

\texttt{P(E\ unito\ F)\ =\ P(E)+P(F)+P(E\ intersecato\ F)}

\textbf{Probabilità condizionale}: probabilità che l'evento E accada
sapendo che si è verificato l'evento F \texttt{P(E\textbar{}F)}.

L'evento E è indipendente da F se \texttt{P(E\textbar{}F)\ =\ P(E)}.

\texttt{P(E\ intersecato\ F)\ =\ P(E\textbar{}F)*P(F)\ =\ P(F\textbar{}E)*P(E)}

\textbf{Formula di Bayes}

\texttt{P(F\textbar{}E)\ =\ {[}P(E\textbar{}F)P(F){]}\ /\ P(E)}

Deriva dalla probabilità condizionata, sarà utile nella classificazioni
di tipo \emph{bayesiano} (non sono sicuro che sia scritto giusto).

Dato un insieme di eventi F\_i, tra loro esclusivi ed esasutivi (gli Fi
coprono tutti i possibili esiti, la propabilità dell'unione di tutti gli
F\_i è 1). Allora \texttt{E\ =\ unione\ su\ i\ (E\ intersecato\ F\_i)},
la probabilità di E è quindi uguale alla sommatoria della probabilità di
tutte le intersezioni.

Il tutto per arrivare a:

\texttt{P(F\_i\ \textbar{}\ E)\ =\ {[}P(E\ \textbar{}\ F\_i)P(F\_i){]}\ /\ sommatoria\ su\ j\ (\ P(E\textbar{}F\_j)P(F\_j))}

\textbf{Valore atteso}: detto anche media, con X e Y variabili
aleatorie.

\texttt{E{[}X{]}\ =\ sommatoria\ su\ i\ (x\_i\ *\ P(x\_i))}

\texttt{E{[}aX\ +\ b{]}\ =\ aE{[}X{]}\ +\ b}

\texttt{E{[}X\ +\ Y{]}\ =\ E{[}X{]}\ +\ E{[}Y{]}}

\texttt{E{[}g(X){]}\ =\ sommatoria\ su\ i\ (g(x\_i)\ *\ P(x\_i))}

\texttt{E{[}X\^{}n{]}\ =\ sommatoria\ su\ i\ ((x\_i)\^{}n\ *\ P(x\_i))}
detto anche n-esimo momento

\textbf{Varianza}: quanto varia il valore ottenuto attorno alla media
dei vari esperimenti.

\texttt{sigma\^{}2\ =\ VAR(X)\ =\ E{[}\ (X-mu)\^{}2\ {]}} dove
\texttt{mu} è il valore atteso. \texttt{=\ E{[}X\^{}2{]}\ -\ mu\^{}2}.

\textbf{Deviazione standard}: o scarto quadratico medio, è la radice
quadrata della varianza, ed è la media di quando ci si discosta dal
valore attesso.
