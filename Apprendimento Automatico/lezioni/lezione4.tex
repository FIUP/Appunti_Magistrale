% !TEX encoding = UTF-8
% !TEX TS-program = pdflatex
% !TEX root = ../apprendimento_automatico.tex
% !TEX spellcheck = it-IT
\section{Lezione 4 - Laboratorio}\label{lezione-4---laboratorio}

Durante il corso useremo Python 2.7.x

Python è un linguaggio orientato agli oggetti e ognuno di questi è caratterizzato da:

\begin{itemize}
\item
  identità: è un identificativo dell'oggetto (!= puntatore).
\item
  tipo: rappresenta le operazioni che si possono fare con un oggetto,
  python è un linguaggio a tipizzazione dinamica e il tipo viene
  determinato a runtime.
\item
  valore: rappresenta il valore effettivo contenuto nell'oggetto.
\end{itemize}

In python non c'è il concetto classico di variabile, ma vengono usati
dei riferimenti.

\begin{lstlisting}[language=Python, caption=Codice Python]
x = 2
y = 3
y = x #y e x puntano allo stesso oggetto
\end{lstlisting}

La funzione \texttt{id()} permette di sapere l'identificatore di unoggetto.
Gli oggetti in Python sono immutabili.

Contenitori:

\begin{itemize}
\item
  liste
\item
  set (insiemi)
\item
  tuple
\item
  dizionari
\end{itemize}

Tutti questi contenitori possono essere eterogenei, una lista può tenere
sia numeri che stringhe contemporaneamente.

Le liste in python sono mutabili.

Un contenitore si dice iterabile se gli elementi possono essere iterati.

Un contenitore si dice sequenziale se è definita una sequenza di
elementi e può essere acceduto mediante indice (liste e tuple).

Un contenitere si dice associativo quando si comporta come un
dizionario, quindi solo i dizionari.

In Python non esitono i caratteri, esistono solo stringhe di lunghezzauno.

Gli indici per accedere ad una collezione con le \texttt{{[}{]}} possonoanche essere negativi, in questo caso si procede all'indietro.

\begin{lstlisting}[language=Python, caption=Indici delle liste]
>>> s = "Giacomo"
>>> s[3]
'c'
>>> s[-3]
'o'
>>> s[1:-3] #slicing
'iac'
\end{lstlisting}


\textbf{List comprehension}

\begin{lstlisting}[language=Python, caption=List comprehension]
>>> [x**2 for x in range(1,10)]
[1, 4, 9, 16, 25, 36, 49, 64, 81]
\end{lstlisting}

Permette di scrivere liste in modo compatto.

\textbf{operatore in}

\begin{lstlisting}[language=Python, caption=Utilizzo di in]
if k in dictiornary:
    # do something with k
\end{lstlisting}

\textbf{copy()}

\begin{lstlisting}[language=Python]
a = [1,2,3,4]
b = a              # b riferisce a 
c = a.copy()    # c è una copia di a (oggetto diverso)
\end{lstlisting}

\subsection{numpy}\label{numpy}

\begin{lstlisting}[language=Python, caption=Numpy]
>>> import numpy as np

>>> a = np.array([1,4,5,8], float)
>>> a
array([ 1.,  4.,  5.,  8.])
\end{lstlisting}

Questo modulo contiene alcuni metodi utili per la creazioni di matrici o
array.

Ad esempio, se la variabile \texttt{a} è una matrice è possibile utilizzare il metodo \texttt{a.transpose()} per ottenere la matrice trasposta.

Inoltre, se \texttt{a} e \texttt{b} sono delle matrici, è possibile effettuare le classiche operazioni aritmetiche in modo \textit{point wise}, ovvero elemento per elemento, con \texttt{a + b}, \texttt{a * b}, ecc.

\subsection{scipy}\label{scipy}

Libreria per la risolzione dei sistemi.

Anche questa ha un suo tipo per le matrici che è diverso da quello di
\texttt{numpy}.

Tra tipi \texttt{matrix} di \texttt{scipy} l'operazione \texttt{*} effettua il prodotto tra matrici.
