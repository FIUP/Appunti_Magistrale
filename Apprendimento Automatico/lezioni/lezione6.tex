% !TEX encoding = UTF-8
% !TEX TS-program = pdflatex
% !TEX root = ../apprendimento_automatico.tex
% !TEX spellcheck = it-IT
\section{Lezione 6 - Apprendimento di concetti}\label{lezione-6---apprendimento-di-concetti}

\subsection{Il concetto di Concetto}\label{il-concetto-di-concetto}

In uno spazio delle istanze \emph{X}, un \textbf{concetto} è una
funzione booleana su \emph{X}, cioè una funzione che prende in input un
oggetto dello spazio \emph{X} e ritorna un booleano che specifica se
l'elemento appartiene a quel concetto o meno.

Un concetto \emph{C} su uno spazio delle istanze \emph{X} viene definito
come una coppia \emph{(x,C(x))} con $x \in X$, e \emph{C(x)} è la
funzione concetto applicata ad \emph{x}.

Si dice che un ipotesi booleana \emph{h} per lo spazio delle istanze
\emph{X} \textbf{soddisfa} $x \in X$ se \emph{h(x) = 1}.

La stessa ipotesi \emph{h} si dice che è \textbf{consistente} con un
esempio \emph{(x,C(x))} se \emph{h(x) = C(x)}.

La definizione di consistenza può essere poi estesa ad un insieme se
l'ipotesi \emph{h} è consistente con tutti gli elementi presenti
nell'insieme.

\subsection{Ordine parziale}\label{ordine-parziale}

Siano $h_i$ e $h_j$ due funzioni booleane definite su uno spazio
delle istanze \emph{X}, diciamo che $h_i$ è \textbf{più generale} o
equivalente di $h_j$ ($h_i \geq_g h_j$) se:

$$
\forall x \in X,  h_j(x) = 1 \Rightarrow h_i(x) = 1
$$

Cioè tutti gli esempi che sono soddisfatti dall'ipotesi più specifica
sono sempre soddisfatti anche dall'ipotesi più generale.

Può essere che due ipotesi possono non essere comparabili tra loro.

\subsection{Find-S}\label{find-s}

Algoritmo che permette di trovare tra tutte le ipotesi, quella più
specifica e consistente con l'insieme di apprendimento.

Si parte da un training set \emph{Tr} e si inizializza \emph{h} con
l'ipotesi più specifica di tutte.

Per ogni istanza positiva \emph{x} del training set, cioè per tutti gli
esempi che appartengono al concetto, si modifica \emph{h} in modo che
riesca a soddisfare l'esempio \emph{x}.

Una volta terminate le istanze presenti nel training set viene ritornata
\emph{h}.

L'algoritmo parte dall'ipotesi più specifica possibile e man mano che
procede nell'analisi del training set la generalizzarla, in modo da
trovare la prima ipotesi consistente con il training set che sia il più
specifica possibile.

L'ipotesi più specifica di tutte è quella che rifiuta tutti i valori,
poi per ogni istanza del training set positiva, questa viene
generalizzata il meno possibile in modo che venga soddisfatta l'istanza
che si sta esaminando.

Bisogna notare che l'ipotesi più specifica non è sempre la migliore,
inoltre per funzionare bene il training set dovrebbe essere molto
grande.

\subsection{Candidate Elimination}\label{candidate-elimination}

\textbf{Version space}: sottoinsieme dello spazio \emph{H} contenete
solo ipotesi che sono consistenti con gli esempi del training set. Per
essere contenuta nel version space, un'ipotesi deve essere più generale
o equivalente a quella ottenuta con Find-S.

Dal momento che Find-S ritorna solamente un ipotesi e non è detto che
quella ritornata sia l'ipotesi migliore per il training set è stato
proposto l'algoritmo Candidate Elimination che ritorna tutte le ipotesi
contenute nel version space.

\textbf{Confine più specifico}: \emph{S}, insieme delle ipotesi \emph{s}
in \emph{H}, consistenti con il traingin set e tali che non esistano
altre ipotesi consistenti e più specifiche.

\textbf{Confine più generale}: \emph{G}, insieme delle ipotesi \emph{g}
in \emph{H}, consistenti nel training set e tali che non esistano altre
ipotesi più generali che siano consistenti con il training set.

Il version space è quindi contenuto tra i due confini, cioè contiene
tutte quelle ipotesi più generali di quelle contenute in \emph{S} e meno
generali di quelle contenute in \emph{G}, \emph{S} e \emph{G} inclusi.

\subsubsection{Algoritmo}\label{algoritmo}

Si inizializzano gli insiemi \emph{G} e \emph{S} in modo che contengano
rispettivamente le ipotesi più generali e più specifiche.

\begin{lstlisting}[language=python, caption=Canditate Eliminatio]
foreach d = (x,c(x)) in Tr do
    if c(x) == 1
        # rimuovi da G ogni ipotesi inconsistente con d
        # per ogni ipotesi s in S e inconsistente con d
            # rimuovi s da S.
            # aggiungi ad S tutte le generalizzazioni minime h di s tali che sono consistenti con d ed esiste un altra ipotesi g in G più generale di h.
            # rimuovi da S tutte le ipotesi s' che sono più generali di altre ipotesi in S.
    if c(x) == 0
        # rimuovi da S tutte le ipotesi inconsistenti con d
        # per ogni ipotesi g in G inconsistente con d
            # rimuovi g da G
            # aggiungi a G tutte le specializzaioni minime h di g tali che siano consistenti con d e che esiste un'altra ipotesi s in S più specifica di h.
            # rimuovi da G tutte le ipotesi g' che sono più specifiche di altre ipotesi in G.
\end{lstlisting}

Quando viene trovato un esempio \emph{d} nel training set che soddisfa
il concetto che si cerca di apprendere:

\begin{itemize}
\item
  Vengono rimosse da \emph{G} e da \emph{S} tutte le ipotesi che sono
  inconsistenti con \emph{d}, questo perché il version space deve
  contenere solo ipotesi consistenti con il training set.
\item
  Per ogni ipotesi \emph{s} rimossa da \emph{S} vengono aggiunte tutte
  le generalizzazioni minime di \emph{s} che sono in grado di soddisfare
  \emph{d}, questo per andare a definire delle nuove ipotesi specifiche
  e consistenti con il \textit{Tr}.
\item
  Vengono poi rimosse tutte le ipotesi \emph{s'} da \emph{S} che sono
  più generali di altre ipotesi presenti in \emph{S}, così facendo
  \emph{S} conterrà sempre e solo le ipotesi più specifiche.
\end{itemize}

Se \emph{d} non soddisfa il concetto viene applicato lo stesso
scambiando i due insiemi.
