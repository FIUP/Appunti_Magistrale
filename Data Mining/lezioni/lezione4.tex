\chapter{Laboratorio}

\section{Un po' di cose su R}\label{un-po-di-cose-su-r}

\begin{itemize}
\item
  Tutti gli oggetti sono vettori
\item
  \texttt{ls()} per vedere le variabili disponibili
\item
  \texttt{x\ \textless{}-\ c(2,3,4,5)} crea un vettore con 1,2,3,4.
\item
  notazione \texttt{{[}1:20{]}} per un vettore con la successione da 1 a
  20
\item
  \texttt{xx\ \textless{}-\ seq(from=100,\ to=1)} crea sempre una
  sequenza di numeri, con parametro opzionale \texttt{by} per
  specificare lo step
\item
  \texttt{rep(2,5)} crea un vettore con 5 elementi uguali a 2
\item
  \texttt{a\ \textless{}-\ c(rep(2,3),4,5,rep(1,5))},
  \texttt{a\ =\ 2\ 2\ 2\ 4\ 5\ 1\ 1\ 1\ 1\ 1}
\item
  \texttt{2*x} esegue il prodotto scalare
\item
  \texttt{length(x)} per la lunghezza del vettore
\item
  \texttt{max(x)} e \texttt{min(x)}
\item
  \texttt{sum(x)} che ritorna un vettore di un solo elemento con la
  somma
\item
  \texttt{mean(x)}, \texttt{var(x)}, \texttt{range(x)}
\item
  \texttt{x{[}7{]}} per estrarre il settimo elemento di \texttt{x},
  l'indice credo parta da 1
\item
  \texttt{x{[}-4{]}} ritorna un vettore senza il quarto elemento
\item
  \texttt{x\ \textless{}-\ matrix(c(2,3,5,7,11,13),nrow\ =\ 3)} crea una
  matrice con gli elementi specificati e 3 righe. Alternativamente è
  possibile specificare anche il numero di colonne.
\item
  \texttt{x2\ \textless{}-\ scan("nome\ file",\ sep="")} con
  \texttt{sep} opzionale, per caricare il contenuto di un file in un
  vettore, per caricare una matrice
  \texttt{x2\ \textless{}-\ matrix(scan(...),\ ncol\ =\ 3,\ byrow=TRUE}.
\item
  \texttt{str(x)} specifica la struttura dell'oggetto
\item
  \texttt{dim(x)} ritorna la dimensione di una matrice, se invocato con
  un vettore ritorna \texttt{NULL}.
\item
  \texttt{x{[}18,{]}} per ottenere la 18-esima riga di una matrice
\item
  \textbf{Dataframe}: matrice le cui colonne possono avere formati
  diversi
\item
  \texttt{ciliegi\ \textless{}-\ read.table("nome\ file")}.
\item
  \texttt{names(ciliegi)} è il vettore con i nomi delle colonne del
  dataframe
\item
  \texttt{names(ciliegi)\ \textless{}-\ c("diametro",\ "altezza",\ "volume")}
  permette di impostare il nome delle colonne, può anche essere
  specificato come parametro opzionale \texttt{col.names} della
  funzione \texttt{read.table}.
\item
  \texttt{summary(ciliegi)} fornisce degli indicatori per ciascuna
  colonna
\item
  \textbf{Mediana}: elemento centrale di una distribuzione ordinata in
  senso crescente, \textbf{primo e terzo quartile}: generalizzazione
  della mediana, rispettivamente l'elemento che sta al 25 e 75 per cento
  della distribuzione. La differenza tra i due quartili da l'idea di
  quanto è variabile la distribuzione.
\item
  I dataframe possono essere acceduti anche con il nome della colonna
  \texttt{ciliegi\$volume}.
\item
  \textbf{attach di un file}: aggiungere al workspace un oggetto, ovvero
  \texttt{attach(ciliegi)} permette di accedere al nome della colonna
  direttamente utilizzando \texttt{volume}. Come complementare c'è il
  comando \texttt{detach}.
\item
  \texttt{hist(diametro)} crea l'istogramma per il diametro
\item
  \texttt{help(hist)} per avere l'help di una funzione
\item
  l'istrogramma che viene generato di default può contenere dei buchi,
  conviene quindi adattare il numero di colonne utilizzando il parametro
  \texttt{breaks}
\item
  \texttt{boxplot(diametro)} fornisce il box plot di un valore, è un
  grafico che rappresenta la mediana, i quartili e il 5 e 95\%. Risulta
  più espressivo dell'istogramma. L'ampiezza della scatola rappresenta
  la variabilità dei dati.
\item
  \texttt{ciliegi{[}altezza\textgreater{}80,{]}} prende tutti i ciliegi
  con altezza maggiore di 80.
\item
  \texttt{library(MASS)} permette di caricare la libreria MASS
\item
  \texttt{search()} permette di visualizzare la lista degli ottetti in
  cui R va a cercare quando deve eseguire un comando
\item
  Gli attributi qualitativi vengono trattati come tipo Factor
\item
  \texttt{table(painters\$School)} crea la tabella con le frequenze
  delle varie qualità
\item
  \texttt{barplot(..)} fa il plot delle barre per una variabile discreta
\item
  \texttt{pie(...)} fa il grafico a torta, anche se è sconsigliabile
  utilizzare un grafico a torta perché per l'occhio umano fa fatica a
  vedere la differenza tra gli angoli.
\item
  come scale colori si possono utilizzare \texttt{heat.colors(k)},
  \texttt{rainbow(k)}, \ldots{}
\item
  \texttt{plot(x,y)} disegna un diagramma di dispersione, il parametro
  \texttt{pch} specifica il tipo di carattere, \texttt{pch=16}
  rappresenta i pallini pieni, \texttt{col} specifica il colore da
  utilizzare, possono indicare \texttt{col=painter\$School} per far
  variare il colore in base al valore dell'attributo quantitativo
\end{itemize}
