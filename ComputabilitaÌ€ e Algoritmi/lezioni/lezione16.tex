\subsubsection{Calcolabilità della minimalizzazione}\label{calcolabitliuxe0-della-minimalizzazione}

Se $ f : \mathbb{N}^{k+1} \rightarrow \mathbb{N} $ è in $ \mathcal{C} $, allora anche
$\mu y.f(\vec{x},y)$ è in $ \mathcal{C} $.

\paragraph{Dimostrazione}\label{dimostrazione}

Sia  $ f : \mathbb{N}^{k+1} \rightarrow \mathbb{N} $ in $ \mathcal{C} $ e sia \textit{P} il programma in
forma normale che calcola \textit{f}.

L'idea è quella di eseguire \emph{P} incrementando via via un contatore e, quando viene trovato un valore del contatore che azzera la funzione, questo viene ritornato.

\begin{verbatim}
 1       k         m+1          m+k|m+k+1|m+k+2|
|\vec{x} | .....  |x_1|x_2|....|x_k|  0  |  0  |
\end{verbatim}

Sia $m = \max{\rho(P), k }$, il programma che calcola la minimalizzazione illimitata è:

\begin{lstlisting}[language=URM]
T([1..k], [m+1..m+k]) //Copio l'input
P[m+1, ..., m+k+1 -> 1]
J(1, m+k+2, END) #LOOP
S(m+k+1)
J(1,1,LOOP)
#END
\end{lstlisting}

Dal momento che è possibile trovare un programma che calcola la
minimalizzazione illimitata, questa è calcolabile.

\subsubsection{Hot-fix delle funzioni} \label{hotfix}
Data una funzione  $ f : \mathbb{N} \rightarrow \mathbb{N} $ tale che esiste  $ g : \mathbb{N} \rightarrow \mathbb{N} $ calcolabile e tale che

$$
\Delta = \{ x | f(x) \neq g(x) \}
$$

e finito.

Allora \emph{f} è calcolabile e può essere modificata in modo da
coincidere con \emph{g}.

\subsubsection{Funzioni finite}\label{funizioni-finite}

\textbf{Funzione finita} $ \Theta: \mathbb{N}^{k} \rightarrow \mathbb{N}  $è una funzione
finita quando è definita come:

$$
\Theta(\vec{x}) = \begin{cases}
y_1, &\text{ se } \vec{x} = \vec{x}_1 \\
y_2, &\text{ se } \vec{x} = \vec{x}_2 \\ 
\cdots \\
y_n, &\text{ se } \vec{x} = \vec{x}_n \\
\uparrow, &\text{ altrimenti}
\end{cases}
$$

Tutte le funzioni di questo tipo sono calcolabili

\paragraph{Dimostrazione}\label{dimostrazione-1}

(per semplicità è ridotta a funzione unarie)

\begin{align*}
\Theta &= \begin{cases}
y_1, &\text{ se } x = x_1 \\
y_2, &\text{ se } x = x_2 \\ 
\cdots \\
y_n, &\text{ se } x = x_n \\
\uparrow, &\text{ altrimenti}
\end{cases} \\
&= y_1 \cdot \overline{sg}(|x - x_1|) + \cdots + y_n \cdot \overline{sg}(|x - x_n|) +  \underbrace{\mu z.\prod\limits_{i=1}^{n}|x - x_i|}_{\text{funzione indipendente da } z}
\end{align*}

La minimalizzazione su \emph{z} è calcolabile ma risulta indefinita, perché non è possibile minimizzarla rispetto a \emph{z}, quindi anche la funzione $ \Theta $ risulta essere indefinita se tutti i valori sono diversi da $ x_1 \ldots x_n $.

\section{Tesi di Church}\label{tesi-di-church}

Ogni funzione è calcolabile tramite un procedimento effettivo se e solo
se è URM calcolabile.

Church non ha proprio detto questo, perché non c'era URM quando è stata
enunciata e ha utilizzato un modello alternativo: le funzioni parziali
ricorsive $ \mathcal{R} $ (G\"{o}edel).

La classe $ \mathcal{R} $ delle \textbf{funzioni parziali ricorsive} è la \textbf{minima} classe di funzioni che contiene:

\begin{enumerate}[(a)]
\item  zero: $ z(\vec{x}) = 0 $ per ogni \textit{x}
\item  successore: $ S(x) = x+1 $
\item  proiezioni: $ U_{i}^k (x_1 \ldots x_k) = x_i $
\end{enumerate}

ed è chiusa rispetto:

\begin{enumerate}
\item
  composizione generalizzata
\item
  ricorsione primitiva
\item
  minimalizzazione illimitata.
\end{enumerate}

Una classe di funzioni \emph{X} è \textbf{ricca} se contiene le funzioni
di base ed è chiusa rispetto alle 3 operazioni classiche.

C'è almeno una classe ricca, perché anche la classe ``tutte le
funzioni'' è ricca, anche se ha poco senso considerarla.

Si vuole quindi che $ \mathcal{R} $ sia contenuta in \emph{X} ricca e che anche
$ \mathcal{R} $ sia ricca.

Questo è possibile perché l'intersezione di due classi ricche è
ovviamente anch'essa ricca.

$ \mathcal{R} $ può essere definita come l'intersezione di tutte le classi di
funzioni ricche.

$$
\mathcal{R} = \bigcap_{X \text{ ricca}} X
$$

anche se precedentemente $ \mathcal{R} $ è stata definita come

$$
\mathcal{R} = \{ (a), (b), (c) \text{ con tutte le funzioni che si ottengono da queste utilizzando } 1,2,3 \}
$$

È dimostrabile che le due definizioni sono equivalenti, anche se non lo
dimostriamo.

Si può anche definire la classe $ \mathcal{PR} $ delle \textbf{funzioni
primitive ricorsive}: la minima classe che contiene solamente le funzioni
di base chiusa rispetto la composizione e la ricorsione primitiva.
Ovviamente $ \mathcal{PR} $ è più piccola di $ \mathcal{R} $.

\subsection{$ \mathcal{C} $ = $ \mathcal{R} $}\label{c-figa-r-figa}

È già stato dimostrato che $ \mathcal{C} $ è una classe ricca, quindi sicuramente
$ \mathcal{R} \subseteq \mathcal{C} $.

Resta da dimostrare che $ \mathcal{C} \subseteq \mathcal{R} $.

Sia \textit{f} in $ \mathcal{C} $, ovvero esiste un programma \emph{P} in forma normale
che la calcola $f = f_p^{(k)}$.

\begin{verbatim}
P= I_1 ... I_s

|x_1 ... x_k| 0 ... 0
\end{verbatim}

Supponiamo di avere le funzioni:

$$
C_{p}^1(\vec{x},y) = \begin{cases}
\text{contenuto di R1 dopo \textit{t} passi del programma se non è terminato} \\
\text{altrimenti ritorna il valore finale del registro}
\end{cases} 
$$

$$
J_p(\vec{x},t) = \begin{cases}
\text{la prossima istruzione da eseguire all'istante \textit{t} di } P(\vec{x}) \text{ se non è terminato}\\
0 \text{ altrimenti }
\end{cases}
$$

Si ha che entrambe le funzioni sono del tipo $\mathbb{N}^{k+1} \rightarrow \mathbb{N} $
e totali.

Se $f(x)\downarrow$ allora \emph{P} termina su $\vec{x}$ in un qualche
numero di passi

$$
t_0 = \mu t.J_p(\vec{x}, t)
$$

e

$$
f(\vec{x}) = C_{p}^1(\vec{x}, t_0) =  C_{p}^1(\vec{x}, \mu t.J_p(\vec{x}, t))
$$

Se invece $f(\vec{x})\uparrow$ si ha che

$$
\mu t.J_p(\vec{x}, t) = \uparrow
$$

e

$$
f(\vec{x}) = C_{p}^1(\vec{x}, t_0) =  C_{p}^1(\vec{x}, \mu t.J_p(\vec{x}, t))
$$

Ovvero la combinazione di queste funzioni riesce a descrivere un programma URM.

Resta da dimostrare che queste due funzioni sono contenute in $ \mathcal{R} $ e pertanto, dato che \textit{f} è la combinazione di queste funzioni, anche \textit{f} è in $ \mathcal{R} $.
