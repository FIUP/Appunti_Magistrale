% !TEX encoding = UTF-8
% !TEX TS-program = pdflatex
% !TEX root = computabilità e algoritmi.tex
% !TEX spellcheck = it-IT

Il programma \emph{P} sarà fatto da un certo numero di istruzioni che andranno a modificare determinati registri.

L'idea è quella di andare a codificare la configurazione di memoria $ <r_1 r_2 \ldots.> $ utilizzata dal programma con un numero

$$
x = \prod\limits_{i\geq 1} p_i^{r_i}
$$

Da notare che il prodotto non è infinito, perché una volta fissato il programma, il numero di registri che questo utilizza è limitato.

Per come è definita la codifica si ha che

$$
r_i = (x)_i \text{(esponente dell'\textit{i}-esimo numero primo nella decoposizione del numero \textit{x})}
$$

È quindi possibile definire una funzione $C_p: \mathbb{N}^{k+1} \rightarrow \mathbb{N}$, $C_p(\vec{x},t)$ = contenuto della memoria dopo \textit{t} passi e $J_p: \mathbb{N}^{k+1} \rightarrow \mathbb{N}$ con lo stesso significato di prima.

Le due funzioni possono essere raccolte nella coppia

$$
\vartheta_p = \Pi(C_p(\vec{x},t),J_p(\vec{x},t))
$$

che risulta definita per ricorsione primitiva con

\begin{align*}
C_p(\vec{x},0) &= \prod\limits_{i=1}^k p_{i}^{x_i} \\
J_p(\vec{x},0) &= 1
\end{align*}

Abbreviando con $C = C_p(\vec{x},t)$ e $J = J_p(\vec{x},t)$, si può definire il caso ricorsivo come

$$
C_p(\vec{x},t+1) = \begin{cases}
qt(p_{n}^{(C)_n}) &\text{ se } 1 \leq J \leq S \text{ e } I_J = Z(n) \\
p_n \cdot C &\text{ se } 1 \leq J \leq S \text{ e } I_J = S(n) \\
p_n^{(C)_m} \cdot qt(p_{n}^{(C)_n}) &\text{ se } 1 \leq J \leq S \text{ e } I_J = T(m,n) \\
C &\text{ altrimenti } 
\end{cases}
$$

Nel caso 1 viene fatto il quoziente del numero da azzerare in modo da azzerare l'esponente relativo al registro nella rappresentazione numerica della memoria.

Nel caso 2 viene fatta la moltiplicazione del numero primo associato al registro per aumentare di 1 l'esponente del numero primo associato al registro nella rappresentazione numerica della memoria.

Il caso altrimenti si verifica quando il programma termina oppure l'istruzione di salto punta fuori dal programma, in entrambi i casi la memoria del programma non cambia.
$$
J_p(\vec{x},t+1) \begin{cases}
J+1 &\text{ se } 1 \leq J < S \text{ e } (I_J = Z(n), S(n), T(m,n)) \text{ oppure } (I_J = J(m,n,q), (C)_m \neq (C)_n) \\
q &\text{ se } 1 \leq J < S \text{ e } I_J = J(m,n,q) \text{ e } (C)_m = (C)_n \text{ e } 1 \leq q < S \\
0 &\text{ altrimenti}
\end{cases}
$$

Nel caso 1, se l'istruzione non è salto, oppure è un salto ma con condizione falsa, l'istruzione \emph{t + 1} è quella successiva. 
Se il salto ha condizione è vera e l'istruzione a cui saltare è interna al programma, l'istruzione successiva è \emph{q}. 
Altrimenti se il programma è terminato oppure l'istruzione di salto è finita fuori dal programma viene ritornato 0.

Le due funzioni vengono quindi definite utilizzando una definizione ``per casi'' e combinando delle funzioni definite precedentemente con le operazioni di base (primitive ricorsive), quindi anche queste due funzioni sono in $\mathcal{PR}$ e quindi sono anche in $\mathcal{R}$.

Quindi anche

$$
f(\vec{x}) = C_P(\vec{x}, \mu t.J_P(\vec{x},t)) \in \mathcal{R}
$$

\section{Funzioni primitive ricorsive}\label{funzioni-primitive-ricorsive}

$\mathcal{PR}$ è la minima classe di funzioni che contiene le funzioni:

\begin{enumerate}[(a)]
\item  zero
\item  succ
\item  proiezione
\end{enumerate}

e chiusa rispetto a:

\begin{enumerate}
\item  composizione
\item  ricorsione primitiva
\end{enumerate}

Si ha che $URM_{for, while}$ definisce $\mathcal{C}_{for,while} = \mathcal{C} = \mathcal{R} $.

Se non viene utilizzato il \texttt{while} $\mathcal{C}_{for} = \mathcal{PR} \subsetneq \mathcal{R} = \mathcal{C}$

Questo perché senza il \texttt{while} possono essere calcolate solo le funzioni totali.

Uno si può chiedere se il \texttt{while} serve solo a fare confusione o se serve anche per calcolare funzioni totali. Ovvero

$$
\mathcal{PR} = \mathcal{R} \cap Tot \text{ oppure }\mathcal{PR} \subsetneq\mathcal{R}\cap Tot
$$

Questo si dimostra utilizzando la \textbf{funzione di Ackermann}:

\begin{align*}
\psi(0,y) &= y+1 \\
\psi(x+1, 0) &= \psi(x,1) \\
\psi(x+1, y+1) &= \psi(x, \psi(x+1,y))
\end{align*}

Gli argomenti delle chiamate ricorsive di questa funzione diminuisce secondo l'ordine lessicografico.

$$
(x,y) \leq_{lex} (x',y') \begin{cases}
\text{se } x < x'\\
\text{oppure } x = x' \text{ e } y < y'
\end{cases}
$$

Una caratteristica di questo ordine è che è \textbf{ben fondato}, ovvero non ha catene discendenti infinite. 

Più formalmente $(D, \leq)$ è ben fondato se non ha sequenze del tipo $d_0 > d_1 > d_2 > \ldots $.

Se la relazione d'ordine è ben fondata e ad ogni passo viene prodotta una configurazione strettamente maggiore, si ha la garanzia che la sequenza di invocazioni termini.

$$
\underbrace{(0,0)(0,1)(0,2) \ldots }_{\mathbb{N}}\underbrace{(1,0)(1,1)(1,2) \ldots }_{\mathbb{N}}\underbrace{(2,0)(2,1)(2,2) \ldots }_{\mathbb{N}}
$$

Ogni sottoinsieme non vuoto di $(\mathbb{N}^{2}, \leq_\{lex\})$ ha un minimo dato $X \subseteq \mathbb{N}^{2}$.

\begin{align*}
x_0 &= \min\{x \: | \: (x,y) \in X\} \\
y_0 &= \min\{y \: | \: (x_0, y) \in X\}
\end{align*}

Tutto questo per dire che la funzione di Ackermann ha una ricorsione sensata.

\textbf{Induzione ben fondata}: Dato $(D, \leq)$ è ben fondato e \emph{P} è una qualche proprietà. 

Se $(\forall d \in D (\forall d' < d P(d')) \Rightarrow P(d)) \Rightarrow \forall d \in D P(d)$. 
Questo ci serve per dimostrare le proprietà della funzione di Ackermann.

\subsection{Totalità della funzione di Ackermann}\label{totalituxe0-di}

$$
\forall (x,y) \in \mathbb{N}^2\:\:\: \psi(x,y)\downarrow
$$

\subsubsection{Dimostrazione}\label{dimostrazione}

Sia $(x,y) \in \mathbb{N}^2 $ tale che $\forall (x',y') <_{lex} (x,y)$, $\psi(x',y')\downarrow $

allora $\psi(x,y)\downarrow$

Ci sono vari casi
\begin{align*}
(x = 0)& \psi(x,y) = \psi(0,y) = y +1 = \downarrow \\
(x > 0, y = 0)& \psi(x,y) =\psi(x,0) = \underbrace{\psi(x-1,1)}_{<_{lex}} = \downarrow \text{ per ipotesi induttiva} \\
(x > 0, y > 0)& \psi(x,y) = \psi(x-1, \underbrace{\psi(x,y-1)}_{\text{per ipotesi induttiva è }\downarrow = u}) = \psi(x-1, u) = \downarrow \text{ per ipotesi induttiva} 
\end{align*}

\subsection{La funzione di Ackermann è calcolabile}\label{la-funzione-di-ackermann-uxe8-calcolabile}

Intuizione: per calcolare $\psi(x,y)$ uso $\psi(x',y')$ per un numero finito di coppie $ (x',y') $.

\textbf{Insieme valido}: $S \subseteq \mathbb{N}^{3}$, $(x,y,z) \in S \Rightarrow z = \psi(x,y)$ e se $(x,y,\psi(x,y)) \in S$, allora $ S $ contiene anche tutte le triple $(x',y',\psi(x',y'))$ necessarie per calcolare $\psi(x,y)$

Quindi $S \subseteq \mathbb{N}^{3}$ è valido se

\begin{itemize}
\item
  $ (0,y,z) \in S \Rightarrow z=y+1 $
\item
  $ (x+1, 0, z) \in S \Rightarrow (x, 1, z) \in S $
\item
  $ (x+1, y+1, z) \in S \Rightarrow \text{ esiste } u \in \mathbb{N} \text{ tale che } (x+1,y, u) \text{ e } (x,u,z) \text{ sono in } S $.
\end{itemize}

Si può quindi dimostrare che $\psi(x,y) = z$ se e solo se esiste un insieme di triple $S \subseteq \mathbb{N}^3$ valido e finito tale che $ (x,y,z) $ appartiene a \textit{S}.

Ogni tripla \emph{(x,y,z)} sappiamo che può essere codificata come un numero. 
Quindi l'insieme \emph{S} che contiene le varie triple può essere codificato come un insieme di numeri, il quale a sua volta può essere rappresentato come un numero, utilizzando i numeri primi.

Su questo è possibile definire

$$
\psi(x,y) = " \mu (S,z). S \text{ valido e }(x,y,z) \in S "
$$

e quindi è calcolabile. 
Non andiamo più nel dettaglio, ci basta questa dimostrazione a spanne.

\subsection{La funzione di Ackermann non è primitiva ricorsiva}

Per calcolare la funzione di Ackermann abbiamo utilizzato la minimalizzazione illimitata, ma non è detto che non ci siano altri modi per calcolarla.

Per dimostrare che $ \psi $ non è primitiva ricorsiva è necessario dimostrare che non è possibile calcolarla senza la minimalizzazione.

Indicando con $ \psi_x(y) = \psi(x,y) $ si ha che

\begin{align*}
\psi_{x+1}(y) &= \psi_x(\underbrace{\psi_{x+1}(y-1)}_{\psi_x(\psi_{x+1}(y-2))}) \\
						&\ldots \\
						&= \psi_{x}^y (\psi_{x+1}(0)) \\
						&= \psi_{x}^y(\psi(1)) = \psi_{x}^{y+1}(1)
\end{align*}

\begin{align*}
\psi_{0}(y) &= y+1 \\
\psi_{1}(y) &= \psi_{0}^{y+1}(1) = y+2 \sim y+1\\
\psi_{2}(y) &= \psi_{1}^{y+1}(1) = 2y+3 \sim 2y\\
\psi_{3}(y) &= \psi_{2}^{y+1}(1) \sim 2^y\\
\psi_{4}(y) &= \psi_{3}^{y+1}(1) \sim 2\overbrace{^{2^{2^{\ldots}}}}^{y}\\
\end{align*}

Il parametro \textit{y} specifica quindi il numero di iterazioni da fare. es: $ \psi_4(3)\approx 2^{2^{2^{16}}}$.

Sia $ f \in \mathcal{PR} $ e \textit{k} = \textit{\# numero di passi di ricorsione primitiva per} \textit{f}.
Allora $ P(\vec{x})\downarrow $ in \#passi $ < \psi_{k+1}(\max\{\vec{x}\})$.

Si riesce a dimostrare anche che $f(\vec{x}) < \psi_{k+2}(\max\{\vec{x}\})$, noi non lo facciamo, ma questo ci basta per dimostrare che la funzione di Akcermann non è primitiva ricorsiva.

Assumiamo per assurdo che $\psi \in \mathcal{PR}$, allora sia \textit{n} = \textit{\# numero di annidamenti di ricorsione primitiva nel programma per $ \psi $}.

$$
\psi(n+2, n+2) < \psi_{n+2}(\max\{n+2, n+2\}) = \psi_{n+2}(n+2) = \psi(n+2, n+2)
$$

Dal momento che questo numero è illimitato, non è possibile calcolarlo a priori con la ricorsione primitiva, è quindi necessario calcolarlo con la minimalizzazione illimitata.

Si ha quindi che

$$
\psi \in \mathcal{R} \cap Tot \text{ e } \psi \notin \mathcal{PR}
$$

quindi

$$
\mathcal{PR} \subsetneq \mathcal{R} \cap Tot
$$



























