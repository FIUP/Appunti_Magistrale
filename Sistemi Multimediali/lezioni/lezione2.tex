% !TEX encoding = UTF-8
% !TEX TS-program = pdflatex
% !TEX root = sistemi multimediali.tex
% !TEX spellcheck = it-IT
\chapter{Immagini}\label{lezione-2---immagini}

Le immagini possono essere rappresentate come degli array
bi-dimensionali.

\section{Colori delle immagini - RGB}\label{colori-delle-immagini---rgb}

Il modo più semplice per rappresentare un'immagine è quello di rappresentare l'intensità del colore.

Alternativamente è possibile campionare il valore dei colori utilizzando i tre colori primari utilizzando la scala \textbf{RGB}.

Tipicamente la retina umana riesce a distinguere 150 livelli di intensità di colore, quindi tipicamente viene utilizzato un byte per
rappresentare i colori. Così facendo il colore di un pixel in bianco e
nero richiede un solo byte, mentre utilizzando RGB servono 3 byte,
ovvero 24 bit. Dal momento che 24 bit è un numero sfigato in
informatica, si preferisce usare 4 byte, ovvero 32 bit, utilizzando i
bit extra per rappresentare maggiori soglie di colore, tipicamente il
verde, oppure utilizzare ulteriori 8 bit per rappresentare il canale
alpha, ovvero la trasparenza.

Storicamente i colori sono stai rappresentati anche con RGB a 16 bit,
utilizzando 5 bit per il rosso e il blu e 6 bit per il verde. Questa
codifica porta a dei problemi nel caso di immagini sfumate che vengono
rappresentate con delle bande che la retina riesce a distinguere. Al
giorno d'oggi questa codifica viene utilizzata per retro compatibilità
dal momento che non ci sono più i problemi computazionali che hanno
portato a quella codifica.

\subsection{Codifica Giallo-Ciano-Magenta}\label{codifica-giallo-ciano-magenta}

Quando l'immagini viene visualizzata utilizzando luce riflessa, come la
carta stampata, il colore che viene utilizzato serve per bloccare i
fotoni in modo che chi visualizza l'immagine percepisca il colore
corretto.

Segue quindi che il controllo del colore su carta stampata è
complementare a quello del monitor. Perciò è stata proposta la codifica
GialloCianoMagenta che permette di ottenere in forma complementare la
scala RGB.

Così facendo è possibile convertire RGB in GCM utilizzando 1-R, 1-G,
1-B. Ad esempio per ottenere una linea di colore rosso, viene stampata
una linea utilizzando il giallo e il magenta.

Tipicamente per rappresentare il nero viene utilizzando l'inchiostro
nero, questo per motivi pratici: meno sbavature e minor consumo di
inchiostro.

La rappresentazione utilizzata all'interno del calcolatore è
indifferente, tipicamente viene utilizzato RGB perché è stata la prima
proposta.

La codifica RGB prende il nome di \textbf{additiva}, perché si parte dal
nero, mentre la codifica GCM prende il nome di \textbf{detrattiva},
perché si parte dal bianco e si va a togliere colore.

\subsection{Codifica HSI}\label{codifica-hsi}

Per ottenere colori complessi, utilizzare RGB non è intuitivo, è stata
quindi proposta una la codifica HSI, ovvero \textbf{Hue Saturation
Intensity}, la quale viene rappresentata utilizzando un disco o un
trinagolo contenente tutte le possibili sfumature di colore
realizzabile.

Così facendo l'utilizzatore finale riesce a scegliere più agevolmente
il colore desiderato.

Dal disco/triangolo viene solamente scelto la sfumatura (\textbf{Hue} e
\textbf{Saturation}), per regolare l'intensità viene utilizzato uno
slider che regola la quantità di nero.

Nonostante ciò, la codifica interna al calcolatore è RGB e si basa
sempre su un sistema a 3 coordinate, indipendentemente dalla codifica
utilizzata.

\subsection{Colori rappresentabili}\label{colori-rappresentabili}

RGB permette di codificare tutti i colori, tuttavia non sempre è
possibile rappresentarli tutti, pertanto viene effettuata una proiezione
dei colori utilizzando solamente due dimensioni, fissando l'intensità
del colore al massimo.

C'è poi un limite fisico sui colori che possono essere rappresentati e
la maggior parte della volte è possibile rappresentare solamente un
sottospazio dei colori, che prende il nome di \textbf{gamut} e dipende
dal monitor o dalla carta utilizzata.

\textbf{immagine 1}

Tipicamente per limitare questo problema nel caso di applicazioni che
richiedono la massima fedeltà dei colori, vengono utilizzanti dei
\textbf{profili colori} che permettono di mappare i colori visualizzati
vengono resi il più fedele possibili ai colori teorici.

Così facendo, quando viene utilizzato un profilo colore per
rappresentare il bianco RGB assoluto (\#FFFFFF), il monitor in realtà
rappresenta un colore leggermente diverso (come \#FCFCFD) ma che per le
proprietà fisiche del monitor viene visualizzato come il bianco puro.

Tipicamente questa configurazione è determinata dal driver del
monitor/stampante e al massimo l'utente può scegliere quale dei
possibili profili utilizzare.

\section{Manipolazione delle
immagini}\label{manipolazione-delle-immagini}

Ci sono un'infinità di motivi per voler comprimere dal dimensione di
un'immagine e questo può essere fatto sia mantenendo la stessa qualità
(\textbf{loss-less}) che sacrificando informazioni poco importanti.

Ci sono tre tipologie principali di filtri:

\begin{itemize}
\tightlist
\item
  \textbf{Filtri puntuali}: sono delle funzioni che vengono applicate
  pixel per pixel, senza tenere conto di che cosa c'è attorno,
  tipicamente regolano il contrasto, la luminosità o la gamma
  correction.
\item
  \textbf{Filtri locali}: sono delle funzioni che lavorano su gruppi di
  pixel vicini. Vengono utilizzati per modificare i rapporti di forza
  che ci sono tra pixel vicini, come quando si vuole fare anti-aliasing,
  blur, sharpening, ecc. Tipicamente vengono applicati utilizzando una
  matrice kernel che scorre l'immagine e che definisce dei pesi per
  modificare i valori dei pixel.
\item
  \textbf{Filtri globali}: sono filtri che vengono applicati sulla
  totalità dell'immagine, ovvero considerano il rapporto del pixel su
  tutta l'immagine. Sono quelli più pesanti da applicare ma sono quelli
  più efficaci. Tipicamente utilizzano le varie trasformate, come
  Fourier, Coseno, e si basano su tanta matematica complessa.
  Fortunatamente ci sono già degli algoritmi molto efficaci per
  applicare questi filtri. Questi filtri sono quelli che permettono di
  trasformare notevolmente la dimensione dell'immagine, rendendo
  possibile comprimerla, eliminando i dati relativi a delle frequenze
  poco significative.
\end{itemize}

Tipicamente l'effetto dei filtri è speculare rispetto alla granularità
con la quale lo si applica, ovvero un filtro puntuale ha un effetto
globale su tutta l'immagine, mentre un filtro globale fa delle modifiche
fine che sono difficili da percepire.

\subsection{Filtri puntuali}\label{filtri-puntuali}

Sono i filtri che stanno alla base della manipolazione delle immagini.

\textbf{esempio del filtro che cambia la luminosità di un'immagine}

Il filtro di luminosità scansiona tutta l'immagine e modifica il valore
dei pixel, moltiplicandolo per un fattore di scala.

Il filtro per l'aumento del contrasto esegue due passate, nella prima
calcola l'istogramma delle luminosità dell'immagine e nella seconda
passata scala la luminosità del pixel in modo da aumentare la differenza
tra le parti più chiare e quelle più scure.

Il filtro di gamma correction modifica l'intensità del colore
utilizzando una curva esponenziale:

$$
I_{out} = I_{in}^{\gamma}
$$

Questo filtro risulta particolarmente importante perché i pixel del
monitor, così come le lampade, tendono ad avere una trasformazione
esponenziale dell'intensità. È quindi necessario pre-filtrare l'intensità
luminosa con una curva esponenziale complementare alla curva del
dispositivo in modo che quest'ultimo visualizzi l'immagine con
l'intensità luminosa corretta. Tipicamente questo filtro è integrato
nella regolazione del profilo colore di un monitor.

L'applicazione di un filtro non sempre è reversibile, ad esempio
modificando la luminosità dell'immagine può succedere che il nuovo
valore per il colore diventi superiore a 255, in questo caso viene
eseguito il \textbf{clamping} del valore e viene posto a 255. Quando
questo succede, vengono perse delle informazioni e se si applica il
filtro inverso non si ottiene l'immagine precedente.

\subsection{Filtri locali}\label{filtri-locali}

Sono quei filtri che producono un pixel di output a partire da una
matrice di pixel dell'immagine originale.

Ci sono due tipologie principali:

\begin{itemize}
\tightlist
\item
  \textbf{convoluzione}: utilizza le informazioni dei pixel adiacenti
  per modificare il valore di un singolo pixel (filtri a kernel).
  Tipicamente vengono utilizzati per lavorare sui particolari delle
  immagini.
\item
  \textbf{deformazioni/distorsione}: sono quei filtri ri-campionano
  l'immagine di partenza, riscalando o distorcendo la forma
  dell'immagine.
\end{itemize}

Un esempio di filtro deformazionale è quello che dimezza la
dimensione dell'immagine originale, producendo una nuova immagine
scartando un pixel ogni pixel. Lo stesso filtro può essere migliorato
utilizzando un interpolazione lineare per permettere di ridimensionare
l'immagine di un fattore qualsiasi utilizzando un passo di campionamento
diverso. Non c'è nessun vincolo nell'utilizzo dell'interpolazione
lineare, è possibile utilizzare anche quella quadratica o cubica e allo
stesso modo è possibile scegliere a piacere la dimensione dell'intorno
di pixel sul quale effettuare l'interpolazione. Le schede grafiche
riescono ad applicare questo filtro direttamente utilizzando l'hardware.

Per quanto riguarda i filtri su kernel, è possibile utilizzare il
kernel:

\[
 \begin{bmatrix}
 0 & -1 & 0 \\
 -1 & 5 & -1 \\
 0 & -1 & 0
 \end{bmatrix}
 \:\: =\:\:\:\:
 \begin{bmatrix}
 0 & -1 & 0 \\
 -1 & 4 & -1 \\
 0 & -1 & 0
 \end{bmatrix}
 \:\: + \:\:
 \begin{bmatrix}
 0 & 0 & 0 \\
 0 & 1 & 0 \\
 0 & 0 & 0
 \end{bmatrix}
\]

che in qualche modo approssima la discesa di gradiente sul campionamento
dell'immagine, per via della prima componente, la quale approssima il
valore della derivata seconda. Così facendo, nelle zone in cui c'è una
tonalità costante non modifica di molto, ma nelle zone in cui c'è una
netta variazione di intensità andrà ad aumentare la variazione, andando
quindi a rendere più visibile i particolari (\textbf{filtro di
sharpening}).

Per effettuare il blur dell'immagine (sfocatura) è necessario utilizzare
un kernel tipo

$$
 \begin{bmatrix}
 0 & 1 & 0 \\
 1 & 1 & 1 \\
 0 & 1 & 0
 \end{bmatrix}
$$

Un caso d'uso di questo kernel è nell'implementazione di un filtro
anti-aliasing.

Utilizzando come kernel quello sotto riportato è possibile evidenziare i
contorni degli elementi nell'immagine.

$$
\begin{bmatrix}
0 & -2 & 0 \\
-2 & 8 & -2 \\
0 & -2 & 0
\end{bmatrix}
$$

\subsection{Filtri globali}\label{filtri-globali}

Si tratta di filtri che prendono in considerazione tutti i pixel
dell'immagine e generano dei coefficienti che li approssimano.

Il problema principale di questi filtri è che sono computazionalmente
pesanti, l'approccio naive tipicamente richiede $O(n^4)$ sul
numero di pixel (larghezza o altezza).

Dei miglioramenti si ottengono utilizzando immagini con dimensione
espressa come potenza di due e dall'osservazione che è possibile
applicare separatamente il filtro prima sulle righe e poi sulle colonne,
sommando gli effetti. In tutto si riesce a scendere ad $O(n^2\log(n))$  e considerando che il numero totale di pixel è $O(n^2)$ il
risultato è già buono.

Il funzionamento di questi algoritmi parte dal campionamento del segnale
dato dall'immagine per ottenere un'approssimazione del segnale che
generato l'immagine utilizzando le trasformate discrete di Fourier.

Così facendo è possibile rappresentare le immagini utilizzando le
armoniche ottenute applicando Fourier (\textbf{spettro del segnale}) e
una caratteristica delle trasformate discrete è che per campionare
\emph{n} campioni servono \emph{n} armoniche.

Per rappresentare un'armonica sono necessari due coefficienti, uno per
la fase e uno per l'ampiezza, pertanto risultano utili per effettuare
le modifiche ma non per comprimere le immagini.

\subsubsection{Trasformata di Fourier}\label{trasformata-di-fourier}

Partendo da un'immagine di dimensione $L \times H$ la trasformazione con
Fourier produce una matrice sempre di dimensione $L \times H$ con ogni
cella che contiene due valori, l'ampiezza e la fase di un'armonica. La
totalità delle armoniche permette di ottenere la funzione per il calcolo
dei valori dei pixel.

Al centro della matrice ci sarà l'armonica che rappresenta la componente
media e man mano che ci si sposta verso i bordi si ottiene un'aumento
della frequenza.
