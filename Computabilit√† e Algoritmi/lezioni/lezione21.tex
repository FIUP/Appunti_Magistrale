% !TEX encoding = UTF-8
% !TEX TS-program = pdflatex
% !TEX root = computabilità e algoritmi.tex
% !TEX spellcheck = it-IT

\chapter{Esercizi in preparazione al parziale}


\section{Teorema SMN}

\subsection{Esercizio 1}

Dimostrare che esiste $ S : \mathbb{N} \rightarrow \mathbb{N}$ tale che $|\: W_{S(x)}\:|  = 2 x$ e $ |\: E_{S(x)}\:| =x $

\subsubsection{Soluzione}

Si cerca una funzione $ f(x,y) $ che con $ x $ fissato abbia esattamente le caratteristiche richieste, ci pensa il teorema SMN a fare il resto.

\begin{align*}
f(x,y) &= \begin{cases}
y \dotminus x, &\text{ se } y < 2x \\
\uparrow, &\text{altrimenti}
\end{cases} \\
 &= y\dotminus x + \underbrace{\mu z. y +1 -2x}_{\text{fa divergere se } y \geq 2x}
\end{align*}

Questa funzione è calcolabile e quindi per il teorema SMN esiste $ S : \mathbb{N} \rightarrow \mathbb{N} $ tale che 

$$
\phi_{S(x)}(y) = f(x,y)
$$

e per costruzione $W_{S(x)} = [0,2x-1]$ e $ E_{S(x)} = [0,x-1] $.

\section{Calcolabilità delle funzioni}

\subsection{Esercizio 1}

Dimostrare che 

$$
f(x) = \begin{cases}
\phi_x(x) &\text{se} \phi_x(x)\downarrow \\
0 &\text{altrimenti}
\end{cases}
$$

non è calcolabile.

\subsubsection{Soluzione}

Non si può usare il \textit{trick} della diagonale, perché la funzione deve essere proprio uguale alla diagonale.

$$ h(x) = f(x)+1 = \begin{cases}
\phi_x(x)  + 1 &\text{se} \phi_x(x)\downarrow \\
1 &\text{altrimenti}
\end{cases}
 $$

Così facendo $ h \neq \phi_x(x) \forall x $ e pertanto non è calcolabile.

Ma $ h $ è il successore di $ f $, quindi $ f $ non può essere calcolabile, perché se così non fosse $ h(x) = f(x) +1 $ sarebbe calcolabile per composizione di funzioni calcolabili.ò

Da notare che questa cosa non vale in se e solo se:

$$ h(x) = f(x) +1 \text{ non calcolabile }\Rightarrow f(x)  \text{ non calcolabile } $$

ma \textbf{NON \`{E} VERO}

$$ f(x) \text{ non calcolabile } \Rightarrow f(x) = h(x) +1 \text{ non calcolabile }$$
%------------

\subsection{Esercizio 2}

Sia $ \mathbb{P} $ l'insieme dei numeri pari, dimostrare che la funzione caratteristica è primitiva ricorsiva.

$$
\mathcal{X}_\mathbb{P} = \begin{cases}
1 &\text{ se $ x $ è pari} \\
0 &\text{ altrimenti}
\end{cases}
$$

\subsubsection{Soluzione}

\begin{align*}
\mathcal{X}_\mathbb{P}(0) &= 1
\mathcal{X}_\mathbb{P}(x+1) &= \overline{sg}\big(\mathcal{X}_\mathbb{P}(x)\big)
\end{align*}

Dal momento che in un esercizio non sappiamo dell'esistenza del segno negato è necessario andare a mostrare che anche quella funzione è primitiva ricorsiva:

\begin{align*}
 \overline{sg}(0) &= 1 \\
  \overline{sg}(x+1) &= 0
\end{align*}


\subsection{Esercizio 3}

Dimostrare che la funzione $ half(x) = \lfloor\frac{x}{2}\rfloor $ è primitiva ricorsiva.

\subsubsection{Soluzione}

\begin{align*}
half(0) &= 0 \\
half(x+1) &= half(x) + rm_2(x)
\end{align*}

come prima bisogna dimostrare che $ rm_2 $ è ricorsiva primitiva

\begin{align*}
rm_2(x) &= 0
rm_2(x+1) &= \overline{sg}(rm_2(x))
\end{align*}

ma anche $ \overleftarrow{sg} $ è in $ \mathcal{PR} $

\begin{align*}
\overline{sg}(0) &= 1 \\
\overline{sg}(x+1) &= 0
\end{align*}

\section{Macchina URM}

\subsection{Esercizio 1}

Consideriamo la macchina $ \text{URM}^- $ che al posto dell'istruzione successore ha quella che effettua il predecessore.
Come sono in relazione le due classi? 

Ovvero:
 $$\mathcal{C}^- \text{ ? } \mathcal{C}$$
 
 \subsubsection{Soluzione}
 
 \`{E} chiaro che  $\mathcal{C}^- \subseteq \mathcal{C}$ perché la funzione predecessore può essere sostituita da un programma che calcola il predecessore.
 
 Per vedere \textbf{se}  $\mathcal{C}^- \supseteq \mathcal{C}$ bisogna provare a simulare il successore con un programma per la macchina $ \text{URM}^- $.

Ma dato un $ P \in \text{URM}^- $ qualsiasi $ P(\vec{x}) $ dopo $ n $ passi, $ \forall n $, tutti i registri $ r_i \leq \max x_i$.

Se $ n = 0 $ è ovvio perché i registri non cambiano.

Se $ n \rightarrow n+1 $, l'istruzione aggiunta può essere:
\begin{itemize}
	\item precessore ma il valore non aumenta
	\item zero azzera un registro
	\item il trasferimento non incrementa i valori
\end{itemize}

Dal momento che i registri di un programma in $ \text{URM}^- $ non aumentano mai, non è possibile implementare la funzione successore, quindi \textbf{non vale} la relazione $\mathcal{C}^- \supseteq \mathcal{C}$.


\subsection{Esercizio 2}
$ \text{URM}^S$ che non ha \texttt{S(n)} e \texttt{J(m,n,t)} ma ha \texttt{JS(m,n,t)} che prima fa il test e poi incrementa il registro $ m $.

\subsubsection{Soluzione}

$\mathcal{C}^S \subseteq \mathcal{C}$ perché un programma URM riesce ad emulare l'istruzione \texttt{JS}.

$\mathcal{C}^S \supseteq \mathcal{C}$ non vale, perché un programma $ \text{URM}^S$ non riesce a calcolare la funzione successore.


\section{Diagonalizzazione}

\subsection{Esercizio 1}

$ F_0 = \{f \: | \: \text{cod}(f) \subseteq \{0\} \text{ e parziali}\}$ è numerabile?

\subsubsection{Soluzione}

L'unica differenza tra queste funzioni è data dai domini che sono tanti quanti i numeri naturali.

Suppongo per assurdo che $ F_0 $ sia enumerabile, allora:

\begin{verbatim}
		f0	f1 f2
0      0   0   0
1      0   0
2
\end{verbatim}

considerando la diagonale, posso definire

$$
f(x) = \begin{cases}
0 &\text{se } f_x(x)\uparrow \\
\uparrow &\text{se } f_x(x)=0
\end{cases}
$$

$ f $ dovrebbe appartenere a $ F_0 $ ma per costruzione non compare nella enumerazione, quindi $ F_0 $ non è enumerabile.

\subsection{Esercizio 2}

$ f : \mathbb{N} \rightarrow \mathbb{N} $ totale e decrescente se $ \forall x,y \: x \leq y \Rightarrow f(y) \leq f(x) $.

$ F_d = \{ f : \mathbb{N} \rightarrow \mathbb{N} \: \ \: f \text{ totale e descrescente e } \text{cod}(f) \subseteq \{0,1\} \} $ è numerabile?

\subsubsection{Soluzione}

Le funzioni dell'insieme possono essere o costanti uguali a 0 o 1, oppure uguali a 1 fino ad un certo valore per poi diventare costanti a 0.

$$ \forall f \in F_d \: n(f)  = \text{sup}(\{x \: | \: f(x) = 1\})$$

$ n(f) $ è il punto in cui cambia il valore della funzione e caratterizza $ f $, ovvero

$$
\forall f_1, f_2 \in F_d \: n(f_1) = n(f_2) \Rightarrow f_1 = f_2
$$

quindi $ n : F_d \rightarrow \mathbb{N} \cup \{\infty\} $ è una funzione iniettiva e suriettiva, pertanto $ |\: F_d \:| = |\: \mathbb{N} \cup \{ \infty \} \:| $, ovvero $ F_d $ è numerabile.


\subsection{Esercizio 3}

Trovare $ f : \mathbb{N} \rightarrow \mathbb{N} $ totale e non calcolabile tale che $ f(x)  =x $ per infiniti $ x $.

\subsubsection{Soluzione}

$$
f(x) = \begin{cases}
x, &\text{ se $ x $ è pari} \\
\phi_{\frac{x-1}{2}}(x) +1, &\text{ se } \phi_{\frac{x-1}{2}}(x) \downarrow \\
0, &\text{ se } \phi_{\frac{x-1}{2}}(x) \uparrow 
\end{cases}
$$

\subsection{Esercizio 4}

Trovare una funzione $ f $ crescente, totale e non calcolabile.

\subsubsection{Soluzione}

$$
h(x) = \begin{cases}
\phi_x(x)+1, &\text{ se }\phi_x(x)\downarrow \\
0, &\text{ se } \phi_x(x) \uparrow
\end{cases}
$$

$ h $ è non calcolabile perché è definita sulla diagonale distorta.

$$
f(x) = \sum\limits_{y \leq x} h(y)
$$

Così definita $ f $ è:
\begin{itemize}
	\item totale
	\item crescente: $ f(x+1) = f(x) + h(x+1) \geq f(x) $
	\item $ \forall x $ se $ \phi_x(x) \downarrow \: f(x) = \sum\limits_{y \leq x} h(y) \geq h(x) = \phi_x(x) +1 \neq \phi_x(x) $ e se $ \phi_x(x) \uparrow $, $ h(x) = 0 \neq \phi_x(x) $, ovvero $ f $ non è calcolabile.
\end{itemize}

\subsection{Esercizio 5}

Può esistere $ f : \mathbb{N} \rightarrow \mathbb{N} $ non calcolabile tale che $ \forall g : \mathbb{N} \rightarrow \mathbb{N} $ non calcolabile $ f+g  $ è calcolabile?

$$
(f+g)(x) = f(x) + g(x)
$$

\subsubsection{Soluzione}

Non può esistere, perché per assurdo, sia $ f $ una funzione tale che $ (f+f) $ è calcolabile:

\begin{align*}
h(x) &= f(x) + f(x) \\
		&= 2 f(x)
\end{align*}

$h(x)$ è quindi calcolabile, però utilizzando questa funzione si può definire $ f(x) = qt(2, h(x)) $, dimostrando la calcolabilità di $ f $ che per ipotesi non è calcolabile.


