% !TEX encoding = UTF-8
% !TEX TS-program = pdflatex
% !TEX root = computabilità e algoritmi.tex
% !TEX spellcheck = it-IT
\chapter{Lezione 1}
\section{Computabilità e
Algoritmi}\label{lezione-1---computabilituxe0-e-algoritmi}

Il lunedì e il martedì ci sarà Computabilità, mentre Mercoledì e Giovedì
c'è Colussi con la parte di algoritmi.

Ci sono dei problemi che non possono essere risolti in modo algoritmico,
come la terminazione o la prova di correttezza di un programma.

In computabilità, non viene preso in considerazione il consumo di
risorse, in modo che le dimostrazioni effettuate siano indipendenti dal
modello di calcolo adottato.

I problemi appartengono a varie classi:

\begin{itemize}
\tightlist
\item
  \textbf{P}: problemi che possono essere risolti da un algoritmo in
  tempo polinomiale
\item
  \textbf{NP}: problemi che possono essere risolti in tempo polinomalie
  ma in modo non deterministico
\item
  \textbf{EXP}: problemi che possono essere risolti da un algoritmo in
  tempo esponenziale
\end{itemize}

\subsection{L'informatica e la
computabilità}\label{linformatica-e-la-computabilituxe0}

\emph{Computer science is no more about computers tha astronomy is about
telescopes. Dijkstra}.

L'idea dell'informatica nasce dalla logica, ricercando un procedimento
generale (macchina) su base combinatoria per trovare tutte le verità.

Libro: \emph{Nigel Cutland ``Computability. An Introduction to Recursive
Function Theory'' Cambridge University Press}.
