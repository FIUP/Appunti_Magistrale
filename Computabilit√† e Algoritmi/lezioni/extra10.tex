\section{Appello 2014-04-03}

\subsection{Esercizio 1}

Dimostrare il teorema di struttura dei predicati semidecidibili.

\subsubsection{Soluzione}

Se $Q(x,y)$ è decidibile, allora la sua funzione caratteristica $\mathcal{X}_Q$ è calcolabile e può essere utilizzata per definire quella semi-caratteristica di $P$.

$$
SC_P(x) = \begin{cases}
1  & \exists y . Q(x,y) \\
\uparrow &\text{altrimenti}
\end{cases} = \mathbb{1}\Bigg( \mu w. \overline{sg} \Big(\mathcal{X}_Q(x,w)  \Big)\Bigg)
$$

Se invece $P$ è semi-decidibile, allora esiste un programma $e$ che calcola la sua funzione semi-caratteristica. Inoltre, se $P$ è vero, esiste un numero di passi che fanno terminare il programma $e$:

$$
P(x) \equiv \mu y . H(e, x, y )
$$

Indicando con $Q(x,y)$ il predicato $H(e,x,y)$ si ottiene il predicato decidibile desiderato.


\subsection{Esercizio 2}

Sia $A \subseteq \mathbb{N}$ e sia $f : \mathbb{N} \rightarrow \mathbb{N}$ una funzione calcolabile. Dimostrare che se $A$ è ricorsivo allora $f^{-1}(A) = \{ x \in \mathbb{N} | f(x) \in A \}$ è RE. L'insieme $f^{-1}(A)$ è anche ricorsivo?

\subsubsection{Soluzione}

Sia $e$ un programma che calcola $f$. Questo programma può essere utilizzato per definire la funzione semi-caratteristica di $f^{-1}(A)$:

$$
SC_{f^{-1}(A)}(x) = \begin{cases}
1 & x \in f^{-1}(A) \\
\uparrow &\text{altrimenti}
\end{cases}  = \begin{cases}
1 & f(x) \in A \\
\uparrow &f(x) \notin A
\end{cases} = \mathbb{1}\Bigg( \mu w . \overline{sg}\bigg(S\Big(e,x,(w)_1, (w)_2 \Big) \wedge \mathcal{X}_A\Big((w)_1\Big) \bigg)\Bigg)
$$

Questa funzione è calcolabile perché definita utilizzando funzioni calcolabili, e quindi $f^{-1}(A)$ è RE.

Dal momento che non è garantito che $f$ è totale, l'insieme $f^{-1}(A)$ non è ricorsivo, perché può esistere un $x$ per il quale $f(x)$ non è definita e quindi non può essere determinato se $f(x)$ è in $A$ o meno.

\subsection{Esercizio 3}

$$
A = \{x : W_x \cap E_x \neq \emptyset \}
$$

\subsubsection{Soluzione}

L'insieme è saturo, perché contiene tutti i programmi le cui funzioni calcolate hanno $dom(f) \cap cod(f) \neq \emptyset$.

$$
\mathcal{A} = \{f \in \mathcal{C} : dom(f) \cap cod(f) \neq \emptyset \}
$$

Quindi per il teorema di Rice $A$ non è ricorsivo.

$A$ sembra essere ricorsivo, perché una volta trovato un punto in comune tra il dominio e il codominio, il test di appartenenza può terminare.

$$
SC_A(x) = \mathbb{1} \Bigg(\mu w . \overline{sg} \Big( S\big(x, (w)_1, (w)_1, (w)_2 \big) \Big) \Bigg)
$$

Essendo la funzione semi-caratteristica calcolabile, $A$ è RE e non ricorsivo per quanto detto prima. $\overline{A}$ deve quindi essere non RE.

\paragraph{Dimostrazione che $\overline{A}$ è non RE}

$\overline{\mathcal{A}}$ probabilmente è non RE, perché per determinare l'appartenenza è necessario valutare tutto il dominio e il codominio per determinare se l'intersezione è vuota.

La funzione $f(x) = 1$ è in $\overline{A}$ perché $dom(f) \cap cod(f) = \{1\}$ e quindi $f \notin \overline{\mathcal{A}}$.

La sua parte finita:

$$
\vartheta(x) = \begin{cases}
1 & x = 0 \\
\uparrow &\text{altrimenti}
\end{cases}
$$

è in $\overline{\mathcal{A}}$ perché $dom(\vartheta) \cap cod(\vartheta) = \emptyset$. Quindi per Rice-Shapiro $\overline{A}$ è non RE. 

\subsection{Esercizio 4}

$$
B = \{ x \in \mathbb{N} : \forall k \in \mathbb{N}. k+x \in W_x \}
$$

\subsubsection{Soluzione}

$B$ sembra essere non-RE perché per decidere l'appartenenza bisogna controllare infiniti valori.

Si va quindi di riduzione $\overline{K} \leq_m B$.

$$
g(x,y) = \begin{cases}
1 & \neg H(x,x,y) \\
\uparrow & H(x,x,y)
\end{cases} = \mathbb{1} \Big( \mu w. H(x,x,y) \Big)
$$

\`E calcolabile, quindi per SMN esiste $f$ che fa da funzione di riduzione:

\begin{itemize}
	\item $x \in \overline{K}$: $\phi_{f(x)}(y) = g(x,y) = 1 \forall y$. $\phi_{f(x)}(y)$ è quindi sempre definita ed in particolare $\forall k \in \mathbb{N}, f(x) + k \in W_{f(x)}$ e quindi $f(x) \in B$.
	\item $x \in K$: $\exists y_0 \in \mathbb{N} .\phi_{f(x)}(y) = g(x,y) = \uparrow \forall y > y_0$. Quindi $\exists k \in \mathbb{N} . f(x) + k \notin W_{f(x)}$ e quindi $f(x) \notin B$.
\end{itemize}

Si ha quindi che $B$ è non-RE.

$\overline{B} = \{ x : \exists k \in \mathbb{N} . x + k \notin W_x \}$ sembra essere non-RE perché per trovare un $k$ che soddisfa la condizione di appartenenza è necessario aspettare la terminazione del programma su un input in cui non termina.

$$
g(x,y) = \begin{cases}
\uparrow & x \in \overline{K} \\
1 & x \in K
\end{cases} = SC_K(x)
$$

Per SMN esiste $f$:

\begin{itemize}
	\item $x \in \overline{K}$: $\phi_{f(x)}(y) = g(x,y) = \uparrow \forall y$ e quindi $f(x) \in \overline{B}$.
	\item $x \in K$: $\phi_{f(x)}(y) = g(x,y) = 1 \forall y$, in particolare $W_{f(x)} = \mathbb{N}$ e quindi $f(x) \in B$.
\end{itemize}

$\overline{B}$ quindi è non-RE.

\subsection{Esercizio 5}

Dimostrare che l'insieme non è saturo:

$$
C = \{ x : \phi_x(x) = x^2 \}
$$

\subsubsection{Soluzione}

Il secondo teorema dice che data $f$ calcolabile e totale, esiste $e$ tale che $\phi_e = \phi_{f(e)}$.

$$
g(x,y) = \begin{cases}
y^2 &y = x \\
\uparrow &\text{altrimeniti}
\end{cases} = y^2 \cdot \mathbb{1}\Big( \mu w . |y - x|\Big)
$$ 

Per SMN esiste $f$. $\phi_{f(x)}(x) = g(x,x) = x^2 \rightarrow f(x) \in C$. 

Per il secondo teorema $\exists e . \phi_e(e) = \phi_{f(e)}(e)= g(e,e) = e^2 \rightarrow e \in C, W_e = \{e^2\}$. 

Sia $e' \in \mathbb{N} : \phi_{e'} = \phi_e, e' \neq e \rightarrow \phi_{e'}(e') = \uparrow \neq e'^2 \rightarrow e' \notin C$.

Ma dato che $\phi_e = \phi_{e'}$ si ha che l'insieme $C$ non è saturo.