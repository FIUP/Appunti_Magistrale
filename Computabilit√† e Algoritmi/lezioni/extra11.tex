\section{Appello 2016-01-25}

\subsection{Esercizio 1}

Dimostrare che $A \subseteq \mathbb{N}$ è ricorsivo se e solo se $A$ e $\overline{A}$ sono RE.

\subsubsection{Soluzione}

Se $A$ è ricorsivo posso definire 

$$
SC_A(x) = \mathbb{1}\Big( \mu w . \overline{sg}\big(\mathcal{X}_A(x)\big)\Big)
$$

che risulta calcolabile perché definita utilizzando funzioni calcolabili e quindi $A$ è anche RE. Per ottenere $SC_{\overline{A}}$ basta togliere il segno negato, oppure utilizzare $\mathcal{X}_{\overline{A}}$.

Se invece sono entrambi RE, posso definire 

$$
\mathcal{X}_A(x) = \bigg( \mu w . S\Big(e_1, x, \big(w\big)_1, \big(w\big)_2\Big) \vee S\Big(e_2, x, \big(w\big)_1,\big(w\big)_2\Big) \bigg)_1
$$

dove $e_1$ e $e_2$ sono i programmi che calcolano le due funzioni semi-caratteristiche (quella di $\overline{A}$ deve essere composta con il segno negato).

\subsection{Esercizio 2}

Definire una funzione $f : \mathbb{N} \rightarrow \mathbb{N}$ totale e non calcolabile tale che $f(x) = x$ per infiniti argomenti $x \in \mathbb{N}$ oppure dimostrare che questa funzione non esiste.

\subsubsection{Soluzione}

La funzione deve essere definita su tutto $\mathbb{N}$ e $f(x) = id(x) = x$.

Dato che le due funzioni sono uguali e sono definite sullo stesso dominio, un programma che calcola $id$ calcola anche $f$ ed essendo $id$ calcolabile, esistono infiniti programmi che sono in grado di calcolarla e che quindi riescono a calcolare anche $f$.

Tuttavia, la funzione può essere non calcolabile:

$$
f(x) = x \cdot \mathbb{1}\Big(\mathcal{X}_K(x)\Big)
$$

Così definita, è totale, uguale all'identità e non calcolabile perché $\mathcal{X}_K$ è non calcolabile.

\subsection{Esercizio 3}

$$
A = \{ x | \phi_x \text{ strettamente crescente}  \}
$$

\subsubsection{Soluzione}

$A$ probabilmente è non-RE perché per valutare l'appartenenza è necessario controllare infiniti punti del dominio.

$A$ è saturo:

$$
\mathcal{A} = \{f | \forall x,y \in dom(f) : x < y \rightarrow f(x) < f(y) \}
$$

Una qualsiasi funzione gradino:

$$
f(x) = \begin{cases}
0 & x \leq x_0 \\
1 & x > 0
\end{cases}
$$

non appartiene a $\mathcal{A}$, ma la parte finita 

$$
\vartheta(x) = \begin{cases}
0 & x= x_0 \\
1 & x = x_0+1 \\
\uparrow &\text{altriementi}
\end{cases}
$$

appartiene ad $\mathcal{A}$ e quindi per Rice-Shapiro, $A$ è non-RE.

Per quanto riguarda $\overline{A}$, per decidere l'appartenenza basta trovare una coppia di punti sui quali la funzione è non crescente, quindi potrebbe essere RE.

$$
SC_{\overline{A}}(x) =\mathbb{1} \Bigg( \mu w . \overline{sg}\bigg(S\Big(x, (w)_1, (w)_2, (w)_3 \Big) \wedge S\Big(x, (w)_4, (w)_5, (w)_6 \Big) \wedge (w)_1 < (w)_4 \wedge (w)_2 \geq (w)_5\bigg) \Bigg)
$$

Essendo la funzione semi-caratteristica calcolabile, $\overline{A}$ è non-RE.

\subsection{Esercizio 4}

$$
B = \{ x : x > 0 \wedge x/2 \notin E_x \}
$$

\subsubsection{Soluzione}

$B$ probabilmente è non-RE, perché dovrei provare il programma $x$ su infiniti input.

$$\overline{K} \leq_m B$$

$$
g(x,y) = \begin{cases}
\uparrow & x \notin K \\
y/2 & x \in K
\end{cases} = y/2 \cdot SC_K(x)
$$

Per SMN esiste la funzione di riduzione $f$:

\begin{itemize}
	\item $x \notin K$: $\phi_{f(x)}(y) = g(x,y) = \uparrow \forall y \:\Rightarrow E_{f(x)} = \emptyset \Rightarrow f(x) \in B$.
	\item $x \in K$: $\phi_{f(x)}(y) = g(x,y) = y/2 \forall y \: \Rightarrow \phi_{f(x)}(f(x)) = f(x)/2 \Rightarrow f(x)/2 \in E_{f(x)}  \Rightarrow f(x) \notin B $
\end{itemize}

E quindi $B$ è non RE.

$\overline{B}$ invece sembra essere ricorsivo, perché basta eseguire in parallelo il programma su tutti i possibili input fino a che non viene trovato in output il valore $x/2$.

$$
SC_{\overline{B}}(x) =\mathbb{1} \Bigg( \mu w. \overline{sg} \bigg(S(x,(w)_1, x/2, (w)_2 ) \bigg) \Bigg)
$$

$\overline{B}$ è RE e non ricorsivo.

\subsection{Esercizio 5}

$$\exists x . \phi_x(y) = x+y$$


\subsubsection{Soluzione}

Definisco $g(x,y) = x+y$.

Per SMN:

$$
\phi_{f(x)}(y) = g(x,y) = x+y
$$

Per il secondo teorema di ricorsione

$$
\exists x . \phi_x(y) = \phi_{f(x)}(y) = g(x,y) = x +y
$$