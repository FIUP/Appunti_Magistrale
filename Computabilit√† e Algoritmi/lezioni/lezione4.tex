% !TEX encoding = UTF-8
% !TEX TS-program = pdflatex
% !TEX root = computabilità e algoritmi.tex
% !TEX spellcheck = it-IT

\section{Algoritmo}\label{algoritmo}

\textbf{Algoritmo}: descrizione di una sequenza di passi elementari che
permettono di raggiungere un certo obiettivo, come la trasformazione di
certi dati di input in dati di output. Il vincolo dei passi elementari
si traduce nell'utilizzo di operazioni semplici, direttamente eseguibili
da un calcolatore. Nel caso l'algoritmo sia deterministico, questo può
essere visto come una funzione che mappa un certo input in un
determinato output, in questo caso si dice che la funzione \textbf{è
calcolata} dall'algoritmo.

\textbf{Funzione calcolabile}: una funzione è calcolabile in modo
effettivo se \textbf{esiste} un algoritmo in grado di calcolarla. A noi
interessa sapere che \textbf{esiste} un algoritmo che calcola quella
determinata funzione, ma non sempre ci interessa conoscere l'algoritmo.

Alcune funzioni calcolabili sono:

$$ MCD(x,y) $$

$$
P(n) = \begin{cases}
1, &\text{se \textit{n} è primo} \\
0, &\text{altrimenti}
\end{cases}
$$

$$ p(n) = n-\text{esimo numero primo}$$

$$ f(n) = n-\text{esima cifra di }\pi$$

Una funzione non calcolabile è data da

$$
g(n) = \begin{cases}
1, &\text{se $\pi$ contiene esattamente \textit{n} cifre ``5'' consecutive} \\
0, &\text{altrimenti}
\end{cases}
$$

Non si sa se la funzione \emph{g(n)} è calcolabile, perché si riesce a
trovare un algoritmo che funziona solamente nel caso in cui la sequenza
è presente.

$$
h(n) = \begin{cases}
1, &\text{se $\pi$ contiene almeno \textit{n} cifre ``5'' consecutive} \\
0, &\text{altrimenti}
\end{cases}
$$

Stranamente \emph{h(n)} è calcolabile, perché si può considerare

$$
k = sup( \{n | \pi \text{contiene \textit{n} cifre ``5'' consecutive}\} )
$$

\emph{k} può essere un numero qualsiasi o infinito. In ogni caso, la
funzione \emph{h} può essere definita come


$$
h(n) = \begin{cases}
1, &\text{se }  n \leq k\\
0, &\text{altrimenti}
\end{cases}
$$

Da notare che l'algoritmo così definito non si preoccupa di quanto sia
difficile trovare \emph{k}, ma solo di trovare una soluzione per il
problema.
 Questo deriva dall'\textbf{esiste} precedentemente riportato in
grassetto.
Noi ci accontentiamo del fatto che qualsiasi valore può
assumere \emph{k} non conosciamo un algoritmo, l'algoritmo per trovare
\emph{k} può esistere o meno, ma questo non ci interessa perché vogliamo
che la teoria della calcolabilità non venga influenzata dalla conoscenza
sul dominio ma che sia una caratteristica della funzione. 
Ad esempio, in questo caso non ci interessa sapere se il $\pi$ è un numero normale o
meno e non vogliamo che la definizione di calcolabilità sia influenzata
da ciò.

Con \emph{g(n)} la stessa funzione può essere implementata con
l'algoritmo


$$
g'(n) = \begin{cases}
1, &\text{se } n \in X \\
0, &\text{altrimenti}
\end{cases}
$$

ma l'insieme \emph{X} può essere un insieme infinito e per valutare
l'appartenenza è necessario un programma infinito.

\section{Le funzioni non calcolabili}\label{le-funzioni-non-calcolabili}

\textbf{Modello di calcolo effettivo}: ovvero le caratteristiche che
deve avere un algoritmo per poter essere eseguibile in modo effettivo:

\begin{itemize}
\item
  \textbf{Lunghezza sintattica finita}: deve essere una sequenza finita
  di istruzioni.
\item
  \textbf{Modello realistico}: esegue su un agente di calcolo che può
  essere realizzato, dotato di una memoria e che lavora a passi discreti
  (\emph{macchina digitale}) e deterministici.
\item
  \textbf{Memoria e input illimitati}: questo perché non si vuole
  limitare la teoria della computabilità allo stato tecnologico attuale,
  idealmente si può sempre aggiungere un nuovo banco di RAM.
\item
  \textbf{Set di istruzioni finito}: le istruzioni che la macchina
  riesce ad eseguire sono limitate ed hanno anche una complessità
  limitata.
\item
  \textbf{Computazione}: può terminare dopo un numero finito di passi
  producendo un output, oppure può non terminare senza produrre output.
\end{itemize}

Questo modello equivale ad una macchina di Turing.

\subsection{Notazione utilizzata}\label{notazione-utilizzata}

Notazioni:

\begin{itemize}
\item
  $ \mathbb{N}= \{0,1,2,3,\ldots{}\}$
\item
  $A \times B = \{(a,b) | a \in A, b \in B\}$ (prodotto cartesiano)
\item
  $A \times A \times \ldots \times A = A^n$
\item
  Relazione $r \subseteq A \times B$, ovvero un sotto insieme del prodotto
  cartesiano di uno o più insiemi.
\item
  Una funzione è una particolare relazione: $f \subseteq A \times B$ e vale anche
  $\forall(a,b), (a,b') \in f \therefore b = b' \text{ e } f(a) = b$
\item
  Dominio di una funzione: $dom(f) = \{a | \exists (a,b) \in f\}$ e
  per indicare che \textit{f} è definita su \textit{a} si utilizza $f(a)$\emph{freccia verso
  basso}.
\item
  Cardinalità $|A|$, numero di elementi presenti
  nell'insieme, nel caso di elementi finiti, mentre nel caso di insiemi
  infiniti si ha che:

  \begin{itemize}
  \item $|A| = |B|$ se esiste $f: A \rightarrow B$ e biunivoca
  \item $|A| \leq |B|$ se esiste $f: A \rightarrow B$ e iniettiva
  \end{itemize}
\item
  Un insieme è \textbf{numerabile} quando $|A| \leq |\mathbb{N}|$, ovvero esiste una funzione suriettiva $f: |\mathbb{N}| \rightarrow A$. 
  Se $ A $ e $ B $ sono numerabili, anche il
  loro prodotto è numerabile. Allo stesso modo se una sequenza di
  insiemi è numerabile, anche la loro unione è enumerabile.
\end{itemize}

\subsection{Esistenza delle funzioni non calcolabili}\label{esistenza-delle-funzioni-non-calcolabili}

Con il modello di calcolo precedentemente descritto ci sono delle
funzioni che non possono essere calcolate.

Fissando un insieme di funzioni unarie e parziali

$ F = \{f | f: \mathbb{N} \rightarrow \mathbb{N}\}$

e un modello di calcolo con i relativi algoritmi $\mathcal{A}$.

Le funzioni calcolabili nel modello di calcolo sono date da:

$$
F_\mathcal{A} = {f | \exists A \in \mathcal{A} \text{ che calcola } f}
$$

dato un algoritmo $ A \in \mathcal{A}$ riusciamo a trovare una funzione $f_A$ che
viene calcolata.

La domanda risulta quindi essere ``$F_\mathcal{A} \subseteq F$ o $ F_\mathcal{A} \subsetneq F $?'' e la risposta è che l'inclusione è stretta.

\paragraph{Dimostrazione}\label{dimostrazione}

Sia \emph{I} il set finito di istruzioni della macchina di calcolo.

L'insieme degli algoritmi calcolabili risulta quindi essere:

$$
\mathcal{A} \subseteq I \cup I \times I \cup I \times I \times I \cup I \times I \times ... = \bigcup_n I^n
$$

pertanto

$$
|\mathcal{A}| \leq |\bigcup_n I^n| \leq |\mathbb{N}|
$$

perché la funzione $ f: \mathcal{A} \rightarrow F_\mathcal{A}$  è
ovviamente suriettiva, pertanto si ha:

$$
|F_\mathcal{A}| \leq |\mathcal{A}| \leq |\mathbb{N}|
$$

L'insieme delle funzioni è certamente infinito e certamente esiste un
sottoinsieme \emph{T} delle funzioni totali che non è enumerabile.

$$
T = \{ f | f : \mathbb{N} \rightarrow \mathbb{N} \text{e totatli}\}
$$

La non enumerabilità si dimostra per assurdo, perché se
$f_0,\ f_1, \ldots$ è un'enumerazione di \emph{T}:

\begin{verbatim}
    f0      f1      f2      ...
0   f0(0)   f1(0)   f2(0)   ...
1   f0(1)   f1(1)   f2(1)   ...
2   ...     ...     ...
\end{verbatim}

È quindi possibile definire  $d(n) = f_n(n) è 1$ che non è
presente nell'enumerazione sopra riportata, perché $ d(n) \neq f(n) \forall n$.

Pertanto 

$$ |\mathbb{N}| \leq |T| \leq |F|$$

e
$$
\begin{rcases}
F_\mathcal{A} \subseteq F \\
|F_\mathcal{A} < |F||
\end{rcases}
 \Rightarrow \: F_\mathcal{A} \subsetneq F
$$

Ovvero, esistono delle funzioni che non sono calcolabili. Il numero di
queste funzioni è dato da $ | F \ F_\mathcal{A}| $ che è
infinita, questo perché se fosse finita, \textit{F} sarebbe enumerabili.
