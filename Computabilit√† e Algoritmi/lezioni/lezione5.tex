% !TEX encoding = UTF-8
% !TEX TS-program = pdflatex
% !TEX root = computabilità e algoritmi.tex
% !TEX spellcheck = it-IT

\section{Modelli di calcolo}

Nel tempo sono stati proposti diversi modelli di calcolo:

\begin{itemize}
\item
  Macchina di Turing, 1936
\item
  Lambda calcolo, Church 1936
\item
  Funzioni parziali ricorsive, G\"{o}edel-Kleene
\item
  Sistemi di Post, grammatiche libere da contesto
\item
  Markov System, 1951
\item
  Macchina a registri, 1963
\end{itemize}

La lista dei modelli è lunga, pertanto si può pensare che sia necessaria una teoria della computabilità per ogni modello. 
Tuttavia si è notato che le classi delle funzioni calcolabili dai vari modelli erano sempre quelle. 
Queste congetture sono state poi formalizzate dalla \textbf{tesi di Church-Turing}: ogni funzione è calcolabile con un procedimento effettivo se e solo se è calcolabile da una macchina di Turing.

\subsection{URM - Ulimited Register Machine}\label{urm---ulimited-register-machine}

Il modello di calcolo che utilizzeremo è quello della macchina URM, la quale è dota di una serie di registri, ognuno dei quali contiene dei numeri naturali.

\begin{verbatim}
 R1 R2 R3
|r1|r2|r3|..|..|..
\end{verbatim}

$$r_i \in \mathbb{N}$$

C'è inoltre un agente in grado di eseguire un programma, ovvero una sequenza di azioni $ I_1, I_2, \ldots, I_n$.
Queste istruzioni possono essere:

\begin{itemize}
\item
  Aritmetiche:
  \begin{itemize}
  \item \texttt{Z(n)}: sposta 0 in $r_n, r_n \gets 0 $
  \item \texttt{S(n)}: incrementa di 1 il valore del registro $ r_n, r_n \gets r_n +1 $
  \item \texttt{T(m,n)}: copia il contenuto del registro $r_m$ in $r_n$, $ r_n \gets r_m$.
  \end{itemize}
\item Salto condizionato:
  \begin{itemize}
  \item \texttt{J(m,n,t)}: se $ r_m $ è uguale a $ r_n $, salta all'istruzione $ I_t  $, altrimenti procede con l'istruzione successiva. 
	  Se viene fatto un salto ad un'istruzione che si trova fuori dal programma, l'esecuzione termina.
  \end{itemize}
\end{itemize}

Un programma d'esempio è dato da:

\begin{lstlisting}[language=URM]
I1: J(2,3,5)
I2: S(1)
I3: S(3)
I4: J(1,1,1)
\end{lstlisting}

l'esecuzione del programma risulta quindi essere:

\begin{verbatim}
     R1 R2 R3 R4
    |1 |2 |0 |0 |

I1: false, non salta

I2:  R1 R2 R3 R4
    |2 |2 |0 |0 |
    
I3:  R1 R2 R3 R4
    |2 |2 |1 |0 |

I4: true, esegue I1:

I1: false, non salta

I2:  R1 R2 R3 R4
    |3 |2 |1 |0 |
    
I3:  R1 R2 R3 R4
    |3 |2 |2 |0 |

I4: true, esegue I1:

I1: true, salta alla terminazione
\end{verbatim}

La macchina ha memoria infinita e lo stato iniziale della computazione
del programma \emph{P} viene indicato con:

$$
P(a_1, a_2, \ldots)\downarrow
$$

La notazione

$$
P(a_1, a2, \ldots) \downarrow a
$$

indica che il programma \emph{P} eseguito sulla configurazione $a_1, a_2, \ldots, a_n, \ldots, 0$ termina producendo in $r_1$ il valore \emph{a}.

Per indicare che il programma non termina viene utilizzato:

$$
P(a_1, a_2, \ldots)\uparrow
$$

\subsection{Funzioni URM-calcolabili}\label{funzioni-urm-calcolabili}

Una funzione parziale $f: \mathbb{N}^k \rightarrow \mathbb{N}$ si dice \textbf{URM-calcolabile} se esiste \emph{P} programma URM tale che per ogni input $a_1, \ldots, a_k$ in $\mathbb{N}$, il programma \emph{P} termina producendo $a \in \mathbb{N}$ se e solo se la funzione è definita sulla tupla e il risultato della funzione coincide con \emph{a}.

Più formalmente:

$$
P(a_1, a_2, \ldots, a_k)\downarrow \: \text{calcola $ f $ se}
\begin{cases}
&(a_1, a_2, \ldots, a_k) \in dom(f) \\
&f(a_1, a_2, \ldots, a_k) = a
\end{cases}
$$


Con la lettera $\mathcal{C}$ indichiamo la classe delle funzioni URM-calcolabili e con $\mathcal{C}^k$ ci si restringe alle funzioni \emph{k}-arie.
Un esempio di funzione calcolabile è dato da \emph{f(x,y) = x+y}, laquale può essere calcolata dal programma:

\begin{lstlisting}[language=URM]
	/*
		 R1 R2 R3 ...
		| x| y| 0|...
		uso r3 come k, e cerco di fare in modo che r1 sia uguale a x+k
	*/
	I1: J(2,3,END) #LOOP
	I2: S(1) // x = x + 1
	I3: S(3) // k = k + 1 
	I4: J(1,1,LOOP)
	#END
\end{lstlisting}

Un altro esempio è dato da $ f(x) = \begin{cases}
\frac{x}{2} &\text{ se \textit{x} è pari} \\
\uparrow &\text{ altrimenti}
\end{cases}$

\begin{lstlisting}[language=URM]
 /*
	 R1 R2 R3 ...
	| x| 0| 0|...
	
	uso r2 come k e r3 come 2k.
	Incremento 2k di 2 e k di 1, quando 2k = x, allora k è x/2
*/
I1: J(1,3,END) #LOOP
I2: S(2) // k = k + 1
I3: S(3) // 2k = 2k + 1 
I3: S(3) // 2k = 2k + 1 
I4: J(1,1,LOOP)
I5: T(2,1) #END
\end{lstlisting}

Come esercizio, provare ad implementare \emph{f(x) = x-1, f(0) = 0}.

Da notare che per individuare la funzione calcolata da un programma URM è necessario specificare sia il programma che il numero di argomenti, altrimenti non è possibile definire l'arietà della funzione.

$$
f_{p}^{(k)} : \mathbb{N}^k \rightarrow \mathbb{N}
$$

$$
f_p(\vec{x}) = 
\begin{cases}
y &\text{ se} \: P(\vec{x})\downarrow y \\
\uparrow &\text{ se il programma non termina}
\end{cases}
$$

Quando una funzione è calcolabile, esistono infiniti programmi in grado di calcolarla, perché risulta semplice andare ad aggiungere delle istruzioni ``inutili'' per generare un nuovo programma che calcola la funzione, mentre se non è calcolabile, non esiste nessun programma in grado di calcolarla.

\subsubsection{Esercizio teorico - URM e URM'}\label{esercizio-teorico---urm-e-urm}

Sia URM' la macchina senza \texttt{T(m,n)} e sia $\mathcal{C}'$ la classi di funzioni URM'-calcolabili, questa classe come si relaziona con $\mathcal{C}$?
Come prima cosa ci si può chiedere se

$$ \mathcal{C}' \subseteq \mathcal{C} $$

Sia $ f: \mathbb{N}^k \rightarrow \mathbb{N}$ URM' calcolabile, ovvero $ f \in \mathcal{C}' $, allora esiste \emph{P'} programma in URM' tale che $f = f_p$.
Per come è definito il set di istruzione URM', \emph{P'} è anche un programma in URM, quindi $f \in \mathcal{C}$

Resta ora da capire se l'inclusione è stretta o meno. L'istruzione \texttt{T(m,n)} può essere sostituita dalle istruzioni

\begin{lstlisting}[language=URM]
Z(n)
J(n,m,END) #LOOP
S(n)
J(1,1,LOOP)
#END
\end{lstlisting}

quindi, essendoci un modo per simulare l'istruzione del trasferimento si ottiene che

$$
\mathcal{C} \subseteq \mathcal{C}'
$$

e quindi le due classi \textbf{sono equivalenti}. 
Tuttavia, questa è una congettura e non una dimostrazione formale.

\paragraph{Dimostrazione formale} Sia $f \in \mathcal{C}$, ovvero esiste un \emph{P} in URM e $k \in \mathbb{N}$ tale che $f = f_{p}^{(k)}$, si vuole dimostrare che esiste un programma \emph{P'} in URM', tale che $f_{p'}^{(k)} = f_{p}^{(k)}$.
Questo può essere dimostrato per induzione su \emph{h = \#numero di
istruzioni \texttt{T(m,n)} in P}.
Banalmente se \emph{h=0} si ha che \emph{P} è già un programma URM', mentre se $h > 0$, il programma \emph{P} avrà un'istruzione $I_s$ che sarà un trasferimento \texttt{T(m,n)}. 
Tale programma programma può essere riscritto come \emph{P''}, con $I_s$ uguale a \texttt{J(1,1,l+2)}, dove all'istruzione $I_{l+2}$ iniziano le istruzioni alternative per fare la copia del valore, come precedentemente riportato, e al termine delle quali c'è un salto all'istruzione $I_{s+1}$

Resta un problema riguardo ad i salti oltre la fine, è quindi necessario normalizzare i salti in modo che tutti quelli che puntano al termine del programma finiscano all'istruzione $I_{l+1}$, dove $I_l$ è l'ultima istruzione del programma \emph{P}. L'istruzione $I_{l+1}$ di \emph{P''} sarà quindi un salto all'effettiva fine del programma.

Applicando questo procedimento si ottiene un programma \emph{P''} che avrà un'istruzione \texttt{T(m,n)} in meno e che calcola la stessa funzione calcolata da \emph{P}.
Ripetendo questo processo si ottiene induttivamente il programma \emph{P'} che non ha istruzioni del tipo \texttt{T(m,n)}.

\subsubsection{Esercizio teorico - URM e URM-S}\label{esercizio-teorico---urm-e-urm-s}

Come si relazione la classe $\mathbb{C}^S$ con $\mathbb{C}$? Dove $\mathbb{C}^S$ è la classe delle funzioni calcolabili da URM-S ovvero dalla macchina URM che non ha l'istruzione \texttt{T(m,n)} ma ha l'istruzione \texttt{TS(m,n)} che fa lo swap del contenuto di dei due registri.

Esercizio lasciato per casa.

\todo[inline]{TODO}

\subsubsection{Esercizio teorico - URM e URM-SL}\label{esercizione-teorico---urm-e-urm-sl}

Sempre quella domanda, con la differenza che viene tolta l'istruzione di salto \texttt{J(m,n,l)}.

Ovviamente $\mathbb{C}^{SL}$ è contenuta in $\mathbb{C}$, ma non vale la relazione $\mathbb{C} \subseteq \mathbb{C}^{SL}$ perché $\mathbb{C}^{SL}$ contiene solamente funzioni totali, ovvero che terminano sempre, mentre $\mathbb{C}$ contiene anche delle funzioni parziali.

Per le istruzioni disponibili in URM-SL si riesce a calcolare \emph{f(x) = k} oppure \emph{f(x) = x+k} per $k \in \mathbb{N}$. 
Ovvero $f \in \mathbb{C}^{SL}$ se e solo se \emph{f(x) = k oppure x+k}.

Questo vuol dire che per ogni programma $P_{sl}$ la funzione che viene calcolata $f_{p_{SL}}$ è del tipo di \emph{f}.

La dimostrazione viene fatta per induzione sulla struttura di un programma URM-SL, ragionando sul cosa succede se si aggiunge un'ulteriore istruzione al programma già esistente e che calcola un valore costante.
