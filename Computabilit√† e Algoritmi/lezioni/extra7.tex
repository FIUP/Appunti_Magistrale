\section{Appello 2014-07-02}

\subsection{Esercizio 1}

Dimostrare che $A \subseteq \mathbb{N}$ è ricorsivo solo se $A, \overline{A}$ sono RE.

\subsubsection{Soluzione}

$A$ ricorsivo $\Rightarrow$ $A, \overline{A}$ sono RE.

Essendo $A$ ricorsivo, la sua funzione caratteristica $\mathcal{X}_A$ è calcolabile e può essere utilizzata per definire la funzione semi-caratteristica di $A$:

$$
SC_A(x) = \mathbb{1}\Big( \mu w . \overline{sg}\big(\mathcal{X}_A(x)\big) \Big)
$$

La funzione semi-caratteristica di $\overline{A}$ può essere definita in modo analogo

$$
SC_{\overline{A}}(x) = \mathbb{1}\Big( \mu w . \mathcal{X}_A(x) \Big)
$$

$A, \overline{A}$ sono RE $\Rightarrow$ $A$ ricorsivo 

Entrambe le funzioni semi-caratteristiche sono calcolabili e possono essere combinate per definire la funzione caratteristica.

Sia $e_1$ un programma tale che $\phi_{e_1} = SC_A$ e $e_2$ tale che $\phi_{e_2} = \overline{sg}(SC_{\overline{A}})$, entrambi esistono perché le loro funzioni sono calcolabili.

\begin{align*}
\mathcal{X}_A(x) &= \begin{cases}
1 & x \in A \\
0 & x \notin A
\end{cases} = \begin{cases}
SC_A(x) & x \in A \\
\overline{sg}(SC_{\overline{A}}(x)) & x \notin A
\end{cases}   \\
&= \Bigg( \mu w . \bigg(  S\Big(e_1, x,  (w)_1, (w)_2\Big) \wedge S\Big(e_2, x,  (w)_1, (w)_2\Big) \bigg) \Bigg)_1
\end{align*}

\subsection{Esercizio 2}

Definire una funzione $f : \mathbb{N} \rightarrow \mathbb{N}$ totale, non calcolabile tale che $f(x) = x/2$ per ogni $x \in \mathbb{N}$ e pari, oppure dimostrare che tale funzione non esiste.

\subsubsection{Soluzione}

La funzione deve essere totale e non viene specificato nulla per quanto riguarda il valore della funzione sui numeri dispari.

Sia $\{ \phi_n \}$ una qualsiasi enumerazione delle funzioni calcolabili.

La funzione $f$ può essere definita come 

$$
f(x) = \begin{cases}
x/2 &x \text{ è pari} \\
\phi_x(x)+1 &\text{altrimenti}
\end{cases}
$$

Si ha quindi che la funzione $f$:
\begin{itemize}
	\item è totale, perché definita su tutto $\mathbb{N}$
	\item vale $x/$ se $x$ è pari
	\item è diversa da tutte le funzioni calcolabili se $x$ è dispari e quindi negli infiniti punti dispari non è calcolabile
\end{itemize}

\subsection{Esercizio 3}

$$
A = \{ x | \forall k \in \mathbb{N} . x +k \in W_x \}
$$

\subsubsection{Soluzione}

$A$ sembra non essere RE, perché per provare l'appartenenza è necessario andare a provare infiniti valori di $k$.

Si può quindi provare la riduzione $\overline{K} \leq_m A$.

Serve quindi una funzione tale che se $x \in \overline{K}$, $f(x)$ sia definita $\forall x' \geq x$ ed una funzione definita su tutto $\mathbb{N}$ soddisfa questa condizione, mentre se $x \in K$, $f(x)$ non deve essere definita in almeno un punto $x' \geq x$.

$$
g(x,y)= \begin{cases}
1 & \neg H(x,x,y) \\
\uparrow & H(x,x,y) 
\end{cases}
$$

Non posso usare $x \in \overline{K}$ perché la funzione semi-caratteristica non è calcolabile, devo ragionare sul numero di passi $y$ impiegati dal programma per terminare.

Essendo calcolabile, per SMN esiste $f$ calcolabile e totale, che funziona da funzione di riduzione, perché:

\begin{itemize}
	\item $x \in \overline{K}$: $\forall y \: \neg H(x,x,y)$ è vero e quindi $\forall y \: \phi_{f(x)}(y) = 1$ ed in particolare $\phi_{f(x)}(f(x)+k) = 1 \: \forall k$, ovvero $f(x) \in A$.
	\item $x \in K$: $\exists y_0 $ tale che $\forall y \: > y_0$, $\phi_{f(x)}(y) = g(x,y) = \uparrow$, è quindi possibile trovare almeno un valore $y' > y_0$ tale che $y' > x + 0$, pertanto $f(x) \notin A$.
\end{itemize}

Si ha quindi che $A$ non è RE.

$\overline{A} = \{ x | \exists k \in \mathbb{N} . x +k \notin W_x \}$, anche in questo caso sembra non essere RE, perché per verificare l'appartenenza è necessario che il programma $x$ non termini quando riceve in input $x+k$.

Si può quindi provare la riduzione $\overline{K} \leq_m \overline{A} $.

Serve quindi una funzione tale che se $x \in \overline{K}$, $f(x)$ non termini quanto riceve in input $x+k$ e si può osservare che i programmi che calcolano $\emptyset$ sono in $\overline{A}$. Se invece $x \in K$, $f(x)$ deve essere definita su tutti gli input $> x$.

$$
g(x,y) = \begin{cases}
\uparrow & x \in \overline{K}  \\
1 & x \in K
\end{cases} = SC_K(x)
$$

Essendo calcolabile, per SMN esiste $f$ calcolabile e totale, che funziona da funzione di riduzione, perché:

\begin{itemize}
	\item $x \in \overline{K}$: il predicato $\neg H(x,x,y)$ vale $\forall y$ e quindi $\phi_{f(x)}(y) = \uparrow \forall y$, pertanto $f(x) \in \overline{A}$.
	\item $x \in K$: $\phi_{f(x)}(y) = g(x,y) = 1 \forall y$ ed in particolare $\phi_{f(x)}(y)\downarrow \forall y > x$ e quindi $f(x) \in A$
\end{itemize}

Segue quindi che anche $\overline{A}$ non è RE.

\subsection{Esercizio 4}

$$V = \{x | E_x \text{ infinito}\}$$

\subsubsection{Soluzione}

$V$ è saturo, perché $\mathcal{V} = \{f | \:  |cod(f)| = \infty \}$.

Banalmente la funzione $id \in \mathcal{V}$ e tutte le sue parti finite non appartengono per definizione a $\mathcal{V}$, quindi per Rice Shapiro $V$ è non RE.

Analogamente $id \notin \overline{\mathcal{V}}$ e la funzione $\emptyset$ appartiene a $ \overline{\mathcal{V}}$ ed è parte finita di $id$, quindi per Rice Shapiro, $ \overline{V}$ è non RE.

\todo{Sembra troppo facile, forse c'è un trabocchetto}

\subsection{Esercizio 5}

Enunciare il secondo teorema di ricorsione e dimostrare che $\exists k | W_k = \{k * i | i \in \mathbb{N}\}$.

\subsubsection{Soluzione}

Il teorema dice che, data una funzione $f : \mathbb{N} \rightarrow \mathbb{N}$ calcolabile e totale, $\exists e . \phi_e = \phi_{f(e)}$

$$
g(x, y) = \begin{cases}
1 &\text{\textit{x} è multiplo di \textit{y}} \\
\uparrow &\text{altrimenti}
\end{cases} = \mathbb{1}\Big( \mu z . |xz - y| \Big)
$$

Essendo $g$ calcolabile, per SMN esiste $f : \mathbb{N} \rightarrow \mathbb{N}$ calcolabile e totale tale che $g(x,y) = \phi_{f(x)}(y) \forall y$ e quindi si ha che

$$
W_{f(x)} = \{x * i | i \in \mathbb{N}\}
$$

e per il secondo teorema di ricorsione, esiste $x$ tale che $\phi_x = \phi_{f(x)}$, ovvero $\exists x$ tale che:

$$
W_x = W_{f(x)} = \{x * i | i \in \mathbb{N}\}
$$
