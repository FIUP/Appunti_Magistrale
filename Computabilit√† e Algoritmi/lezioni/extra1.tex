% !TEX encoding = UTF-8
% !TEX TS-program = pdflatex
% !TEX root = computabilità e algoritmi.tex
% !TEX spellcheck = it-IT

\chapter{Riassunto in preparazione all'esame}



\section{Definizioni utili / conoscenze base}

\begin{itemize}
	\item \textbf{Teorema SMN}: Dato $m,n \geq 1$, esiste $S : \mathbb{N}^{m+1} \rightarrow \mathbb{N}$ calcolabile e totale, tale che $\phi_{e}^{(m+n)}(\vec{x}, \vec{y}) =\phi_{S(e,\vec{x})}^{(n)}(\vec{y}) $ per qualche $\vec{x} \in \mathbb{N}^m$ e $\vec{y} \in \mathbb{N}^n$.
	
	\item \textbf{Insiemi ricorsivi e RE}: dato un sottoinsieme $X \subseteq \mathbb{N}$, se la funzione caratteristica dell'insieme è calcolabile si dice che l'insieme è \textbf{ricorsivo}, mentre se è calcolabile solo la funzione semi-caratteristica si dice che è \textbf{Ricorsivamente Enumerabile}. Se un insieme è ricorsivo, il predicato $x \in A$ è decidibile, mentre se è RE, il predicato è semi-decidibile.
	
	\item \textbf{Insieme saturo}: un insieme $A \subseteq \mathbb{N}$ si dice \textbf{saturo} se, dati $x \in A$ e $y \in \mathbb{N}$, tali che $\phi_x = \phi_y$, allora anche $y \in A$. Ovvero un insieme è saturo quando contiene tutti gli indici che calcolano una determinata funzione. Alternativamente $A$ è saturo se e solo se esiste $\mathcal{A} \subseteq \mathcal{C}$ tale che $A = \{ x \: | \: \phi_x \in \mathcal{A} \}$, cioè l'appartenenza all'insieme dipende da delle caratteristiche della funzione calcolata dal programma e non da come questo è definito.
	
	\item $K = \{ x \: | \: \phi_x(x) \downarrow \} = \{ x \: | \: x \in W_x \}$, ovvero l'insieme dei programmi che terminano quando ricevono se stessi in input, \textbf{non} è ricorsivo, ma è RE. L'insieme \textbf{non è saturo} (si dimostra cercando un $e'$ che calcola la stessa funzione di $e$ ma che non termina su se stesso).
	
	\item \textbf{Teorema di Rice}: Il teorema di Rice asserisce che qualsiasi proprietà non banale delle funzioni calcolate dai programmi  non è decidibile.
	In altre parole, sia $A \subseteq \mathbb{N}$ l'insieme che contiene tutti i programmi la cui funzione calcolata gode di una determinata proprietà (ovvero $A$ è un insieme saturo), se la proprietà non è banale ($A \neq \emptyset, \neq \mathbb{N}$) allora $A$ non è ricorsivo.
	
	\item $T = \{ x | \phi_x \text{ è totale} \}$, è saturo, non è ricorsivo, è RE.
	
	\item \textbf{Funzione finita}: una funzione si dice finita se è definita solamente in un numero finito di elementi.
	
	\item \textbf{Pezzo di una funzione}: date due funzioni $f,g : \mathbb{N} \rightarrow \mathbb{N}$, $f$ è un \textbf{pezzo} di $g$ ($f \subseteq g$) se
	
	$$
	\forall x \: f(x) \downarrow \Rightarrow g(x) \downarrow \text{ e } f(x) = g(x)
	$$
	Ovvero \textit{f} è un pezzo di \textit{g} solo se nei punti in cui è definita è uguale a \textit{g}.
	
	\item \textbf{$g(x,y)$}: durante la \textbf{riduzione} $A \leq_m B$ si cerca una funzione $g(x,y)$ calcolabile e tale che vista come funzione di $y$ appartenga a $B$ solo se $x \in A$. Tipicamente viene utilizzata una funzione costante rispetto ad $y$, tuttavia in alcuni casi, ad esempio quando c'è la condizione $\phi_x(x) > x$ è necessario usare anche $y$, in modo che venga soddisfatta la condizione, ad esempio $g(x,x) > x$. Alternativamente $y$ viene utilizzato quando la funzione deve avere determinate proprietà per qualche $n \in \mathbb{N}$.
	
	\item \textbf{Riduzioni tipiche:}
	\begin{itemize}
		\item $K \leq_m A$ per provare che $A$ non è ricorsivo
		\item $A \leq_m K$ per provare che $A$ è RE (se so che $A$ non è ricorsivo, posso definire $SC_A$ per provare che è RE, senza fare la riduzione)
		\item $\overline{K} \leq_m A$ per provare che $A$ non è RE.
	\end{itemize}
	
	\item Una parte finitaria di una funzione totale è sempre calcolabile
\end{itemize}

\section{Minimalizzazione illimitata}\label{ex:minima}

Data $f : \mathbb{N}^{k+1} \rightarrow \mathbb{N}$, si vuole definire $h : \mathbb{N}^{k} \rightarrow \mathbb{N}$ tale che calcoli il minimo $z$ che azzera la funzione $f$, ovvero:

$$
h(\vec{x}) = \mu z . f(\vec{x},z)
$$

Ci sono dei problemi se $f(\vec{x},z)$ è sempre diversa da 0, perché in questo caso la funzione $h$ è $\uparrow$. In modo simile c'è un problema se esiste un valore $z' < z$ per il quale la funzione $f$ è indefinita perché, anche in questo caso, la funzione $h$ risulta essere $\uparrow$.

Più formalmente, la minimalizzazione illimitata può essere vista come:

$$
h(\vec{x}) = \mu z.f(\vec{x},z) = \begin{cases}
\text{minimo \textit{z} tale che }f(\vec{x},z) = 0 \text{ se esiste, e }\forall z' < z, f(\vec{x}, z') \downarrow \neq 0. \\
\uparrow \text{ altrimenti}
\end{cases}
$$

La classe $\mathcal{C}$ delle funzioni URM calcolabili è chiusa rispetto all'operazione di minimalizzazione illimitata, ovvero, se $f : \mathbb{N}^{k+1} \rightarrow \mathbb{N}$ è in $\mathcal{C}$, allora anche $\mu z . f(\vec{x},z)$ è in $\mathcal{C}$.

\textbf{Dimostrazione:} Sia \textit{P} il programma URM in forma normale che calcola \textit{f}. L'idea è quella di eseguire \textit{P} incrementando via via un contatore, e quando viene trovato un valore che azzera la funzione, questo viene ritornato.

Si tratta quindi di scrivere un programma che esegua il programma $P$, ogni volta su un valore diverso e in modo che non ci siano conflitti sui registri. Come prima cosa è quindi necessario stabilire il numero di registri utilizzato da \textit{P}:

$$
m = \max\big(\rho(P), k\big)
$$

Dopodiché è possibile definire un programma \textit{P'} che:

\begin{enumerate}
	\item Copia il contenuto dei registri $r_1, \ldots, r_k$ in $r_{m+1}, \ldots, r_{m+k}$
	\item Esegue il programma \textit{P} con in input il valore dei registri $r_{m+1}, \ldots, r_{m+k}$ e pone l'output del programma nel registro $r_1$
	\item Controlla se il contenuto del registro è 0. Se è 0 ritorna \textit{z}, altrimenti continua incrementando \textit{z}.
\end{enumerate}

\begin{lstlisting}[language=URM]
T([1..k], [m+1..m+k])
P[m+1, ..., m+k+1 -> 1]
J(1, m+k+2, END) #LOOP
S(m+k+1)
J(1,1,LOOP)
#END
\end{lstlisting}

Dal momento che è possibile trovare un programma che calcola la minimalizzazione illimitata, questa è calcolabile.

\section{Funzioni primitive ricorsive}

La classe delle funzioni primitive ricorsive $\mathcal{PR}$ è la \textbf{minima} classe di funzioni che contiene le funzioni

\begin{enumerate}[(a)]
	\item $zero(x) = 0$
	\item $succ(x) = x+1$
	\item $proj(x) = U_{i}^{k}(x_1, \ldots, x_k) = x_i$
\end{enumerate}

ed è chiusa rispetto a 

\begin{enumerate}
	\item composizione
	\item ricorsione primitiva
\end{enumerate}

Si ha che $\mathcal{PR} \subsetneq \mathcal{R} = \mathcal{C}$ perché la classe $\mathcal{R}$ delle funzioni parziali ricorsive è chiusa anche rispetto la minimalizzazione illimitata.

Da notare che la minimalizzazione illimitata permette di definire delle funzioni che non sono totali e che $\mathcal{PR} \subsetneq \mathcal{R} \cap \text{Totali}$ perché esiste la funzione di Ackermann che è totale e calcolabile ma non è primitiva ricorsiva.

\section{Riduzione}

Si hanno due problemi $\mathcal{A}$ e $\mathcal{B}$ e si vuole poter dire se $\mathcal{A}$ è più facile o più difficile di $\mathcal{B}$.

Più formalmente siano $A,B \subseteq \mathbb{N}$ gli insiemi contenenti le istanze dei problemi. Si ha che il problema $\mathcal{A}$ si riduce al problema $\mathcal{B}$ se 

$$
\forall x \in A \Leftrightarrow f(x) \in B
$$

Quindi, se $A \leq_m B$:

\begin{enumerate}
	\item Se $B$ è ricorsivo, anche $A$ è ricorsivo
	\item Se $A$ non è ricorsivo, anche $B$ non è ricorsivo
	\item Se $B$ è RE, anche $A$ è RE
	\item Se $A$ non è RE, anche $B$ non è RE
\end{enumerate}

La dimostrazione viene effettuata ragionando sulle funzioni caratteristiche/semi-caratteristiche dei due insiemi e combinandole con la funzione $f$ di riduzione.

Ad esempio, si ha che $A$ è RE $\Leftrightarrow A \leq_m K =  \{ x \: | x \in W_x\: \}$ ($K$ è RE). questo perché

\begin{itemize}
	\item $(\Leftarrow)$: $K$ è RE e quindi $SC_K$ è calcolabile. Dal momento che $f$ è funzione di riduzione, questa è calcolabile e $x \in A \Leftrightarrow f(x) \in K$, quindi si può definire $SC_A(x) = SC_K(f(x))$. La funzione semi-caratteristica di $A$ è calcolabile perché è ottenuta componendo funzioni calcolabili e quindi anche $A$ è RE.
	
	\item $(\Rightarrow)$: $A$ è RE e quindi $SC_A(x)$ è calcolabile. Bisogna trovare una funzione di riduzione $f : \mathbb{N} \rightarrow \mathbb{N}$  tale che  $x \in A \Leftrightarrow f(x) \in K$.
	
	Definiamo
	
	$$
	g(x,y) = SC_A(x) = \begin{cases}
	1& x \in A \\
	\uparrow &\text{altrimenti}
	\end{cases}
	$$ 
	
	dato che $g$ è calcolabile, per il teorema SMN esiste $f : \mathbb{N} \rightarrow \mathbb{N}$ tale che $\phi_{f(x)}(y)= g(x,y)$.
	
	Si ha quindi che $f$ è proprio la funzione di riduzione $A \leq_m K$ perché:
	
	\begin{itemize}
		\item Se $x \in A$: $\phi_{f(x)}(y) = 1$ e quindi $f(x) \in W_{f(x)}$, pertanto $f(x) \in K$.
		\item Se $x \notin A$: $\phi_{f(x)}(y) = \uparrow$ e quindi $f(x) \notin W_{f(x)}$, pertanto $f(x) \notin K$.
	\end{itemize}
\end{itemize}
 
\section{Teorema di Rice}

Ogni proprietà del comportamento di un programma non è decidibile.

Sia $A \subseteq \mathbb{N}$ una proprietà del comportamento di un programma (insieme saturo). Se la proprietà non è banale, l'insieme non è ricorsivo. Con proprietà banale si intende $A \neq \mathbb{N}$ e $A \neq \emptyset$. 

\subsection{Dimostrazione per riduzione}

L'obiettivo è quello di dimostrare che $K = \{ x \: | x \in W_x\: \} \leq_m A$. $K$ è noto non essere ricorsivo.

Vogliamo quindi trovare una funzione $f : \mathbb{N} \rightarrow \mathbb{N}$ calcolabile e totale, tale che $x \in K \Leftrightarrow f(x) \in A$.

Consideriamo un programma $e_0 \in \mathbb{N}$ tale che $\phi_{e_0} = \emptyset$ (ovvero che non termina mai). Questo programma può trovarsi in $A$ o in $\overline{A}$.

\begin{itemize}
	\item $e_0 \in \overline{A}$: Sia $e_1 \in A$ un programma qualsiasi, ed esiste perché $A \neq \emptyset$.
	
	Possiamo definire
	
	$$
	g(x,y) = \begin{cases}
	\phi_{e_1}(y) &x \in K \\
	\phi_{e_0}(y) = \uparrow & x \notin K
	\end{cases} = \phi_{e_1}(y) \cdot \mathbb{1} \big( \phi_x(x) \big)
	$$ 
	
	Questa funzione è calcolabile e quindi è possibile utilizzare il teorema SMN per trovare una funzione $f : \mathbb{N} \rightarrow \mathbb{N}$ calcolabile e totale, tale che 
	
	$$
	g(x,y) = \phi_{f(x)}(y) \forall x,y
	$$
	
	e che può essere usata come funzione di riduzione.
	
	Questo perché:
	\begin{itemize}
		\item Se $x	\in K$ allora $\phi_{f(x)}(y) = \phi_{e_1}(y) \: \forall y$ e dato che $A$ è saturo e che $e_1 \in A$, allora anche $f(x) \in A$ perché essendo $A$ saturo contiene tutti gli indici che calcolano $\phi_{e_1} $.
		\item Se $x \notin K$ allora $\phi_{f(x)}(y) = \phi_{e_0}(y) \: \forall y$ e allo stesso modo dato che $e_0 \notin A$ anche $f(x) \notin A$, perché se $f(x) \in A$, allora anche $e_0$ dovrebbe appartenere ad $A$, ma questo è assurdo per ipotesi.
	\end{itemize}
	
	Si ha quindi che $f$ è la funzione di riduzione $K \leq_m A$ ed essendo $K$ non ricorsivo, anche \textbf{$A$ non è ricorsivo}.
	
	\item $e_0 \in A$: Sia $B = \overline{A}$, $B \neq \emptyset$ $B \neq \mathbb{N}$ e $B$ è saturo perché è l'insieme complementare di un insieme saturo, ed infine $e_0 \notin B$. Per $B$ valgono quindi tutte le condizioni del punto precedente, e quindi anche $B = \overline{A}$ non è ricorsivo.
\end{itemize} 


\section{Relazione tra R e RE}

Un insieme $A \subseteq \mathbb{N}$ è ricorsivo se la sua funzione caratteristica $\mathcal{X}_A$ è calcolabile

$$
\mathcal{X}_A(x) = \begin{cases}
1 &x \in A \\
0 &x \notin A
\end{cases} 
$$

Invece è ricorsivamente enumerabile se la funzione semi-caratteristica $SC_A$ è calcolabile:

$$
SC_A(x) = \begin{cases}
1 &x \in A \\
\uparrow &x \notin A
\end{cases} 
$$

Si ha quindi che $A \subseteq \mathbb{N}$ è ricorsivo $\Leftrightarrow$ $A$, $\overline{A}$ sono RE.

\subsection{$A$ ricorsivo $\Rightarrow$ $A$, $\overline{A}$ RE. }

Se $A$ è ricorsivo, la sua funzione caratteristica è calcolabile ed è definita come

$$
\mathcal{X}_A(x) = \begin{cases}
1 &x \in A \\
0 &x \notin A
\end{cases} 
$$

Si può quindi definire la funzione semi-caratteristica come minimalizzazione illimitata di $\mathcal{X}_A$:

$$
SC_A(x) = \mathbb{1} \Bigg( \mu w. \overline{sg} \Big( \mathcal{X}_A(x) \Big) \Bigg)
$$

così facendo si ottiene una funzione calcolabile che vale 1 se $x \in A$ (perché il segno negato vale 0) e risulta indefinita se $x \notin A$, perché la minimalizzazione non termina dato che il segno negato sarà sempre uguale a 1.

Per dimostrare che anche $\overline{A}$ è RE basta osservare che se $A$ è ricorsivo, anche il suo complementare $\overline{A}$ è ricorsivo e quindi è possibile ripetere la stessa dimostrazione utilizzando $\mathcal{X}_{\overline{A}}$ per dimostrare che anche $\overline{A}$ è RE.

\subsection{$A$, $\overline{A}$ RE $\Rightarrow$ $A$ ricorsivo}

Le due funzioni $SC_A$ e $SC_{\overline{A}}$ sono calcolabili e quindi esistono due programmi $e_1$ ed $e_2$ le cui funzioni coincidono con le due funzioni semi-caratteristiche.

Si può quindi definire un programma che esegue i due programmi in parallelo e in base a quello che termina ritorna 0 oppure 1.

Questa funzione può essere definita come 

$$
\mathcal{X_A}(x) = \Bigg( \mu w . S\bigg( e_1, x, (w)_1, (w)_2\bigg)  \vee S\bigg( e_2, x, (w)_1, (w)_2\bigg)  \Bigg)_1
$$

ed è calcolabile perché definita per minimalizzazione illimitata.

Alcune osservazioni:

\begin{itemize}
	\item La funzione $S(e, \vec{x}, y, t)$ vale 1 se il programma $e$ su input $x$ termina entro $t$ passi producendo come output $y$.
	\item La minimalizzazione su $w$ viene effettuata per simulare l'avanzamento dei passi. Vengono utilizzati gli esponenti della rappresentazione binaria di $w$ perché così durante la minimalizzazione vengono provati tutti i possibili valori.
	\item Perché tutto funzioni è necessario che la funzione $\phi_{e_2}$ sia uguale a $\overline{\text{sg}} (SC_{\overline{A}} (x))$, perché deve valere 0 quando $x \in \overline{A}$.
\end{itemize}

\section{Teorema di struttura}

Sia $P(x)$ un predicato $k$-ario, $P(x)$ è semi-decidibile se e solo se esiste $Q(t,x)$ decidibile e tale che

$$
P(\vec{x}) \equiv \exists t \: Q(t,\vec{x})
$$

Ovvero il predicato $P$ è una generalizzazione di un predicato decidibile che viene calcolato su più punti. In generale, quantificare esistenzialmente un predicato decidibile lo rende semi-decidibile.

Questo perché la funzione semi-caratteristica di $P$ non farà altro che applicare la minimalizzazione illimitata alla funzione caratteristica di $Q$ per cercare un valore che soddisfa $Q$:

\begin{align*}
	SC_P(x) &= \begin{cases}
		1 \text{ se } P(x)\\
		\uparrow \text{ altrimenti}
	\end{cases} \\
	&= \mathbb{1} \bigg( \mu t. \text{``} Q(t,x)\text{''} \bigg) \\
	&= \mathbb{1} \bigg( \mu t. | \: \mathcal{X}_Q(t,x) - 1 \: | \bigg)
\end{align*}

Viceversa, assumendo $P(x)$ semi-decidibile, si ha che $SC_P$ è calcolabile, ovvero esiste un certo programma $e$ tale che $\phi_e = SC_P$.

Quando il predicato $P(x)$ è vero, si ha che $SC_P(x) = 1 = \phi_e(x)$, ma questo vuol dire che il programma $e$ termina su input $x$ dopo un certo numero di passi $t$. Formalmente:

$$
\phi_e(x) = 1 \Leftrightarrow \phi_e(x) \downarrow \Leftrightarrow \exists t \: H(e,x,t) = 1
$$

Dove $H$ è la funzione che ritorna 1 se la macchina $e$ termina entro $t$ passi sull'input $x$ che sappiamo essere calcolabile.

Si ha quindi che, indicando il predicato $Q(t,x) \equiv \exists t \: H(e,x,t)$ è decidibile e $P(x) \equiv \exists t \: Q(t,\vec{x})$.

Utilizzando questo teorema è poi possibile andare a dimostrare il \textbf{teorema di proiezione}, ovvero che la classe RE è chiusa rispetto la quantificazione esistenziale. Lo si fa estraendo dal predicato semi-decidibile quello decidibile e poi si ragiona su come esprimere il predicato semi-decidibile come l'altro predicato semi-decidibile:

 $$
 P'(\vec{y}) \equiv \exists x.\exists t . Q(t,x, \vec{y}) \equiv \exists w . \underbrace{Q((w)_1, (w)_2, \vec{y})}_{Q'}
 $$



\section{Teorema di Rice-Shapiro}

Le proprietà del comportamento (funzione calcolata) dei programmi possono essere decidibili se sono finitarie, ovvero dipendono da un numero finito di argomenti (es: \textit{la funzione termina su 5}).

Più formalmente sia $\mathcal{A}\subseteq \mathbb{C}$ e $A = \{x \: | \: \phi_x \in \mathcal{A} \}$ ($A$ saturo). Se $A$ è RE, allora

$$
\forall f \: \in \mathcal{C} . \bigg( f \in A \Leftrightarrow \exists \vartheta \subseteq f \text{ e } \vartheta \in \mathcal{A}  \bigg)
$$

A parole, sia $\mathcal{A}$ un sottoinsieme delle funzioni calcolabili e $A$ l'insieme che contiene gli indici che calcolano queste funzioni. Se $A$ è RE, allora una funzione $f$ calcolabile appartiene ad $\mathcal{A}$ solo se esiste una sua parte finita che appartiene ad $\mathcal{A}$.

\subsection{Dimostrazione}

La dimostrazione si fa in due passi:

\begin{enumerate}
	\item $\exists f, f \notin \mathcal{A}$ e $\exists \vartheta \subseteq f, \vartheta \in \mathcal{A} \Rightarrow$ $A$ non è RE.
	\item $\exists f, f \in \mathcal{A}$ e $\forall \vartheta \subseteq f, \vartheta \notin \mathcal{A} \Rightarrow$ $A$ non è RE.
\end{enumerate}

\paragraph{Caso 1}

Sia $f \notin \mathcal{A}$ e $\vartheta \subseteq f, \vartheta \in \mathcal{A}$. Per provare che $A$ non è RE viene fatta la riduzione $\overline{K} \leq_m A$.

Definiamo quindi 

\begin{align*}
	g(x,y) = \begin{cases}
		\vartheta(y) & x \in \overline{K} \\
		f(y) & x \in K
	\end{cases} = \begin{cases}
	\uparrow & x \in \overline{K} \text{ e } y \notin dom(\vartheta) \\
	f(y) = \vartheta(y)  & x \in \overline{K} \text{ e } y \in dom(\vartheta) \\
	f(y) & x \in K  
	\end{cases} \\
	&= \begin{cases}
	f(y) &x \in K \text{ oppure } y \in dom(\vartheta)\\
	\uparrow & \text{ altrimenti}
	\end{cases} 
\end{align*}

Essendo $x \in K$ semi-decidibile, il predicato $Q(x,y) \equiv x \in K \text{ oppure } y \in dom(\vartheta)$ è semi-decidibile e pertanto la funzione \textit{g} è calcolabile perché può essere definita come:

$$
g(x,y) = f(y) \cdot SC_Q(x,y)
$$

Essendo \textit{g} calcolabile, per il teorema SMN esiste $S : \mathbb{N} \rightarrow \mathbb{N}$ calcolabile e totale, tale che $g(x,y)  = \phi_{S(x)}(y)$ che può funzionare da funzione di riduzione:

\begin{itemize}
	\item $x \in \overline{K}$: $\phi_{S(x)}(y) = g(x,y) \:\forall y$ ed in particolare $ =  \vartheta(y) \: \forall y \Rightarrow \phi_{S(x)} = \vartheta$ ed essendo $A$ saturo, $S(x) \in A$.
	\item $x \in K$: $\phi_{S(x)}(y) = g(x,y) = f(y)\:\forall y$ da ui segue che $\phi_{S(x)} = f \notin A$ e quindi dato che $A$ è saturo, anche $S(x) \notin A$.
\end{itemize}

\paragraph{Caso 2}

Sia $f \in \mathcal{A}$ e tutte le sue parti finite non sono in $\mathcal{A}$, voglio arrivare a dire che $A$ non è RE.

Anche in questo caso si fa la stessa riduzione.

\begin{align*}
g(x,y) &= \begin{cases}
f(y) &\text{ se } \neg H(x,x,y) \\
\uparrow & \text{ altrimenti}
\end{cases} \\
&= f(y) \cdot \mathbb{1} \big( \mu w . \mathcal{X}_H (x,x,y) \big)
\end{align*}

Essendo \textit{g} calcolabile, per SMN esiste $S$ che funziona da funzione riduzione:

\begin{itemize}
	\item $x \in \overline{K}$: $\forall y \: \neg H(x,x,y) = 0$ e quindi $\phi_{S(x)}(y) = f(y) \: \forall y$, ovvero $\phi_{S(x)} = f$ e dato che $A$ è saturo e $f \in A$, anche $S(x) \in A$.
	\item $x \in K$: $\exists y_0$ tale che $H(x,x,y_0) = 1$, perché $x$ termina in un certo numero di passi quando riceve se stesso in input. Tra tutti i possibili valori, prendo il minimo $y_0$ tale che $\forall y < y_0 \: \neg H(x,x,y) = 1$ e $\forall y \geq y_0 \: H(x,x,y) = 1 $.
	
	Si ha quindi che $\phi_{S(x)} (y) = g(x,y) = \begin{cases}
	f(y) &y < y_0 \\ 
	\uparrow &\text{altrimenti}
	\end{cases} $.
	
	Essendo che $\phi_{S(x)}$ è una parte finita di $f$, si ha che $\phi_{S(x)} \notin A$ e quindi $S(x) \notin A$. 
\end{itemize}

\section{Secondo teorema di Ricorsione}

Sia $h : \mathbb{N} \rightarrow \mathbb{N}$ calcolabile e totale. Allora esiste $e \in \mathbb{N}$ tale che $\phi_e = \phi_{h(e)}$. 