% !TEX encoding = UTF-8
% !TEX TS-program = pdflatex
% !TEX root = computabilità e algoritmi.tex
% !TEX spellcheck = it-IT

\section{Appello 2015-06-30 } 

Fatta da Baldan a lezione

\subsection{Esercizio 1}

Dimostrazione del teorema di struttura.

\subsection{Esercizio 2}

$A, B \subseteq \mathbb{N}$, $A \leq_m B$ se esiste $f : \mathbb{N} \rightarrow \mathbb{N}$ calcolabile e totale tale che $x \in A$ se e solo se $f(x) \in B$.

\`{E} vero che per ogni $A \subseteq \mathbb{N}$, $A \leq_m A \cup \{0\}$? Che condizioni possono essere aggiunte perché questo sia vero?

\subsubsection{Soluzione}

Sembra ragionevole pensare che se si è in grado di decidere l'appartenenza a $A \cup \{0\}$, sia possibile definire anche l'appartenenza per $A$.

Conviene quindi provare questa strada, ovvero cercare una funzione $f$ tale che se $x \in A$, allora $f(x) \in A\cup \{0\}$.

Se $x =0$ e $x \in \overline{A}$, devo mandarlo in $\overline{A \cup \{0\} }$.

$$
f(x) = \begin{cases}
x &x \neq 0 \\
x_0 &x = 0
\end{cases}
$$

Riassumendo, c'è un caso banale se $0 \in A$, perché $A \cup \{0\} = A$ e quindi $A \leq_m A$ e come funzione di riduzione posso utilizzare l'identità.

Se invece $0 \in \overline{A}$, basta prendere un elemento $x_0 \neq 0 \notin A$ e ritornare quello, ovvero la funzione di riduzione risulta essere la $f$ sopra riportata.

Resta da dimostrare che questa funzione è calcolabile e che funzioni da funzione di riduzione.
Ovviamente è calcolabile perché può essere definita come

$$
f(x) = x \cdot \text{sg}(x) + x_0 \cdot  \overline{\text{sg}}(x)
$$

Per dimostrare che $f$ è funzione di riduzione da $A \leq_m A \cup \{0\}$:

\begin{enumerate}
	\item $x \in A$ ovvero $x \neq 0$, $f(x) = x \in A\cup\{0\}$ ed è corretto.
	\item $x \notin A$, devo distinguere i due casi, se $x \neq 0$ si ha $f(x) = x$, ovvero $x \in \overline{A \cup \{0\}} $ ed è quello che vogliamo. Se invece $x = 0$, $f(x) = x_0$ e per costruzione $x_0 \in \overline{A \cup \{0\}}$.
\end{enumerate}

$f$ si comporta quindi come una funzione di riduzione, ma funziona perché esiste $x_0$ e quindi il ragionamento fatto finora vale $A \neq \mathbb{N}  - \{0\}$.

Se $A = \mathbb{N}  - \{0\}$, si ha che $A \cup \{0\} = \mathbb{N}$ e quindi non è possibile trovare un'immagine in $\overline{A \cup \{0\}}$ per $0$.

Tornando alla domanda, non è sempre vero che $A \leq_m A \cup \{0\}$, perché se $A = \mathbb{N}-\{0\}$ una funzione di riduzione dovrebbe assicurare che $f : \mathbb{N} \rightarrow \mathbb{N}$ calcolabile e totale che $x \in A$ se e solo se $f(x) \in A \cup \{0\}$ perché non si riesce a trovare $f(0) \notin \mathbb{N}$.

Per rispondere alla seconda parte si ha che $A \neq \mathbb{N} - \{0\}$ è condizione necessaria e sufficiente per far valere la riduzione: \textbf{sufficiente} perché $A \neq \mathbb{N} - \{0\} \Rightarrow A \leq_m A \cup \{0\}$, due casi:

\begin{itemize}
	\item Se $0 \in A$: la riduzione può essere fatta trivialmente dalla funzione identità.
	\item Se $0 \notin A $: sia $x_0 \neq 0$, $x_0 \notin A$ ed $x_0$ esiste per ipotesi. Così facendo è possibile definire $f(x)$ come prima e per quanto dimostrato è funzione di riduzione. 
\end{itemize}

La condizione è anche \textbf{necessaria} perché sennò vale il controesempio iniziale.

In alternativa, un'altra condizione solamente sufficiente è quella che $0 \in A$, è più stretta rispetto la prima e quindi è una risposta peggiore.

\subsection{Esercizio 3}

Gli esercizi 3 e 4 sono spesso di questo tipo.

Studiare la ricorsività dell'insieme $A = \{ x \in \mathbb{N} \: | \: x \in E_x \cup W_x \}$, ovvero dire se $A$ è ricorsivo o ricorsivamente enumerabile e com'è il suo complementare.

\subsubsection{Soluzione}

Per sapere se $x \in E_x$ posso provare tutti gli input, e questo si può fare anche se in alcuni input non termina, andando ad eseguire un po' di passi per ognuno dei valori fino a che non termina.
Allo stesso modo posso provare se $x \in W_x$.

Sembrerebbe quindi che $A$ sia ricorsivamente enumerabile.

Per dimostrare che $A$ non è ricorsivo posso provare a ridurre $K \leq_m A$.

Se avessi un programma $P$ che prende in input $x$ e risponde alla domanda $x \in A$? SI/NO, potrei usare questo programma per risolvere l'halting problem, perché potrei estendere il programma in modo che prima venga eseguito $P_x(x)$ su un input $y$ non rilevante e poi fornire il risultato a $P$.

Voglio quindi una funzione $f$ calcolabile e totale, tale che $x \in K$ se e solo se $f(x) \in A$.

Definisco quindi 

$$
g(x, y) = \begin{cases}
\phi_x(x) & x \in K \\
\uparrow &\text{altrimenti}
\end{cases} = \Phi_U(x,x)
$$

questa funzione è calcolabile e quindi per il teorema SMN esiste $f : \mathbb{N} \rightarrow \mathbb{N}$ calcolabile e totale, tale che $\phi_{f(x)}(y) = g(x,y)$.

$f$ è funzione di riduzione $K \leq_m A$ perché se:

\begin{itemize}
	\item $x \in K$: $\phi_{f(x)}(y) = g(x,y) = \phi_x(x) \downarrow \forall y$. In particolare $W_{f(x)} = \mathbb{N}$ e quindi per ogni valore di $x$ lo trovo nel dominio della funzione e quindi sta anche nell'unione tra il dominio e il codominio e quindi $f(x) \in A$.
	\item $x \notin K$: $\phi_{f(x)}(y) = g(x,y) \uparrow \forall y$, ovvero $W_{f(x)} = \emptyset$ e $E_{f(x)} = \emptyset$ e quindi di sicuro $f(x)$ non compare nell'unione dei due, in particolare $f(x) \notin A$.
\end{itemize}

Dato che $K$ si riduce ad $A$ e che $K$ non è ricorsivo, anche $A$ non è ricorsivo.

Per dimostrare che $A$ è RE si può provare a ridurre $A \leq_m K$ ma è sconsigliato, oppure si può definire la funzione semi-caratteristica di $A$ come funzione calcolabile.
Ovvero:

$$
SC_A(x) = \begin{cases}
1 &x \in A \\
\uparrow &\text{altrimenti}
\end{cases}  = p(x) =  1 \Big( \mu w . \underbrace{S(x, (w)_1, x, (w)_2)}_{x \in E_x} \vee \underbrace{H(x,x,(w)_3)}_{x \in W_x} \Big)
$$

$p(x)$ è quindi la funzione semi caratteristica  perché\footnote{Questa è opzionale, diciamo che se c'è del tempo da perdere conviene farla}:

\begin{itemize}
	\item $x \in A$:  $ x \in E_x $ oppure $ x \in W_x$
	\begin{itemize}
		\item $x \in E_x$: esiste $z$ tale che $\phi_x(z) = x$ e quindi esiste $t$ tale che $S(x,z,x,t)$ e quindi esiste anche un $w$ le cui componenti prime producono lo stesso risultato
		\item $x \in W_x$: esiste $t$ tale che $H(x,x,t)$, quindi esiste $w$ tale che $(w)_3 = t$.
	\end{itemize}
	In entrambi i casi la minimalizzazione termina e produce $p(x) = 1$. Da notare che le osservazioni effettuate sono invertibili e quindi valgono tutte in se e solo se.
\end{itemize}

La funzione $p(x)$ è ovviamente calcolabile per minimalizzazione illimitata.

Possiamo quindi dire che $A$ non è ricorsivo e che è RE, quindi per forza di cose $\overline{A}$ non è RE, perché altrimenti $A$ sarebbe ricorsivo.

\subsection{Esercizio 4}

Studiare la ricorsività dell'insieme $ B = \{ x \in \mathbb{N} \:|\: 1 \leq | E_x| \leq 2 \}$. Dire se è saturo.

\subsubsection{Soluzione}

L'insieme $B$ può essere definito come un insieme di programmi che calcolano funzioni con un codominio di cardinalità 1 o 2, quindi per definizione è saturo.

\begin{align*}
	B &= \{ x \in \mathbb{N} | \phi_x \in \mathcal{B} \} \\
	\mathcal{B} &= \{ f \in \mathcal{C} | 1 \leq |cod(f) | \leq 2 \} \subseteq \mathcal{C}
\end{align*}

$B$ probabilmente non è RE perché per determinare se un programma appartiene a $B$ è necessario provare tutti i possibili valori di input e verificare che in output vengono ottenuti al massimo 2 valori distinti.

Considero la funzione 

$$
f(x) = x \dotminus 2
$$

Si ha che $cod(f) = \mathbb{N}$ e quindi $f \notin \mathcal{B}$ .

Possiamo trovare una funzione

$$
\vartheta(x) = \begin{cases}
	x \dotminus 2 &x \leq 2 \\
	\uparrow &\text{altrimenti}
\end{cases}
$$

Per come è definita $\vartheta \subseteq f$, è finita e $cod(\vartheta) = \{0\}$, ovvero ha cardinalità 1 e quindi $\vartheta \in \mathcal{B}$. Abbiamo quindi una funzione $f$ che non sta in $\mathcal{B}$ ma che ammette una parte finita in $\mathcal{B}$ e quindi per il teorema di Rice Shapiro, $B$ non è RE.

Per quanto riguarda $\overline{B}$, $\overline{\mathcal{B}} = \{ f \:|\: |cod(f)| < 1 \text{ oppure } |cod(f)| > 2 \}$, la funzione $g = \vartheta \notin \overline{\mathcal{B}}$ e $\vartheta \subseteq g$, $\vartheta$ finita e $\vartheta \notin \overline{\mathcal{B}}$ si ha che $\overline{B}$ non è RE.

Quindi sia $B$ che $\overline{B}$ non sono RE e quindi non sono neanche ricorsivi\footnote{Specificarlo è bene}.

\subsection{Esercizio 5}

Tipicamente è un'applicazione del secondo teorema di ricorsione.

Dato l'insieme $C = \{ x \in \mathbb{N} \: | \: [0,x] \subseteq W_x \} $, enunciare il secondo teorema di ricorsione ed utilizzarlo per dimostrare che l'insieme $C$ non è saturo.

\subsubsection{Soluzione}

Siccome dipende da una proprietà di un programma, è probabile che non sia saturo.

L'idea è quella di trovare un certo indice $e$ tale che $W_e = [0,e]$. Se riusciamo a trovare un programma del genere abbiamo sostanzialmente concluso, perché siamo certi che esistono infiniti indici che calcolano la stessa funzione $\phi_e$ e tra tutti questi ce ne è sicuramente uno che è più grande di $e$, ovvero esiste $e' \in \mathbb{N}$ tale che $\phi_{e'} = \phi_e$ e $e' > e$.

Trovato questo $e'$ abbiamo $[0, e'] \nsubseteq W_{e'} = W_e = [0,e]$ e quindi $e' \notin C$ e pertanto $C$ non è saturo.

Dimostriamo che $e$ esiste e per farlo consideriamo la funzione

$$
g(x,y) = \begin{cases}
0 & y \in [0,x] \\
\uparrow &\text{altrimenti}
\end{cases} = \mu z . y \dotminus x
$$

questa funzione è calcolabile e pertanto per il teorema SMN esiste $S : \mathbb{N} \rightarrow \mathbb{N}$ calcolabile e totale tale che $g(x,y) = \phi_{S(x)}(y) \forall x,y$.

Per il secondo teorema della ricorsione esiste un indice $e$ tale che $\phi_e = \phi_{S(e)}$ e quindi

$$
\phi_e(y) = \phi_{S(e)}(y) = g(e,y) = \begin{cases}
0 & y \in [0,x] \\
\uparrow &\text{altrimenti}
\end{cases}
$$

ovvero $W_e = [0,e]$ e quindi $e$ è proprio l'indice che cercavamo.
