\chapter{Soluzione compiti}

\section{Appello 2015-07-16}

\subsection{Esercizio 1 - Teorema di proiezione}

$$
P(x,\vec{y}) \text{ semi-dedibile} \Rightarrow \exists x \: P(x,\vec{y}) \text{ semi-decidibile}
$$

\subsubsection{Soluzione}

Per il teorema di struttura: $P(\vec{x})$ è semi-decidibile $\Leftrightarrow$ esiste un predicato $Q(\vec{x}, y)$ tale che $P(\vec{x}) \equiv \exists y \: Q(\vec{x}, y)$.

Quindi essendo $P(x,\vec{y})$ è semi-decidibile, per il teorema di struttura esiste $Q(x, \vec{y}, z)$ decidibile.

Si può quindi riscrivere $P'$ come:

$$
P'(\vec{y}) \equiv \exists x \: \exists z Q(x, \vec{y}, z) \equiv \exists w . Q( (w)_1, \vec{y}, (w)_2)
$$

con $Q$ decidibile e quindi per il teorema di struttura $P'(\vec{y})$ è semi-decidibile.

\subsection{Esercizio 2 - Teoria sulla riduzione}

$$
A \text{ è RE } \Leftrightarrow A \leq_m K
$$

\subsubsection{Soluzione}

\paragraph{$(\Rightarrow)$} 

Essendo $K$ RE, la sua funzione semi-caratteristica $SC_K(x)$ è calcolabile ed esiste per ipotesi una funzione di riduzione $f : \mathbb{N} \rightarrow \mathbb{N}$.

Si possono quindi combinare queste due funzioni per definire la funzione semi-caratteristica di $A$:

$$
SC_A(x) = SC_K(f(x))
$$ 

la quale risulta calcolabile e quindi $A$ è RE.

\paragraph{$(\Leftarrow)$}

Per ipotesi $A$ è RE, quindi la sua funzione semi-caratteristica è calcolabile.

Serve quindi una funzione di riduzione $f : \mathbb{N} \rightarrow \mathbb{N}$ per dimostrare che $A$ si riduce a $K$.

Vogliamo quindi trovare una funzione tale che se $x \in A$, allora $f(x) \in K$, ovvero $\phi_{f(x)}(x) = \downarrow$ e che se $x \notin A$ allora $f(x) \notin K$ e quindi $\phi_{f(x)}(x) = \uparrow $.

Possiamo quindi definire la funzione

$$
g(x,y) = \begin{cases}
1 &\text{se } x \in A \\
\uparrow &\text{altrimenti}
\end{cases} = SC_A(x)
$$

Questa funzione è calcolabile perché equivale alla funzione semi-caratteristica di $A$. 

Possiamo quindi utilizzare il teorema SMN per trovare una funzione calcolabile e totale che ritorni un programma che calcoli $g$:

$$
\phi_{f(x)}(y) = g(x,y)
$$ 

La funzione $f$ è quindi la funzione di riduzione cercata, perché:

\begin{itemize}
	\item $x \in A$: $\phi_{f(x)}(y)$ è $\downarrow \: \forall y$ ed in particolare $\phi_{f(x)}(x) \downarrow$ e quindi il programma $f(x) \in K$
	\item $x \notin A$: $\phi_{f(x)}(y)$ è sempre indefinita ed in particolare $\phi_{f(x)}(x) $ è indefinita e quindi $f(x) \notin K$.
\end{itemize}


\subsection{Esercizio 3 - Studio della ricorsività (riduzione)}

$$
A = \{ x \in \mathbb{N} \:|\: \exists y \in E_x, \exists z \in W_x, x = y \cdot z  \}
$$

\subsubsection{Soluzione}

L'insieme $A$ sembra essere RE perché per stabilire l'appartenenza ad $A$ di una determinato programma è necessario andare ad eseguirlo su un certo numero di valori, fino a che non viene trovata una coppia che soddisfa la condizione.

La funzione semi-caratteristica può quindi essere scritta come:

$$
SC_A(x) = 1 \bigg( \mu w . \Big( S\big(x, (w)_1, (w)_2, (w_3)\big) \wedge H\big(x, (w)_4, (w)_5\big) \wedge \big( x = (w)_2 \cdot (w)_4 \big)  \Big) \bigg)
$$

che risulta calcolabile e quindi $A$ è RE.

Il primo termine cerca un numero di passi $(w)_3$ entro i quali il programma $x$ produce un output $(w)_2 = y$ avendo in input $(w)_1$. In pratica prova vari valori di input per trovare i valori del dominio.

Il secondo termine cerca tutti i valori di input $(w)_4$ che fanno terminare il programma $x$ in meno di $(w)_5$ passi.

$A$ sembra inoltre essere non ricorsivo e questo può essere dimostrato riducendo $K \leq_m A$.

Si vuole quindi una funzione $f : \mathbb{N} \rightarrow \mathbb{N}$ tale che se $x \in K$, ovvero il programma $x$ termina quando riceve in input se stesso, produca un programma $f(x)$ tale che ci siano almeno un elemento del dominio e uno del codominio che possono essere tra loro moltiplicati per ottenere l'indice del programma. 

Viceversa se $x \notin K$, il programma $f(x)$ non deve avere questa coppia di valori nel dominio/codominio.

\textbf{Osservazione}: basta trovare un programma che termina producendo in output 1 quando in input riceve se stesso. In questo caso si ha che $1 \in E_x$ e che $x \in W_x$ e quindi la condizione di appartenenza ad $A$ è soddisfatta.

Si può quindi definire la funzione

$$
g(x,y) = \begin{cases}
1 &\text{se } x \in K\\
\uparrow &\text{altrimenti}
\end{cases} = SC_K(x)
$$

tale funzione è calcolabile e quindi per il teorema SMN esiste una funzione $f$ tale che $\phi_{f(x)}(y) = g(x,y)$.

Questa funzione è proprio quella di riduzione perché:

\begin{itemize}
	\item $x \in K$: $\phi_{f(x)}(f(x)) = 1$ e $f(x) \in W_{f(x)}$ e quindi $f(x) \in A$.
	\item $x \notin K$: $\phi_{f(x)}$ è sempre indefinita e quindi $E_{f(x)} = W_{f(x)} = \emptyset$, non è quindi possibile trovare $y$ e $z$ e pertanto $f(x) \notin A$.
\end{itemize}

Dal momento che $A$ è RE e non è ricorsivo, $\overline{A}$ non è RE.

\subsection{Esercizio 4 - Studio della ricorsività (Rice-Shapiro)}

\begin{align*}
	B &= \{ x \:| \: x \in \mathcal{B} \} \\
	\mathcal{B} &= \{ f \: |\: |dom(f) \setminus cod(f) | \geq 2 \}
\end{align*}

Ovvero $\mathcal{B}$ contiene tutte le funzioni che hanno il dominio con più di due elementi diversi dal codominio.

\subsubsection{Soluzione}

Dal momento che $\mathcal{B}$ descrive una proprietà (non banale) di una funzione, l'insieme è saturo e quindi di sicuro $B$ non è ricorsivo.

L'idea è quindi quella di applicare Rice-Shapiro per dimostrare che $B$ non è RE. Per fare questo si può trovare una funzione $f \notin \mathcal{B}$ ma che ha almeno un parte finita $\vartheta \in \mathcal{B}$.

$$
f(x) = x \dotminus 2 = \begin{cases}
0 &x \leq 2 \\
x -2 &\text{altrimenti}
\end{cases}
$$

In questo caso si ha $dom(f) = cod(f) = \mathbb{N}$, $| dom(f) \setminus cod(f)| = 0$ e quindi $f \notin \mathcal{B}$.

Come parte finita di $f$ si può considerare

$$
\vartheta(x) = \begin{cases}
0 & x \leq 2 \\
\uparrow &\text{altrimenti}
\end{cases}
$$

Si ha che $dom(\vartheta) = \{0,1,2\}$, $cod(f) = \{0\}$ e quindi $\vartheta \in \mathcal{B}$.

Abbiamo quindi una funzione $f \notin \mathcal{B}$ che ha una parte finita $\vartheta \subseteq f$ che appartiene a $\mathcal{B}$ e quindi per Rice-Shapiro $B$ non è RE.

Resta da valutare $\overline{\mathcal{B}} = \{ f \: | \: |dom(f) \setminus cod(f)| < 2 \}$, che può essere fatto allo stesso modo.

Basta osservare che la funzione $1 \notin \overline{\mathcal{B}}$  e che ha come parte finita la funzione $\emptyset \in \overline{\mathcal{B}}$ e quindi sempre per Rice-Shapiro $\overline{\mathcal{B}}$ non è RE.


\subsection{Esercizio 5 - Secondo teorema di ricorsione}

Dimostrare che 

$$
 \exists x . W_x = \{ kx \: | \: k \in \mathbb{N} \}
$$

\subsubsection{Soluzione}

Il secondo teorema di ricorsione afferma che data una funzione calcolabile e totale $h : \mathbb{N} \rightarrow \mathbb{N}$, esiste $e \in \mathbb{N}$ tale che $\phi_{h(e)} = \phi_e$.

Come prima cosa è necessario cercare una funzione che produca un programma $f(x)$ con le caratteristiche desiderate, ovvero che termini quando riceve un multiplo di se stesso in input.

Si può quindi definire la funzione

$$
g(x,y) = \begin{cases}
k &\text{se } y = kx \text{ per qualche }k \\
 \uparrow &\text{altrimenti}
\end{cases} = \mu k . |kx - y|
$$

Essendo $g$ calcolabile, per il teorema SMN, esiste una funzione $f : \mathbb{N} \rightarrow \mathbb{N}$ calcolabile e totale, tale che il programma $f(x)$ calcoli $\phi_{f(x)}(y) = g(x,y)$.

Si ha quindi che $W_{f(x)} =  \{ kx \: | \: k \in \mathbb{N} \}$, ma noi stiamo cercando un programma $e$.
	
Però la funzione $f$ soddisfa le condizioni del secondo teorema di ricorsione, quindi $\exists e \in \mathbb{N} \text{ tale che } \phi_e = \phi_{f(e)}$ e quindi si ha che

$$
W_e = W_{f(e)} =  \{ ke \: | \: k \in \mathbb{N} \}
$$