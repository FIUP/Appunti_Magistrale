% !TEX encoding = UTF-8
% !TEX program = pdflatex
% !TEX root = InformationRetrieval.tex
% !TEX spellcheck = it-IT

\section{Assessment con ANOVA}

Dato un gruppo con diversi fattori cerca di determinare se le medie sono uguali oppure no.

Ad esempio ho 3 farmaci da dare a devi vari pazienti e devo capire se i farmaci hanno effetti diversi, assumendo che l'effetto dei farmaci posso essere modellato con delle distribuzioni gaussiane.
Ci sia aspetta che se i farmaci hanno effetti diversi i valori medi sperimentali siano diversi.

Più formalmente il metodo si basa sull'ipotesi nulla, ovvero che tutte le medie siano uguali. Ovvero tutti i farmaci hanno lo stesso effetto o nel contesto dell'IR, tutti i sistemi di IR sono uguali.

Lo stesso sistema può essere applicato per i vari componenti di un sistema IR, infatti, al variare dei componenti possiamo osservare come varia la varianza delle prestazioni del sistema (il test statistico di ANOVA si basa sulla varianza).

... continua con il discorso sui calcoli ...

Salta tutte le parti sugli errori e sul potere. Riprende dalla tabella di pagina 35.

NON si fa il discriminative power
