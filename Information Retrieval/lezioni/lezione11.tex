% !TEX encoding = UTF-8
% !TEX program = pdflatex
% !TEX root = InformationRetrieval.tex
% !TEX spellcheck = it-IT

% 3 Novembre 2016
%\section{Modello probabilistico}
%\subsection{Basi di probabilità}

\subsubsection{Classificazione di costo minore - Bayesian Decision Theory}

Prendere una decisione, ovvero assegnare ad un documento la classe rilevante o meno, ha un costo che deve essere pagato se il documento viene classificato in modo errato.

\paragraph{Conditional Risk} Il rischio di errore di classificazione per un documento $d$ viene calcolato come:

$$
R(\rel | d) = \underbrace{P(\rel | d) \lambda (\overbrace{\rel}^{\text{classe predetta}} | \overbrace{\rel}^{\text{classe vera}})}_{\text{costo per la classificazione corretta}} + \underbrace{P(\overline{rel} | d) \lambda (\rel | \notrel)}_{\text{costo per la classificazione errata}}
$$
s
\noindent $\lambda(\cdot | \cdot)$ è la cosi detta \textbf{loss function}, un numero che rappresenta il costo da pagare per la classificazione, sia se è errata, sia se è corretta.

Il rischio di classificazione positiva viene poi confrontato con quello di classificazione negativa:

$$
P(\overline{rel} | d) = P(\overline{rel}|d)\lambda(\overline{rel}|\overline{rel}) + P(rel|d)\lambda(\overline{rel}|rel)
$$

Quindi conviene andare a classificare il documento con la classe che minimizza il costo di classificazione:

$$
\boxed{R(\rel|d) < R(\notrel|d)}
$$

Per comodità notazionale indicheremo con $\lambda_{11} = \lambda (\rel | \rel)$, $\lambda_{1,0} =\lambda (\rel | \notrel)$, ecc.

Espandendo la disequazione del rischio abbiamo:

\begin{align*}
\lambda_{11}P(\rel|d) + \lambda_{10}P(\notrel| d) &< \lambda_{01}P(\rel|d) + \lambda_{00}P(\notrel | d)  \\
P(\overline{rel}|d)(\lambda_{10} - \lambda_{00}) &< P(rel|d)(\lambda_{01} - \lambda_{11})
\end{align*}

\noindent Da notare che se $\lambda_{11} = \lambda_{00} = 0$ (le classificazioni corrette non costano) e se $\lambda_{10} = \lambda_{01} = 1$, ottengo la formula basata solamente sulle probabilità che abbiamo utilizzato finora. Questa particolare funzione di perdita prende il nome di \textbf{zero-one loss function}, perché le classificazioni corrette non hanno costo mentre le classificazioni errate hanno un costo unitario.

\begin{align*}
P(\notrel|d) < P(\rel|d) \frac{(\lambda_{01} - \lambda_{11})}{(\lambda_{10} - \lambda_{00})}
\end{align*}

\noindent Con questa ri-formulazione è possibile andare a modificare i valori $\lambda$ per gestire i casi in cui le classi sono sbilanciate.
Oppure posso ri-formulare ancora:


\subsection{Probabilistic Ranking Principle e BIM}

Oltre che a classificare i documenti, noi vogliamo ordinarli rispetto alla rilevanza nei confronti della query $q$. Introduciamo quindi una variabile aleatoria $Q$, analoga a quella per i documenti della collezione. 

Secondo il \textbf{probabilistic ranking principle}, il ranking deve essere effettuato ordinando i documenti secondo la probabilità $P(C = \rel | D= d, Q= q)$.

Ci mettiamo quindi nel caso in cui utilizziamo il \textbf{Binary Indipendence Model} e una \textbf{zero-one loss function}.

\subsubsection{Classificazione di un documento come rilevante con BIM}

\begin{align*}
	P(\rel | d,q) &> 1 - P(\rel | d, q) \\
	              &> P(\notrel | d,q)
\end{align*}

Noi però non abbiamo a disposizione questi dati, quindi è necessario utilizzare la regola di Bayes per sfruttare le informazioni sui documenti e sulla query.

\begin{align*}
	P(\rel | d,q) &> P(\notrel | d,q) \\
	\frac{P(d,q|\rel)P(\rel)}{P(d,q)} &> \frac{P(d,q|\notrel)P(\notrel)}{P(d,q)} \\
	P(d|q,\rel)P(q|\rel)P(\rel) &> P(d|q,\notrel)P(q|\notrel)P(\notrel) \\
	P(d|q,\rel)P(\rel|q) P(q) &> P(d|q,\notrel)P(\notrel|q) P(q) \\
	P(d|q,\rel)P(\rel|q) &> P(d|q,\notrel)P(\notrel|q) \\
\end{align*}

\subsubsection{Ranking con BIM}

Abbiamo poi che il rapporto tra le due proprietà è una funzione di ranking equivalente a quella del probabilistic ranking principle.

$$
P(rel | d, q) \propto_q \frac{P(rel | d,q)}{P(\overline{rel} | d,q)}
$$

Si possono quindi replicare i passaggi precedenti per ottenere.

\begin{align*}
P(rel | d, q) &\propto_q \frac{P(rel | d,q)}{P(\overline{rel} | d,q)} \\
              &\propto_q \frac{ P(d|q,\rel)P(\rel|q) }{P(d|q,\notrel)P(\notrel|q)}
\end{align*}


\noindent Fissata una query $q$ il rapporto tra $P(rel|q)$ e $P(\overline{rel}|q)$ sono costanti per tutti i documenti della collezione e, dato che a noi interessa l'ordine dei documenti, possiamo semplificare i due termini.\\

$$
\boxed{P(rel | d,q) \propto_q \frac{P(d|q,rel)}{P(d|q,\overline{rel})}}
$$

\noindent Quindi se i costi sono del tipo zero-uno, questo è l'approccio ottimo. Se ci fosse un'altra funzione di perdita, sarebbe necessario portarsi dietro anche i vari $\lambda$.


\subsubsection{Come calcolo le probabilità?}

\begin{align*}
P(d|q,\rel) &= \prod\limits_{i = 1}^{|V|} p_{i}^{x_i}(1 - p_i)^{1-x_i} \\
P(d|q,\notrel) &= \prod\limits_{i = 1}^{|V|} q_i^{x_i}(1 - q_i)^{1-x_i}
\end{align*}


\noindent Andando a sostituire dentro il proporzionale ottengo:

\begin{align*}
P(rel|d,q) &\propto_q \prod\limits_{i = 1}^{|V|} \Bigg(\frac{p_i}{1-p_i} \frac{1-q_i}{q_i}\Bigg)^{x_i} \frac{1-p_i}{1-q_i} \\
		   &= \prod\limits_{i = 1}^{|V|} \Bigg(\frac{p_i}{1-p_i} \frac{1-q_i}{q_i}\Bigg)^{x_i} \prod\limits_{i = 1}^{|V|} \frac{1-p_i}{1-q_i} \\
		   &= \sum\limits_{i = 1}^{|V|} x_i \log\Bigg(\frac{p_i}{1-p_i} \frac{1-q_i}{q_i}\Bigg) + \underbrace{\sum\limits_{i = 1}^{|V|} \log\frac{1-p_i}{1-q_i}}_{\text{fissata una query è un valore costante}}
\end{align*}


Dato che $x_i$ vale 1 se il termine è presente nel documento e 0 se non lo è, posso ridurre la sommatoria considerando solo i termini presenti nel documento, al posto di tutti i termini presenti nel vocabolario. 
Ottengo quindi che, fissata una query, il ranking può essere fatto con 

$$
\boxed{P(rel|d,q) \propto_q \sum\limits_{i \in d} \log\Bigg(\frac{p_i}{1-p_i} \frac{1-q_i}{q_i}\Bigg)}
$$

\noindent Per i termini del vocabolario che non compaiono all'interno delle query, ho che la probabilità $p_i$ di trovarlo in un documento rilevante è uguale a quella di trovarlo in un documento non rilevante $q_i$.
Sostituendo questa ipotesi ho che all'interno del logaritmo c'è un 1 quando il termine non compare nella query e quindi il termine della sommatoria vale 0.

Pertanto posso ridurre il calcolo in:

\begin{align*}
P(rel|d,q) &\propto_q \sum\limits_{i \in d \cap q} \log \bigg( \frac{p_i}{1-p_i} \frac{1-q_i}{q_i} \bigg)
\end{align*}

\noindent Che è molto meno oneroso in termini di tempo computazionale, perché tipicamente la query utilizza 3 termini.
Questa è la versione di base del \textbf{Binary Indipendence Model} e funziona solo sotto le ipotesi che:

\begin{itemize}
	\item i termini sono una variabile binaria, o ci sono nel documento o non ci sono
	\item la presenza di un termine in un documento è indipendente dalla presenza di altre
	\item la probabilità che un termine presente nella query compaia nel documento è uguale a quella che non compaia
\end{itemize}

\noindent Questo modello è molto semplice ed efficiente, ma è molto importante perché è alla base degli altri modelli probabilistici.

Ovviamente la stima di $p_i$ e $q_i$ può essere effettuata come espresso nella sezione precedente e con i termini di normalizzazione, riconducendosi ancora una volta ad \textit{idf}.

$$
\boxed{ P(rel|d,q) \propto_q \sum\limits_{i \in d \cap q} \log \bigg( \frac{N_{t_iR} + 0.5}{N_R - N_{t_iR} + 0.5} \frac{N - N_R - N_{t_i} + N_{tR + 0.5}}{N_{t_i} + N_{t_iR} +0.5} \bigg) }
$$


\subsubsection{Interpretazione grafica del costo degli errori}

\begin{align*}
P(rel|d) \frac{(\lambda_{01} - \lambda_{11})}{(\lambda_{10} - \lambda_{00})} &> P(\overline{rel}|d) \\
\Lambda P(rel|d) &>1 -  P(rel|d) \\
(\Lambda +1 ) P(rel|d) &> 1 \\
P(rel |d ) &> \frac{1}{\Lambda +1}
\end{align*}

\noindent Se $\Lambda = 1$ ho che assegno la classe rilevante se la probabilità è maggiore di 0.5.

Se effettuare un'errore di classificazione ha un costo maggiore nel caso in cui si indichi un documento come non rilevante quando è rilavante, la soglia di accettazione tende a spostarsi verso destra ($> 0.5$).\\

Alternativamente posso vedere $P(rel|d)$ come la $X$ di un piano cartesiano e il complementare come la $Y$. 
Così facendo il valore di $\Lambda$ rappresenta il coefficiente angolare della retta passante per l'origine. Sotto la retta si accetta, sopra no e i valori di soglia vengono trovati con i punti di intersezione con la retta che passa per $(1,0)$ e $(0,1)$.
Questo però vale per le probabilità originali e che per motivi computazionali non sono calcolabili, perché per riuscire a calcolare le stime noi dobbiamo togliere il fattore di normalizzazione e prendere il logaritmo.


