% !TEX encoding = UTF-8
% !TEX program = pdflatex
% !TEX root = InformationRetrieval.tex
% !TEX spellcheck = it-IT

% 2 Dicembre 2016

\subsection{BM25}

$$
P(rel | d,q) sibolo tipo alpha P(d|rel, q)/P(d|not-rel, q) \approx \prod_{i \in V} P(TF=tf_i|rel,q)/P(tf=tf_i | not-rel, q)\approx P(TF = Tf_i|rel)/P(TF=Tf_i|not-rel)
$$

ma la produttoria non può essere calcolata perché lavora con numero troppo piccoli, quindi considero i logaritmi.

$$
\sum_q \underbrace{\log \Big( P(TF=Tf_i | rel) /P(TF=Tf_i \Big)}_{U(Tf_i)} = \sum_{q,tf_i > 0} U(Tf_i) + \sum_{q,Tf_i = 0} U(0)
$$

C'è un passaggio strano che l'autore l'ha motivato così: tutti i modelli prima del '76 partono dall'idea che ci sono le frequenze dei termini da sommare per calcolare lo score del documento. Ma non viene presa in considerazione l'assenza dei termini.
Questo per motivi di efficienza perché nel '76 la potenza di calcolo era ridotta.

$$
\sum_{q,tf_i > 0} U(Tf_i) + \sum_{q,Tf_i = 0} U(0) = \sum_{q,tf_i > 0} U(Tf_i) + \underbrace{\sum_{q,Tf_i = 0} U(0) + \sum_{q, tf > 0} U(0)}_{\text{possono essere raggruppate}} - \sum_{q, tq > 0} U(0) = \sum_{q, Tf>0} U(Tf_i) - U(0) + \underbrace{\sum_q U(0)}_{\text{è 0}}
$$

$$
P(Tf = 1 | REL) = p \quad P(Tf = 1 | not-rel) = q
$$

vado a sostituire sulla formula precedente e ottendo

$$
\sum \log p/q - \sum \log (1-p)/(1-q) = \sum \log p(1-q)/q(1-p)
$$

ovvero hanno ottenuto lo stesso risultato del modello binario.

...

$$
P(\lambda) = \lambda^k e^{-k} / k!
$$

Ci sono termini che appaiono poche volte $\mu$ e altri che ne compaiono molte $\lambda$. Si ha quindi che preso un documento a caso, la probabilità di osservare un termine rilevante è una mistura di $\mu$ e $\lambda$.

L'assunzione statistica è che tanto più sarà frequente sarà il termine nel documento, tanto più quel documento sarà specifico (\textit{d'elite}).

...

...

...

...

- non posso chiedervi di  calcolare delle formule con bm25

- posso chiedervi che data, una collezione di documenti con tf e rilevanza, calcolare qual'è il documento più rilevante (calcolo di log p/q (1-q)/(1-p), con numeri appropriati in modo che non serva la calcolatrice.

- posso chiedervi le ipotesi di partenza da 2-poisson a bm25. Interessa il ragionamento. Dato il formulone generale, quali sono le ipotesi semplificative che permettono di arrivare a bm25.

- dell'articolo sul bm25 interessano solo le prime








