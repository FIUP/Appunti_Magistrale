% !TEX encoding = UTF-8
% !TEX program = pdflatex
% !TEX root = InformationRetrieval.tex
% !TEX spellcheck = it-IT

\chapter{Anatomia e performance di un sistema di IR}

Nel nostro progetto noi non stiamo misurando lo stemmer di per se, ma stiamo misurando quanto lo stemmer migliora i risultati del sistema rispetto all'esecuzione senza stemming.

Si ha però che il sistema è composto da molti blocchi: tokenizzatore, stoplist, stemmer, modello e quindi non è semplice valutare il singolo componente, perché le prestazioni del sistema possono essere influenzate dagli altri componenti o dai parametri con i quali sono configurati.

Le misure nell'IR sono quindi misure end-to-end che considerano la totalità del sistema. Per valutare un singolo componente stabilisco quindi una pipeline di componenti e vario solamente quello che voglio misurare.

\section{Grid@CLEF}

L'idea è quindi quello di sviluppare un sistema di testing che funzioni con più lingue (MLIA: Multilingual Information Access) per capire come migliorare le performance dei singoli componenti.

Questo si può ottenere conducendo una serie di griglie di esperimenti, sistematici e ripetibili, su diversi linguaggi e con diversi componenti, effettuando uno sforzo comunitario per valutare sia i singoli componenti che la loro interazione con li altri.

Per implementare questo sistema è stata sviluppata una soluzione asincrona, basata su una produzione di dump intermedi XML, da scambiare tra i vari gruppi di ricerca. Ad esempio quelli che lavoro su un tokenizzatore, possono fornire i loro output ad un gruppi di ricerca che lavora agli stemmer per il tedesco.

Ci sono stati dei problemi implementativi però qualcosa si è riuscito a fare.

Mancava però una metodologia per analizzare i risultati di questi esperimenti.

\todo{aggiungere SIGIR RIGOR 2015}

\section{Valutazione delle performance - General Linear Mixed Model}

Posso combinare i dati dell'esecuzione dei vari sistemi della griglia in una matrice in cui le righe rappresentano i topic e le colonne sono i vari sistemi. Il valore delle cella è dato dall'average precision del sistema sul dato topic.
La matrice poi può essere plottata in una heat map, in modo che sia più comprensibile.

% immagine heatmap

Si può notare che la variazione è maggiore al variare dei topic, ovvero sistemi diversi tendono ad avere prestazioni simili sullo stesso topic.
Questo perché le performance del sistema dipendo sia dalla struttura del sistema, che dalla difficoltà del topic.

\todo{traduci slide 18 o 22 e copia le formule dalle slide successive}
















