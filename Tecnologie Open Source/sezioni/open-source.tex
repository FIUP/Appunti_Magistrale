\chapter{Il movimento Open Source}

\section{La storia del movimento Open Source}

\subsection{La cattedrale e il bazaar}

Un altro personaggio molto importante nel mondo del software libero è Eric Steven Raymond, informatico e programmatore statunitense di grande esperienza, che nel 1997 pubblica ``\textbf{La cattedrale e il bazaar}'', un saggio sullo sviluppo del software. Esso era essenzialmente destinato ad affrontare il problema del perché Linux, con il suo kernel, funzionasse così bene. Non era scontato che Linux funzionasse, anzi all'inizio, secondo la sua opinione, era destinato a ``scoppiare''. Il problema è che ognuno tendeva a pensare con la propria testa; eppure la cosa non succedeva. Quindi iniziò ad analizzare il fenomeno. 

Una delle prime cose che ha fatto è stato quello di iniziare un progetto, chiamato \textbf{popclient}, per l'invio e la ricezione della posta. Comincia dunque ad utilizzare tutta una serie di principi per la gestione di questo progetto per vedere se riusciva a ricreare il successo che aveva visto con Linux, un progetto che funzionasse bene e avesse una solida base di sviluppatori. Uno dei principi che aveva adottato era di trattare gli utenti come una sorta di \textbf{co-sviluppatori}, come se il programma fosse stato fatto insieme a loro. La comunità popclient divenne dunque molto attiva. 

Il secondo principio su cui basò il proprio lavoro fu: ``Distribuisci presto, distribuisci spesso e presta ascolto agli utenti'', in questo modo si favorisce la risoluzione di bachi in tempi brevi. Molti utenti si facevano avanti e miglioravano il software, sentendosi parte attiva del progetto. L'idea che secondo Raymond aveva fatto il successo di Linux era trasformare il software da uno sviluppo di una persona che ``dona agli altri'' a uno sviluppo ``social'', attorno al quale ruotava una comunità. Spesso la varietà delle persone porta ad una varietà di modi per risolvere il problema, vi sono approcci diversi, e la combinazione dei contributi può significare grossi miglioramenti.

L'articolo di Raymond ebbe una grossissima fortuna, l'impatto all'interno della comunità fu molto sentito, e l'effetto di questo fu che l'attenzione andò oltre la semplice comunità degli appassionati. Una delle conseguenze di questo successo fu che nel 1998 Netscape annunciò di voler rilasciare il codice sorgente del proprio browser, e disse che per decidere questo si era basato anche sull'articolo di Raymond. Una volta Netscape era dominante, prima dell'arrivo del colosso Microsoft con Internet Explorer. Il browser di casa Microsoft è sicuramente disprezzato al giorno d'oggi, ma in realtà per gli standard di allora era molto buono e la Microsoft era riuscita, partendo da zero, a recuperare terreno in tempi notevoli su Netscape. A Netscape non interessava moltissimo del suo navigatore, non poteva infatti guadagnarci (era gratis), ma focalizzava la sua attenzione sul mondo Server. Temeva però che se Internet Explorer fosse divenuto dominante a quel punto i propri server sarebbero stati ``scartati'' in favore di quelli di Microsft. Avrebbe perso una grossa fetta di mercato e buona parte del loro reddito. Ebbero dunque l'idea di rilasciare il codice sorgente di Netscape sotto una licenza libera. In questa decisione aveva avuto un ruolo importante Raymond, che divenne dunque una celebrità.

C'era però il rischio che il termine ``software libero'' si svalutasse o non acquisisse la giusta importanza. Ci fu dunque una riunione su come sfruttare al massimo l'annuncio di Netscape: nasce il termine ``\textbf{open source}''. L'idea era che il software libero garantiva software di maggiore qualità ai fini di offrire una piattaforma stabile, con una buona comunità e aperta. Nel 1998 nasce dunque la \textbf{Open Source Initiative}, fondata da Raymodn e Perens, che aveva come scopo quello di diffondere il mondo open source. In questo movimento entrarono a far parte tutti gli sviluppatori più importante, come RedHat. 

\subsection{Riassumendo}

Nel 1997 nasce il movimento open-source, Raymond pubblica ``\textit{The Cathedral and the Bazaar}'' (presentato alla O'Reilly Perl Conference). Raymond si mise ad analizzare la diffusione di Linux. Si crea un modello di sviluppo che invece di essere a cattedrale è a bazaar (come un mercato, in cui ci sono tante persone che contribuiscono tutte in modo diverso). Con questa nuova ottica si riesce dunque a produrre software in modo migliore.

Nel 1998 Netscape rilasciò il sorgente del proprio browser. Netscape all'epoca controllava il mercato dei browser ma era in crisi con l'incursione del colosso Microsoft e l'avvento di Internet Explorer. Temevano che la Microsoft sarebbe divenuta dominante e quindi rilasciarono il codice sorgente (primo grande rilascio della storia), il che costrinse inoltre loro ad un'ingente pulizia del codice (codice scritto male e con tanta spazzatura). Questo fu un passo epocale, infatti \textit{Firefox} nacque da questo. All'interno della comunità ci fu subito la percezione che questo rilascio fosse molto particolare e fosse qualcosa da sfruttare. L'interesse era creare un sistema funzionante e che si facesse rispettare. Nasce dunque il concetto di open-source e la OSI. Ci si rese conto che la produzione di software libero coincide con la produzione di software migliore. I grossi leader entrarono in questo movimento. Nel 2000 Sun richiese il codice sorgente di StarOffice. Mancava infatti un software di produttività che fosse all'altezza di ad esempio \textit{Microsoft Office}.

Per evitare che il termine venisse diluito e svalutato ci fu la necessità di renderlo un \textbf{marchio}, in modo che un software per essere considerato libero dovesse rispettare questo marchio ed essere approvato dalla OSI. 

\subsection{Open Source Definition}

Ci fu quindi la creazione di una serie di \textbf{linee guida} chiare per considerare un software come open-source. L'idea era quella di riutilizzare le Linee guide della Debian, le DFSG, in vigore dal 1997 rendendole più adatte in senso generale. Queste linee guida sono state sviluppate dall'interazione di decine di sviluppatori e sono state pensate in modo da non lasciare fuori tanto software libero importante già esistente.

\begin{itemize}
	
	\item Libera redistribuzione;
	\item Inclusione del codice sorgente;
	\item La licenza deve permettere modifiche e lavori derivati e deve permettere la loro distribuzione con i medesimi termini della licenza del software originale;
	\item Integrità del codice sorgente dall'autore, modifiche che esplicitino ``chi ha fatto cosa'', in modo che chiunque sia responsabile di ciò che scrive;
	\item Nessuna discriminazione di persone o gruppi;
	\item Nessuna discriminazione per i campi di impiego;
	\item La licenza dev'essere distribuita, per fare chiarezza;
	\item La licenza non può essere specifica per Debian;
	\item La licenza non deve contaminare altro software, non si possono imporre condizioni su software già esistente. La licenza dev'essere indipendente.
	
\end{itemize}


\section{La filosofia open source}

Questi principi, che sono delle vere e proprie clausole, sono tutti pragmatici: quello che conta è costruire software che sia affidabile e veloce in modo analogo a Linux:

\begin{itemize}

\item \textbf{Licenze libere e permissive}; quando una licenza è chiusa si tratta gli utenti non come co-sviluppatori ma come ``utenti di serie B''. È come dire: ``\textit{Vabbè, se proprio vuoi fammi il bug reporting...}'' oppure ``\textit{Tu sei talmente inutile per me che nemmeno ti aiuto ad aiutarti...}''. È un principio che in casa Microsoft va bene, perché l'obiettivo non è costruire una comunità. Nel caso del mondo open-source fare così è come ``darsi la mazza sui piedi'';
\item \textbf{Costruzione di una comunità attorno al software}; il software non è più una cosa che viene \textit{usata}, ma un modo di vivere, una cosa in cui le persone sono coinvolte. Con la crescita e lo sviluppo di una comunità attorno al software non solo i bachi vengono risolti più rapidamente, ma si hanno anche nuovi apporti mentali e contributi da parte delle persone;
\item \textbf{Trasparenza del processo di sviluppo}; la gente deve vedere quello che sta succedendo, perché una volta che lo vede magari contribuisce. Questo tipo di comunicazione è fondamentale;
\item \textbf{Codice sorgente liberamente disponibile}; altrimenti non possono esserci trasparenza e contributi da parte della comunità;
\item \textbf{Codice sorgente liberamente modificabile};
\item \textbf{Libera redistribuzione}, anche ad uso commerciale;

\end{itemize}

\section{Open Source Definition}

L'Open Source Definition è nata perché quando si è sviluppato il movimento OSI (\textit{Open Source Initiative}) ci si è reso conto che c'era il rischio, soprattutto nel momento in cui il movimento open source avesse avuto un forte impatto, che nella barca sarebbero saltati molti altri partecipanti ma non tutti avrebbero collaborato rispettando appieno le regole. Quindi era vitale decidere delle norme, delle linee guida da rispettare per scrivere del software open source. Esse erano pensate per escludere il minor numero di programmi già esistenti (esempio TEX, Perl, ...). Questo progetto ambiva a dare una caratterizzazione del software libero.

\subsection{Libera redistribuzione}

\begin{center}

\textit{La licenza non può limitare alcuno dal vendere o donare il software che ne è oggetto, come componente di una distribuzione aggregata, contenente programmi di varia origine. La licenza non può richiedere diritti o altri pagamenti a fronte di tali vendite.}

\end{center}

\textbf{Motivazione}: Imponendo la libera redistribuzione, si elimina la tentazione di rinunciare a importanti guadagni a lungo termine in cambio di un guadagno materiale a breve termine, ottenuto con il controllo delle vendite. Se non vi fosse questa imposizione, i collaboratori esterni sarebbero tentati di abbandonare il progetto, invece che di farlo crescere.

\subsection{Codice sorgente}

\begin{center}

\textit{Il programma deve includere il codice sorgente e ne deve essere permessa la distribuzione sia come codice sorgente che in forma compilata. Il codice sorgente deve essere il formato preferito per effettuare modifiche al codice.}

\end{center}

\textbf{Motivazione}: si richiede l'accesso al codice sorgente poichè non si può far evolvere un programma senza poterlo modificare. Il nostro obiettivo è rendere facile l'evoluzione del software, pertanto richiediamo che ne sia resa facile la modifica.

\subsection{Prodotti derivati}

\begin{center}

\textit{La licenza deve permettere modifiche e prodotti derivati, e deve permetterne la redistribuzione sotto le stesse condizioni della licenza del software originale.}

\end{center}

\textbf{Motivazione}: la sola possibilità di leggere il codice sorgente non è sufficiente a permettere la revisione indipendente del software da parte di terzi e una rapida selezione evolutiva. Per garantire una rapida evoluzione, deve essere possibile sperimentare modifiche al software e redistribuirle.

\subsection{Discriminazione contro persone o gruppi}

\begin{center}

\textit{La licenza non deve discriminare alcuna persona o gruppo di persone.}

\end{center}

\textbf{Motivazione}: per ottenere il massimo beneficio dal processo, il massimo numero di persone e gruppi deve avere eguale possibilità di contribuire allo sviluppo del software. Pertanto viene proibita l'esclusione arbitraria dal processo di persone o gruppi.

\subsection{Integrità del codice sorgente originale}

\begin{center}

\textit{La licenza può impedire la distribuzione del codice sorgente in forma modificata, a patto che venga consentita la distribuzione dell'originale accompagnato da ``patch'', ovvero file che permettono di applicare modifiche al codice sorgente in fase di compilazione.}

\end{center}

\textbf{Motivazione}: incoraggiare il miglioramento è bene, ma gli utenti hanno diritto di sapere chi è responsabile del software che stanno usando. Gli autori e i tecnici hanno diritto reciproco di sapere cosa è loro chiesto di supportare e di proteggersi la reputazione.

\subsection{Discriminazione per campo d'applicazione}

\begin{center}

\textit{La licenza non deve impedire di far uso del programma in un ambito specifico.}

\end{center}

\textbf{Motivazione}: L'intenzione principale di questa clausola è di proibire trappole nelle licenze che impediscano al software open source di essere usato commercialmente. Vogliamo che le aziende si uniscano alla nostra comunità, non che se ne sentano escluse.

\subsection{Distribuzione della licenza}

\begin{center}

\textit{I diritti allegati a un programma devono essere applicabili a tutti coloro a cui il programma è redistribuito, senza che sia necessaria l'emissione di ulteriori licenze.}

\end{center}

\textbf{Motivazione}: questa clausola intende proibire la chiusura del software per mezzi indiretti, come un obbligo di sottoscrizione di accordi di non diffusione.

\subsection{Specificità ad un prodotto}

\begin{center}

\textit{I diritti allegati al programma non devono dipendere dall'essere il programma parte di una particolare distribuzione del software.}

\end{center}

\textbf{Motivazione}: questa clausola impedisce un'ulteriore classe di licenze-trappola

\subsection{Vincoli su altro software}

\begin{center}

\textit{La licenza non deve porre restrizioni su altro software distribuito insieme al software licenziato.}

\end{center}

\textbf{Motivazione}: i distributori di software open source hanno il diritto di fare le loro scelte riguardo al software che intendono distribuire.

\subsection{Neutralità rispetto alle tecnologie}

\begin{center}

\textit{La licenza non deve contenere clausole che dipendano o si basano su particolari tecnologie o tipi di interfacce.}

\end{center}

\textit{Nota}: no al mouse click obbligatorio.

\section{Confronto con il Software Libero}

Il software Open Source ha dei criteri leggermente più deboli di quelli previsti per il software libero. Entrambi però sono accomunati dalla necessità che il codice sorgente sia disponibile e la maggior parte delle volte, il software open source è anche libero e viceversa.
Ci sono però delle licenze Open Source che non sono ritenute libere.

In ogni caso perché il software possa essere definito Open Source deve rispettare la OSD, mentre per essere libero deve rispettare le 4 libertà fondamentali del manifesto del software libero delle FSF e non può essere che uno dei due manifesti non venga rispettato, perché ad esempio esistono delle licenze Open che non prevedono il copyleft.

Tra i due movimenti c'è una differenza sostanziale riguardo le motivazioni. Il software libero punta di più sulla questione morale mentre il software Open da maggior peso ai vantaggi pratici.

Sempre nel lato pratico, l'Open Source viene preferito dalle aziende perché il termine ``\textit{free software}'' viene spesso associato al software gratuito.

