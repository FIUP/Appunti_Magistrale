% !TEX encoding = UTF-8
% !TEX program = pdflatex
% !TEX root = MEMOC.tex
% !TEX spellcheck = it-IT

% 13 Ottobre 2016
% Section Modellazione di un problema
% Subsection Un problema più complesso
% Subsubsection Modellazione 

\subsection{Alcune varianti del problema precedente}

Entrambe le varianti presenti risolte anche nel file \texttt{m01.modelli.00.en.pdf}, esercizio 1.

\subsubsection{Capacità dei camion limitata}

Una possibile variante di questo problema può essere l'aggiunta dei vincoli riguardo la capacità massima $K$ dei camion. Precedentemente era assunto che $K$ fosse sufficientemente alto per garantire che un camion fosse in grado si spostare tutto il necessario.

Per gestire questa situazione è necessario cambiare le variabili $y_{i,j} \in \{0,1\}$ in variabili intere, che rappresentano quanti camion servono in una determinata tratta.

Di conseguenza cambiano anche alcuni vincoli:

\begin{align*}
	\min &\sum\limits_{i \in I, j \in J} C_{i,j} x_{i,j} + F \sum\limits_{i \in I, j \in J} w_{i,j}+ (L-F)z \\
	&\vdots \\
	&x_{i,j} \leq K w_{i,j} \quad\forall i \in I, j \in J 
\end{align*}

\subsubsection{Costi fissi per il caricare i camion}

\`E necessario pagare un costo fisso $A_i$ per caricare le travi in $i \in I$.

Per modellare ciò viene utilizzata una variabile binaria che vale 1 se nel sito $i$ viene caricata della merce.

Le modifiche al modello riguardano la funzione obiettivo e i vincoli di attivazione per le nuove variabili $v_i \in {0,1}$:

\begin{align*}
\min &\sum\limits_{i \in I, j \in J} C_{i,j} x_{i,j} + F \sum\limits_{i \in I, j \in J} w_{i,j}+ (L-F)z +\sum\limits_{i \in I}A_i v_i \\
&\vdots \\
&\sum\limits_{j \in J} x_i,j \leq D_i v_i \quad\forall i \in I
\end{align*}

\subsection{Servizi di pronto intervento}\index{set covering}

Una rete di ospedali deve garantire in una certa area il servizio di pronto intervento. L'area è divisa in 6 zone e per ogni zona è stata individuata una sede locale per il servizio. La distanza media in minuti da ogni zona verso la sede locale è riportata nella seguente tabella.

\begin{table}[htbp]
	\centering
	\begin{tabular}{|l|l|l|l|l|l|l|}
		\hline
		& Loc. 1 & Loc. 2 & Loc. 3 & Loc. 4 & Loc. 5 & Loc. 6 \\ \hline
		Zona 1 & 5 & 10 & 20 & 30 & 30 & 20 \\ \hline
		Zona 2 & 10 & 5 & 25 & 35 & 20 & 10 \\ \hline
		Zona 3 & 20 & 25 & 5 & 15 & 30 & 20 \\ \hline
		Zona 4 & 30 & 35 & 15 & 5 & 15 & 25 \\ \hline
		Zona 5 & 30 & 20 & 30 & 15 & 5 & 14 \\ \hline
		Zona 6 & 20 & 10 & 20 & 25 & 14 & 5 \\ \hline
	\end{tabular}
\end{table}

\noindent \`E richiesto che ogni zona abbia una distanza media da una sede di 15 minuti. Gli ospedali chiedono dove aprire queste sedi e ne vogliono aprire il minor numero possibile.

\subsubsection{Modellazione}

Questo tipologia di modello prende il nome di \textbf{covering schema}. Per risolverlo basta porre dei vincoli che ogni area sia servita da almeno un sito. Nei vincoli relativi ad una determinata area vengono prese in considerazione solamente le sedi che riescono a soddisfare il vincolo dei 15 minuti.

\begin{itemize}
	\item Set $I = \{1,2, \ldots , 6 \}$ con le possibili sedi.
	\item Variabili $x_i \in \{0,1\} \: \forall\: i \in I$. Quando una variabile vale 1 viene aperta la sede nella relativa area.
\end{itemize}

\begin{align*}
\min &\sum\limits_{i \in I} x_i \\ 
\st &x_1 + x_2 \geq 1 &\text{Sedi che servono l'area 1} \\
	&x_1 + x_2 + x_6 \geq 1 &\text{Sedi che servono l'area 2} \\
	&x_3 + x_4 \geq 1 &\text{Sedi che servono l'area 3} \\
	&x_3 + x_4 + x_5 \geq 1 &\text{Sedi che servono l'area 4} \\
	&x_4 + x_5+x_6 \geq 1 &\text{Sedi che servono l'area 5} \\
	&x_2 + x_5 + x_6 \geq 1 &\text{Sedi che servono l'area 6} \\
\end{align*}

\subsection{TLC: localizzazione delle antenne}\index{TLC}

Una compagnia telefonica vuole installare delle antenne in alcuni siti per coprire 6 aree. Sono stati identificati 5 possibili siti per le antenne. Dopo alcune simulazioni, è stata stimata l'intensità del segnale proveniente dalle antenne posizionate nei vari siti e i risultati sono riportati nella tabella seguente

\begin{table}[htbp]
	\centering
	\begin{tabular}{|l|l|l|l|l|l|l|}
		\hline
		& Area 1 & Area 2 & Area 3 & Area 4 & Area 5 & Area 6 \\ \hline
		Sito A & 10 & 20 & 16 & 25 & 0 & 10 \\ \hline
		Sito B & 0 & 12 & 18 & 23 & 11 & 6 \\ \hline
		Sito C & 21 & 8 & 5 & 6 & 23 & 19 \\ \hline
		Sito D & 16 & 15 & 15 & 8 & 14 & 18 \\ \hline
		Sito E & 21 & 13 & 13 & 17 & 18 & 22 \\ \hline
	\end{tabular}
\end{table}

\noindent I ricevitori riconoscono solamente i segnali con intensità maggiore di 18. Inoltre, non è possibile avere in un area più di un segnare con intensità maggiore di 18, perché altrimenti ci sarebbero delle interferenze. Infine, un'antenna può essere messa in $E$, solo se è presente anche un'antenna in $D$.

La compagnia vuole determinare in quali siti posizionare le antenne in modo da coprire il maggior numero possibile di aree.

\subsubsection{Modellazione}

Il problema è simile a quello precedente, anche se ci sono dei vincoli leggermente diversi.

Ci sono un set $I$ con le possibili siti e un set $J$ con le possibili aree.
L'intensità del segnale nell'area $j \in J$ proveniente dall'antenna del sito $i \in I$ viene modellata dai parametri $\sigma_{i,j}$. C'è poi il parametro $T$ che rappresenta l'intensità minima del segnale (18) e $N$ che è il massimo numero di segnali che si possono sovrapporre in un'area (1).

La prima differenza con il problema precedente riguarda la funzione obiettivo, prima si voleva minimizzare il costo mentre adesso si vuole massimizzare la copertura. La seconda differenza riguarda i vincoli, che in questo caso sono relativi alla sovrapposizione del segnale e alla soglia minima.

La scelta, e quindi le variabili, riguarda dove andare a posizionare un'antenna, vengono quindi usate delle variabili binarie $x_i$ che valgono 1 se viene posizionata un'antenna nel sito $i$.

Tuttavia, utilizzando solo $x_i$ non si riesce ad esprimere bene la funzione obiettivo, serve quindi un altro gruppo di variabili binarie che indicano se una determinata area $j$ è coperta ($z_j$).

Il modello risulta quindi essere:

\begin{align*}
\max \quad & \sum\limits_{j \in J} z_j \\
\st & \sum\limits_{i \in I, \sigma_{i,j} \geq T} x_i \geq x_j \quad \forall \: j \in J &\text{(1)}\\
	& \sum\limits_{i \in I, \sigma_{i,j} \geq T} x_i \leq N + M_j(1-z_j) \quad \forall j \in J &\text{(2)} \\
	& x_d \geq x_e & \text{vincolo sui siti E e D}
\end{align*}

\noindent I vincoli (1) collegano le variabili relative alla copertura di una determinata area con le antenne che sono in grado di coprirla.
I vincoli (2) impongono che un'area sia coperta da al massimo $N$ segnali (1 per questa istanza del problema). 
La seconda parte di questi vincoli riguarda le aree che non siamo interessati a coprire, ovvero quelle aree per cui $z_j = 0$, e quindi se si verificano delle interferenze non ci sono problemi. Ciò funziona perché quando $z_j = 0$, il vincolo diventa $\sum x_i\leq N+M_j$ dove $M_j$ è un numero grande abbastanza da rendere il vincolo ridondante e non va a limitare lo spazio delle soluzioni.
Un valore ottimo per $M_j$ è la cardinalità dell'insieme dei siti che coprono l'area $j$, ovvero $M_j = | \: \{i \in I : \sigma_{i,j} \geq T \: |$.

La soluzione del problema si trova anche sul file \texttt{m01.modelli.00.en.pdf}, esercizio 2.

\subsection{La lettura dei giornali}\index{job shop scheduling}

Ci sono 4 amici che devono leggere 4 giornali. Ognuno vuole leggere i giornali in un determinato ordine. L'idea è di metterci il meno tempo possibile a leggere i giornali.

Il testo completo è nel file \texttt{m01.modelli.00.en.pdf}, esercizio 4.

\subsubsection{Modellazione}

Questo problema è un'istanza del \textbf{Job Shop Scheduling Problem}: ci sono dei \textit{jobs} (persone) che devono essere eseguiti da delle \textit{machine} (i giornali). Ogni \textit{job} richiede un certo tempo per essere processato ed è composto da vari step che devono essere eseguiti in ordine dalle varie macchine (lettura dei giornali). I vari \textit{jobs} sono anche caratterizzati da un tempo d'inizio.
L'obiettivo è quello di minimizzare il \textit{makespan}, che in questo caso coincide con il tempo necessario per leggere tutti i giornali. 

Ci sono quindi due set $I$ contenente le persone e $K$ i giornali.

Il problema poi ha vari parametri:
\begin{itemize}
	\item $D_{ik}$: tempo in minuti necessario alla persona $i$ per leggere il giornale $k$.
	\item $R_i$: dopo quanti minuti la persona $i$ si sveglia. Il tempo è espresso in minuti a partire dalle $8:30$.
	\item $M$: un numero sufficientemente grande da essere maggiore del makespan ottimo.
	\item $\sigma_{i,l}$: giornale letto dalla persona $i$ in posizione $l \in \{1,2,\ldots, |K| \}$. Questo parametro definisce la sequenza in cui devono essere letti i giornali dalla persona $i$. Da notare che $\sigma_{i,l} \in K$ e quindi il valore di uno di questi parametri può essere utilizzato come indice di un altro di questi parametri.
\end{itemize}

\noindent Le variabili necessario sono:
\begin{itemize}
	\item $h_{i,k}$: tempo in minuti a partire dalla $8:30$ al quale la persona $i$ inizia a leggere il giornale $k$.
	\item $y$: makespan.
	\item $x_{i,j,k}$: variabile binaria che vale 1 se la persona $i$ legge il giornale $k$ prima della persona $j$. Sono le variabili che determinano l'ordine di lettura dei giornali in una soluzione.
\end{itemize}

\noindent L'insieme dei vincoli risulta quindi essere:

\begin{align*}
\min \quad & y \\
\st & y \geq h_{i, \sigma_{i,|K|}} + D_{i, \sigma_{i,|K|}} \quad \forall \: i \in I &(1) \\
	& h_{i, \sigma_{i,l} } \geq h_{i, \sigma_{i,l-1} } + D_{i , \sigma_{i, l-1}} \quad \forall \:i \in I, l = 2 \ldots K &(2) \\
	& h_{i, \sigma_{i,1}} \geq R_i \quad \forall \: i \in I &(3) \\
	& h_{i,k} \geq h_{j,k} + D_{j,k} - M x_{i,j,k} \quad \forall i,j,k \: i \neq j &(4) \\
	& h_{j,k} \geq h_{i,k} + D_{i,k} - M (1 -x_{i,j,k} )\quad \forall i,j,k \: i \neq j &(5) \\
\end{align*}

\noindent Il significato degli insiemi di vincoli è:
\begin{enumerate}[(1)]
	\item Il makespan deve essere maggiore o uguale del tempo di completamento dell'ultima attività di ogni persona. In alte parole questi vincoli assicurano che tutti finiscano di leggere prima dell'istante $y$.
	\item Evita che due step della stessa attività si sovrappongano tra loro. Una persona non può leggere due giornali contemporaneamente.
	\item Il primo step non può essere effettuato prima dell'inizio dell'attività. Una persona non riesce a leggere mentre dorme.
	\item Impone che, se la persona $i$ legge il giornale $k$ prima della persona $j$, l'istante in cui $i$ inizia a leggere $k$ sia un'istante qualsiasi, mentre se $i$ non legge $k$ prima di $j$ ($x_{i,j,k}=0$), allora $i$ deve iniziare a leggere $k$ dopo che $j$ ha finito.
	\item Impone che, se la persona $i$ legge il giornale $k$ prima della persona $j$, l'istante in cui $j$ inizia a leggere $k$ sia successivo all'instante in cui $i$ finisce. Questo e il vincolo precedente sono mutamente esclusivi.
\end{enumerate}



