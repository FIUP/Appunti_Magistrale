% !TEX encoding = UTF-8
% !TEX program = pdflatex
% !TEX root = MEMOC.tex
% !TEX spellcheck = it-IT

% 7 Ottobre 2016

%chapter introduzione
%section informazioni pratiche

\chapter{Programmazione lineare}

\section{Modellazione di un problema}

Un \textbf{modello di programmazione matematico} descrive le caratteristiche della soluzione ottima di un problema di ottimizzazione sotto forma di relazione matematiche:

\begin{itemize}
	\item \textbf{Insiemi}: raggruppano gli elementi del sistema
	\item \textbf{Parametri}: i dati del problema, che rappresentano le quantità note che influiscono sul sistema.
	\item \textbf{Variabili decisionali}: le variabili per le quali si vuole trovare il valore per determinare la soluzione del problema.
	\item \textbf{Vincoli}: le relazioni matematiche che determinano se la soluzione è feasible o meno.
	\item \textbf{Funzione obiettivo}: la funzione che deve essere ottimizzata (massimizzata o minimizzata).
\end{itemize}

I modelli matematici nei quali la funzione obiettivo è \textbf{lineare} e anche i vincoli (constraints) sono un sistema di equazioni o disequazioni \textbf{lineari}, vengono detti lineari.
I modelli vengono poi classificati come:

\begin{itemize}
	\item \textbf{Linear programming model}: tutte le variabili possono assumere valori reali.
	\item \textbf{Integer Linear Programming model}: tutte le variabili possono assumere solo valori interi.
	\item \textbf{Mixed Integer Linear Programming}: le variabili possono assumere sia valori interi che reali.
\end{itemize}

La linearità limita l'espressività ma permette di utilizzare delle tecniche di soluzioni più veloci.

\subsubsection{Esempio di modellazione: I profumi}

Una fabbrica che produce profumi può produrre due nuove tipologie di profumi mescolando tre essente: rose, lilly e violet. Per ogni profumo 1 è necessario utilizzare 1.5 litri di lilly e 0.3 litri di violet. Per ogni profumo 2 è necessario utilizzare 1 litro di rose, 1 litro di lilly e 0.5 litri di viole.
Si hanno in magazzino 27 litri di rose, 21 litri di lilly e 9 litri di rose. La fabbrica ottiene un profitto di 130 euro per i profumi di tipo 1 e 100 per il tipo 2.
Si vuole determinare la quantità ottima di profumi da produrre per massimizzare i guadagni.

\begin{align*}
&\max 130 x_{one} + 100 x_{two} & & \quad\text{funzione obiettivo} \\
\text{s.t. } &1.5 x_{one} + x_{two} &\leq 27 &\\
					&x_{one} + x_{two} &\leq 21 &\\
					&0.3 x_{one} + 0.5 x_{two} &\leq 9 &\\
					&x_{one}, x_{two} &\geq 0 & \quad\text{dominio}
\end{align*}


% Gli altri modelli sono stati spostati in appendice
