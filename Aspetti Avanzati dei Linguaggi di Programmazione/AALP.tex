% !TEX encoding = UTF-8
% !TEX program = pdflatex
% !TEX root = AALP.tex
% !TEX spellcheck = it-IT
\documentclass[a4paper, 11pt]{report} % Font size (can be 10pt, 11pt or 12pt) and paper size (remove a4paper for US letter paper)
\usepackage[italian]{babel}      							% Lingua italiana
\usepackage[margin=.9in]{geometry}             % Imposta i margini del documento

\usepackage[T1]{fontenc} % Required for accented characters
\usepackage[mathletters]{ucs}    % Caratteri matematici come UTF8
\usepackage[utf8,utf8x]{inputenc}      % Ancora utf8

\usepackage{eurosym}                %simbolo dell'euro
\usepackage{listings}
\usepackage[usenames,dvipsnames,svgnames,table]{xcolor}
% Imposta lo spazio nella list of listing in modo simile alla list of figures/tables
%\makeatletter
%\let\my@chapter\@chapter
%\renewcommand*{\@chapter}{%
%  \addtocontents{lol}{\protect\addvspace{10pt}}%
%  \my@chapter}
%\makeatother


\definecolor{codegreen}{rgb}{0,0.6,0}
\definecolor{codegray}{rgb}{0.5,0.5,0.5}
\definecolor{backcolor}{rgb}{0.98,0.98,0.98}

\renewcommand{\lstlistingname}{Codice}% Listing -> codice
\renewcommand{\lstlistlistingname}{Elenco dei frammenti di codice}% List of Listings -> Frammenti di codice

\lstdefinestyle{mystyle}{
    backgroundcolor=\color{backcolor},   
    commentstyle=\color{Peach}\ttfamily,
    keywordstyle=\color{RoyalBlue},
    numberstyle=\tiny\color{codegray},
    stringstyle=\color{SeaGreen}\ttfamily,
    basicstyle=\footnotesize\ttfamily,
    breakatwhitespace=false,         
    breaklines=true,                 
    captionpos=b,                    
    keepspaces=true,                 
    numbers=left,                    
    numbersep=5pt,                  
    showspaces=false,                
    showstringspaces=false,
    showtabs=false,                  
    tabsize=2,
    frame=trbl, % draw a frame at the top, right, left and bottom of the listing
	frameround=ftff, % angolo in basso a destro curvo
	framesep=4pt, % quarter circle size of the round corners,
	inputencoding=utf8,
    extendedchars=true,
    literate={á}{{\'a}}1 {à}{{\`a}}1 {é}{{\'e}}1 {è}{{\`e}}1 {ù}{{\`u}}1 {ò}{{\`o}}1 {ì}{{\`i}}1,
    belowskip=1em,
    aboveskip=1em,
}

 
\lstset{style=mystyle}

\lstdefinelanguage{JavaScript}
{
  % list of keywords
  morekeywords={ true, false, catch, function, break,	new, class, extends, var, require, switch, return, import, if, while, for, this, View, Text, StyleSheet},
  sensitive=false, % keywords are not case-sensitive
  morecomment=[l]{//}, % l is for line comment
  morecomment=[s]{/*}{*/}, % s is for start and end delimiter
  morestring=[b]' % defines that strings are enclosed in double quotes
}

\lstdefinelanguage{JSON}
{
  % list of keywords
  morekeywords={string, boolean, int, Array, Node, Asset, AssetDetail, Filter, FilterItem},
  sensitive=false, % keywords are not case-sensitive
  morecomment=[l]{//}, % l is for line comment
  morecomment=[s]{/*}{*/}, % s is for start and end delimiter
  morestring=[b]" % defines that strings are enclosed in double quotes
}

\lstdefinelanguage{URM}
{
	% list of keywords
	morekeywords={ S, J, T, Z, I},
	sensitive=false, % keywords are not case-sensitive
	morecomment=[l]{//}, % l is for line comment
	morecomment=[s]{/*}{*/}, % s is for start and end delimiter
	morestring=[b]' % defines that strings are enclosed in double quotes
}

\lstdefinelanguage{RDFA}{
	language=html,
	sensitive=true, 
	alsoletter={<>=-},
	ndkeywords={
		% General
		=,
		% HTML attributes
		charset=, id=, width=, height=, property=, about=, rel=, rev=, prefix=, vocab=, content=, datatype=
	},  
	morecomment=[s]{<!--}{-->},
	tag=[s]
}

%\tightlist per compatibilità con pandoc
\providecommand{\tightlist}{%
	\setlength{\itemsep}{0pt}\setlength{\parskip}{0pt}}


\usepackage[labelfont=bf]{caption}

\usepackage[protrusion=true,expansion=true]{microtype} % Better typography
\usepackage{graphicx} % Required for including pictures
\usepackage{wrapfig} % Allows in-line images


\usepackage{subfig}
\usepackage{hyperref}
\usepackage{placeins}
\usepackage{sourcecodepro}
\usepackage{hyperref}                   % collegamenti ipertestuali

\usepackage[colorinlistoftodos,prependcaption]{todonotes} %todo

\usepackage{amsmath}
\usepackage{mathtools}

\usepackage{float}
\usepackage{algorithm}
\usepackage{algpseudocode} % https://en.wikibooks.org/wiki/LaTeX/Algorithms#Typesetting_using_the_algorithmicx_package
\usepackage{amssymb}  %$\mathbb{N}$ per il simbolo dei numeri naturali 

\usepackage{enumerate} % permette di personalizzare enumerate

\usepackage{xmpincl}	%Aggiunge metadati sulla licenza CC
\usepackage{xspace}
\makeatletter
\renewcommand\@biblabel[1]{\textbf{#1.}} % Change the square brackets for each bibliography item from '[1]' to '1.'
\renewcommand{\@listI}{\itemsep=0pt} % Reduce the space between items in the itemize and enumerate environments and the bibliography

\renewcommand{\maketitle}{ % Customize the title - do not edit title and author name here, see the TITLE block below
	\begin{flushright} % Right align
		{\LARGE\@title} % Increase the font size of the title
		
		\vspace{50pt} % Some vertical space between the title and author name
		
		{\large\@author} % Author name
		\\\@date % Date
		
		\vspace{100pt} % Some vertical space between the author block and abstract
	\end{flushright}
}

%% breakablealgorithm http://tex.stackexchange.com/questions/33866/algorithm-tag-and-page-break
\makeatletter
\newenvironment{breakablealgorithm}
{% \begin{breakablealgorithm}
	\begin{center}
		\refstepcounter{algorithm}% New algorithm
		\hrule height.8pt depth0pt \kern2pt% \@fs@pre for \@fs@ruled
		\renewcommand{\caption}[2][\relax]{% Make a new \caption
			{\raggedright\textbf{\ALG@name~\thealgorithm} ##2\par}%
			\ifx\relax##1\relax % #1 is \relax
			\addcontentsline{loa}{algorithm}{\protect\numberline{\thealgorithm}##2}%
			\else % #1 is not \relax
			\addcontentsline{loa}{algorithm}{\protect\numberline{\thealgorithm}##1}%
			\fi
			\kern2pt\hrule\kern2pt
		}
	}{% \end{breakablealgorithm}
	\kern2pt\hrule\relax% \@fs@post for \@fs@ruled
\end{center}
}
\makeatother

\makeatletter % trattino con punto sopra
\newcommand{\dotminus}{\mathbin{\text{\@dotminus}}}

\newcommand{\@dotminus}{%
	\ooalign{\hidewidth\raise1ex\hbox{.}\hidewidth\cr$\m@th-$\cr}%
}
\makeatother

\DeclarePairedDelimiter{\ceil}{\lceil}{\rceil}
\DeclarePairedDelimiter{\floor}{\lfloor}{\rfloor}

%----------------------------------------------------------------------------------------
% TITLE
%----------------------------------------------------------------------------------------

\title{\textbf{Aspetti Avanzati dei Linguaggi di Programmazione}\\ % Title
	A.A. 2016-2017 } % Subtitle

\author{\textsc{Giacomo Manzoli}
	\\ 1130822 % Author
	\\{\textit{Università degli Studi di Padova}}} % Institution

\date{\today} % Date

%----------------------------------------------------------------------------------------


%----------------------------------------------------------------------------------------
%	DOCUMENT HEADER
%----------------------------------------------------------------------------------------

\begin{document}
	
	\maketitle % Print the title section

	%----------------------------------------------------------------------------------------
	% ABSTRACT AND KEYWORDS
	%----------------------------------------------------------------------------------------
	
	%\renewcommand{\abstractname}{Summary} % Uncomment to change the name of the abstract to something else
	
	\clearpage
	\tableofcontents
	
	%\hspace*{3,6mm}\textit{Keywords:} lorem , ipsum , dolor , sit amet , lectus % Keywords
	
	\vspace{30pt} % Some vertical space between the abstract and first section
	
	%----------------------------------------------------------------------------------------
	% ESSAY BODY
	%----------------------------------------------------------------------------------------
	\clearpage
	
	%----------------------------------------------------------------------------------------
	%	CONTENT
	%----------------------------------------------------------------------------------------
	
	% !TEX encoding = UTF-8
% !TEX TS-program = pdflatex
% !TEX root = computabilità e algoritmi.tex
% !TEX spellcheck = it-IT
\chapter{Funzioni calcolabili e Modelli di calcolo}
\section{Introduzione}\label{lezione-1---computabilituxe0-e-algoritmi}

Ci sono dei problemi che non possono essere risolti in modo algoritmico,
come la terminazione o la prova di correttezza di un programma, lo studio di questi problemi prende il nome di teoria della computabilità.

In questa teoria non viene preso in considerazione il consumo di
risorse in modo che le dimostrazioni effettuate siano indipendenti dal
modello di calcolo adottato.

Notoriamente, i problemi appartengono a varie classi di difficoltà:

\begin{itemize}
\item
  \textbf{P}: problemi che possono essere risolti da un algoritmo in
  tempo polinomiale
\item
  \textbf{NP}: problemi che possono essere risolti in tempo polinomiale
  ma in modo non deterministico
\item
  \textbf{EXP}: problemi che possono essere risolti da un algoritmo in
  tempo esponenziale
\end{itemize}

\subsection{L'informatica e la computabilità}\label{linformatica-e-la-computabilituxe0}

\emph{Computer science is no more about computers tha astronomy is about
telescopes. Dijkstra}.

L'idea dell'informatica nasce dalla logica, ricercando un procedimento
generale (macchina) su base combinatoria per trovare tutte le verità.

Libro: \emph{Nigel Cutland ``Computability. An Introduction to Recursive
Function Theory'' Cambridge University Press}.

	
	%\appendix
	
	%%\renewcommand{\glossaryname}{Glossario}

%\newglossaryentry{Cordova}
%{
%	name=\glslink{Cordova}{Cordova},
%	text=Cordova,
%	sort=Cordova,
%	description={Apache Cordova è un framework open source per la realizzazione di applicazioni ibride che offre delle API che permettono di accedere via JavaScript ad alcune funzionalità native del dispositivo, come l'accelerometro o la fotocamera}
%}
\subsection*{A}

\underline{\textbf{ARPA}}: %TODO

\underline{\textbf{ARPANET}}: %TODO

\underline{\textbf{Assembler}}: %TODO

\subsection*{B}

\underline{\textbf{Beowulf}}: %TODO

\subsection*{C}

\subsection*{D}

\underline{\textbf{Debian}}: %TODO

\underline{\textbf{DFSG}}: %TODO

\subsection*{E}

\subsection*{F}

\underline{\textbf{Free Software Foundation}}: %TODO

\underline{\textbf{Freeware}}: %TODO

\underline{\textbf{fsf}}: %TODO

\subsection*{G}

\underline{\textbf{GNU}}: %TODO

\underline{\textbf{GPL}}: %TODO

\subsection*{H}

\subsection*{I}

\subsection*{J}

\subsection*{K}

\underline{\textbf{Kernel}}: %TODO

\subsection*{L}

\underline{\textbf{Linux}}: %TODO

\subsection*{M}

\underline{\textbf{Mimix}}: %TODO

\underline{\textbf{MUTIX}}: %TODO

\subsection*{N}

\underline{\textbf{Netscape}}: %TODO

\underline{\textbf{nslu2}}: %TODO

\subsection*{O}

\underline{\textbf{OSI}}: %TODO

\subsection*{P}

\underline{\textbf{PDP}}: %TODO

\subsection*{Q}

\subsection*{R}

\underline{\textbf{RedHat Enterprise Linux}}: %TODO

\underline{\textbf{Routes}}: %TODO



\subsection*{S}

\underline{\textbf{StarOffice}}: %TODO

\underline{\textbf{Sun}}: 

\underline{\textbf{S\&P}}: %TODO

\underline{\textbf{Shareware}}: %TODO

\underline{\textbf{Symbolics}}: %TODO	

\subsection*{T}

\underline{\textbf{TECO}}: %TODO

\subsection*{U}

\underline{\textbf{Unix}}: %TODO

\subsection*{V}

\subsection*{W}

\subsection*{X}

\subsection*{Y}

\subsection*{Z}


	




	
	
\end{document}