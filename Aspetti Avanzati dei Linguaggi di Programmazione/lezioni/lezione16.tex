% !TEX encoding = UTF-8
% !TEX program = pdflatex
% !TEX root = AALP.tex
% !TEX spellcheck = it-IT

% 29 Novembre 2016

%\section{OOP meets FP}

Il pattern viene utilizzato anche per la gestione delle funzioni.


Le funzioni vengono implementate come oggetti e per ogni arietà è definita una classe che le rappresenta. Non sono proprio classi, ma \texttt{trait}, una sorta di interfacce.

\begin{lstlisting}[language=Scala]
trait Function1[-A, +B] {
	def apply(x:A) : B
}
\end{lstlisting}

Nella definizione dei parametri di tipo sono presenti anche dei simboli \texttt{+} e \texttt{-}. Si tratta di due operatori che permetto di annotare la varianza dei tipi. Con questa configurazione si a che:

\begin{center}
\texttt{Fun[A',B']} $<:$ \texttt{Fun[A,B]} se $A <: A'$ e $B' <: B$
\end{center}

Ovvero il sub-typing per le funzioni segue la classica definizione.

Dato che le funzioni sono classi, potrei anche pensare di definire un array come una funzione.


\section{Static vs dynamic typing}

In Scala viene adottato un sistema di typing statico, ma la sua verbosità viene ridotta dalla type-inference effettuata dal compilatore.

Tuttavia si può anche introdurre il dynamic typing per alcune classi. Ciò viene fatto utilizzando il trait \texttt{Dynamic}. Così facendo, quando il compilatore deve valutare la chiamata su un riferimento \texttt{Dynamic}, prova a fare la verifica statica, ma se non ci riesce predispone un controllo a run-time della presenza.
Ad esempio la chiamata \texttt{ref.sel(arg)} viene riscritta in byte-code come \texttt{ref.applyDynamic("sel")(args)}.

\chapter{Java 8}

Con Java 8 sono state aggiunte in Java delle caratteristiche della programmazione funzionale, come la possibilità di definire delle funzioni anonime. 

Queste caratteristiche permettono di rinnovare il framework delle collezioni, ad esempio per supportare la parallelizzazione delle operazioni sulle collezioni.

Ad esempio per effettuare l'iterazione con il for-each si può utilizzare la valutazione \textbf{interna} della collezione.

\begin{lstlisting}[language=Java]
Collection<Person> people = ...
int maxAge = people.stream()
                          .filter(p -> p.getGender() == MALE)
                          .mapToInt(p -> p.getAge())
                          .max();
\end{lstlisting}

Da notare che così facendo non si perde efficienza, perché ad ogni chiamata non viene creata una nuova collezione, ma viene effettuata una valutazione lazy che può anche essere parallelizzata.

Per ottenere questo risultato è stato necessario aggiungere al linguaggio:

\begin{itemize}
	\item Lambda Expression
	\item Stream
	\item Method References
	\item Type inferece
	\item Default methods
	\item Functional Interfaces
\end{itemize}

Il tutto mantenendo la retro-compatibilità con le precedenti versioni.

\section{Lambda Expressions}

Non è stato possibile aggiungere un tipo freccia in Java perché altrimenti si sarebbe persa la retro-compatibilità.

Pertanto è stato necessario definire delle interfacce che hanno un solo metodo come

\begin{lstlisting}
interface Fun {double apply(int i);}
\end{lstlisting}

Tutte queste interfacce sono definite nel package \texttt{java.util.function} e prendono il nome di \textbf{functional interfaces}.
Da notare che anche le interfacce già esistenti, come \texttt{Runnable}, rientrano in questa categoria.

\begin{lstlisting}[language=Java]
public interface Function<T,R> {R apply(T t);}
public interface Predicate<T> {boolean test(T t);}
public interface Supplier<T> {T get();}
...
public interface Runnable {public void run();}
\end{lstlisting}

Si perde però il sub-typing diretto tra i tipi freccia, ma si può ottenere qualcosa di simile utilizzando le wild-card per le classi generiche.

Alcuni esempi:

\begin{lstlisting}[language=Java]
ToIntFunction<Integer> f1 = z -> z*10;
int x1 = f1.applyAsInt(18);

ToIntBiFunction<Integer, Integer> f2 = (x,y) -> x -y; // C'è un maggior lavoro di type-inference
int x2 = f2.applyAsInt(20,4);

BiPredicate<String, String> f3 = (s1,s2) -> s1.equals(s2);
boolean x3 = f3.test("pippo", "pluto");
\end{lstlisting}

Da notare che con questa definizione anche le interfacce dei vari action listener possono essere istanziate con delle lamba expression.

\begin{lstlisting}[language=Java, caption=Da notare che la variabile frame della lamba expression è libera e prende un valore dal contesto di invocazione del metodo.]
button.addActionListener(event -> JOptionPane.showMessageDialog(frame, "Hai cliccato"));

// mentre prima bisognava scrivere
button.addActionListener(new ActionListener() {
	public void actionPerformed(ActionEvent event) {
		JOptionPane.showMessageDialog(frame, "Hai cliccato");
	}
})
\end{lstlisting}

Con le lambda expression possono essere utilizzate delle variabili libere, che utilizzando i riferimenti del contesto di esecuzione per dare un valore ai riferimenti liberi (\textbf{Clojure}).

Lo stesso vale per Scala. Ad esempio sul REPL di Scala si possono fare le seguenti operazioni:

\begin{lstlisting}[language = Scala]
> (x : Int) => x + more
> Error: not found : value more

> val more = 1
more : Int = 1 
> val addMore = (x: Int) => x + more

// todo copiare anche l'esempio con var
\end{lstlisting}

Sia in Java che in Scala, le \textbf{clojure}, ovvero l'espressione e i riferimenti all'ambienti di esecuzione vengono tenuti assieme. In particolare i riferimenti non sono dereferenziati ma rimangono tali. Quindi se in una clojure vengono utilizzate delle \texttt{var} esterne 


